% -*- latex -*-
%%%%%%%%%%%%%%%%%%%%%%%%%%%%%%%%%%%%%%%%%%%%%%%%%%%%%%%%%%%%%%%%
%%%%
%%%% This TeX file is part of the course
%%%% Introduction to Scientific Programming in C++/Fortran2003
%%%% copyright 2017 Victor Eijkhout eijkhout@tacc.utexas.edu
%%%%
%%%% topdown.tex : chapter of preliminary stuff
%%%%
%%%%%%%%%%%%%%%%%%%%%%%%%%%%%%%%%%%%%%%%%%%%%%%%%%%%%%%%%%%%%%%%


\Level 0 {A philosophy of programming}

\begin{block}{Code for the reader, not the writer}
  Yes, your code will be executed by the computer, but:
  \begin{itemize}
  \item You need to be able to understand your code a month or year
    from now.
  \item Someone else may need to understand your code.
  \item $\Rightarrow$ make your code readable, not just efficient
  \end{itemize}
\end{block}

\begin{block}{High level and low level}
  \begin{itemize}
  \item Don't waste time on making your code efficient, until you know
    that that time will actually pay off.
  \item Knuth: `premature optimization is the root of all evil'.
  \item $\Rightarrow$ first make your code correct, then worry about efficiency
  \end{itemize}
\end{block}

\begin{block}{Abstraction}
  \begin{itemize}
  \item Variables, functions, objects, form a new `language':\\
    code in the language of the application.
  \item $\Rightarrow$ your code should look like it talks about the
    application, not about memory.
  \item Levels of abstraction: implementation of a language should not be
    visible on the use level of that language.
  \end{itemize}
\end{block}

\Level 0 {Programming: top-down versus bottom up}

The exercises in chapter~\ref{ch:prime} were in order of increasing
complexity. You can imagine writing a program that way, which is
formally known as \indexterm{bottom-up} programming.

However, to write a sophisticated program this way you really need to
have an overall conception of the structure of the whole
program.

Maybe it makes more sense to go about it the other way: start with the
highest level description and gradually refine it to the lowest level
building blocks. This is known as \indexterm{top-down} programming.

\url{https://www.cs.fsu.edu/~myers/c++/notes/stepwise.html}

Example:\\
\begin{tabbing}
  Run a simulation
\end{tabbing}
becomes
\begin{tabbing}
  Run a \=simulation:\\
  \>Set up \=data and parameters\\
  \>Until convergence:\\
  \>\>Do a time step
\end{tabbing}
becomes
\begin{tabbing}
  Run a \=simulation:\\
  \>Set up \=data and parameters:\\
  \>\>Allocate data structures\\
  \>\>Set all values\\
  \>Until convergence:\\
  \>\>Do \=a time step:\\
  \>\>\>Calculate Jacobian\\
  \>\>\>Compute time step\\
  \>\>\>Update
\end{tabbing}

You could do these refinement steps on paper and wind up with the
finished program, but every step that is refined could also be a
subprogram.

We already did some top-down programming, when the prime number
exercises asked you to write functions and classes to implement a
given program structure; see for instance exercise~\ref{ex:prime:sequence}.

A problem with top-down programming is that you can not start testing
until you have made your way down to the basic bits of code. With
bottom-up it's easier to start testing. Which brings us to\ldots

\Level 1 {Worked out example}

Take a look at exercise~\ref{ex:collatz}. We will solve this in steps.
\begin{enumerate}
\item State the problem:
\begin{verbatim}
// find the longest sequence
\end{verbatim}
\item Refine by introducing a loop
\begin{verbatim}
// find the longest sequence:

// Try all starting points
// If it gives a longer sequence report
\end{verbatim}
\item Introduce the actual loop:
\begin{verbatim}
// Try all starting points
for (int starting=2; starting<1000; starting++) {
// If it gives a longer sequence report
}
\end{verbatim}
\item Record the length:
\begin{verbatim}
// Try all starting points
int maximum_length=-1;
for (int starting=2; starting<1000; starting++) {
  // If the sequence from `start' gives a longer sequence report:
  int length=0;
  // compute the sequence from `start'
  if (length>maximum_length) {
    // Report this sequence as the longest
  }
}
\end{verbatim}
\item Refine computing the sequence:
\begin{verbatim}
// compute the sequence from `start'
int current=starting;
while (current!=1) {
  // update current value
  length++;
}
\end{verbatim}
\item Refine the update of the current value:
\begin{verbatim}
// update current value
if (current%2==0)
  current /= 2;
else
  current = 3*current+1;
\end{verbatim}
\end{enumerate}

\Level 0 {Coding style}

After you write your code there is the issue of
\indextermbus{code}{maintainance}: you may in the future have to
update your code or fix something. You may even have to fix someone
else's code or someone will have to work on your code. So it's a good
idea to code cleanly.

\begin{description}
\item[Naming] Use meaningful variable names: \n{record_number} instead
  \n{rn} or \n{n}. This is sometimes called `self-documenting code'.
\item[Comments] Insert comments to explain non-trivial parts of code.
\item[Reuse] Do not write the same bit of code twice: use macros,
  functions, classes.
\end{description}

\Level 0 {Documentation}

Take a look at Doxygen.

\Level 0 {Testing}
\index{testing}

If you write your program modularly, it is easy (or at least: easier)
to test the components without having to wait for an all-or-nothing
test of the whole program. In an extreme form of this you would write
your code by \indexterm{test-driven development}: for each part of the
program you would first write the test that it would satisfy.

In a more moderate approach you would use \indexterm{unit testing}:
you write a test for each program bit, from the lowest to the highest
level.

And always remember the old truism that `by testing you can only prove
the presence of errors, never the absence.

\Level 0 {Best practices: C++ Core Guidelines}
\index{C++!Core Guidelines|(textbf}

The C++ language is big, and some combinations of features are not
advisable. Around 2015 a number of \emph{Core Guidelines} were drawn
up that will greatly increase code quality. Note that this is not
about performance: the guidelines have basically no performance
implications, but lead to better code.

For instance, the guidelines recommend to use default values as much
as possible when dealing with multiple constructors:
\begin{lstlisting}
class Point { // not this way
private:
  double d;
public:
  Point( double x,double y,double fudge ) {
    auto d = ( x*x + y*y ) * (1+fudge); };
  Point( double x,double y ) {
    auto d = ( x*x + y*y ); };
};
\end{lstlisting}
This is bad because of code duplication. Slightly better:
\begin{lstlisting}
class Point { // not this way
private:
  double d;
public:
  Point( double x,double y,double fudge ) {
    auto d = ( x*x + y*y ) * (1+fudge); };
  Point( double x,double y ) : Point(x,y,0.) {};
};
\end{lstlisting}
which wastes a couple of cycles if fudge is zero.
Best:
\begin{lstlisting}
class Point { // not this way
private:
  double d;
public:
  Point( double x,double y,double fudge=0. ) {
    auto d = ( x*x + y*y ) * (1+fudge); };
};
\end{lstlisting}

\index{C++!Core Guidelines|)}

