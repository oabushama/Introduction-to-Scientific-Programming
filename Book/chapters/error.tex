% -*- latex -*-
%%%%%%%%%%%%%%%%%%%%%%%%%%%%%%%%%%%%%%%%%%%%%%%%%%%%%%%%%%%%%%%%
%%%%
%%%% This TeX file is part of the course
%%%% Introduction to Scientific Programming in C++/Fortran2003
%%%% copyright 2017/8 Victor Eijkhout eijkhout@tacc.utexas.edu
%%%%
%%%% error.tex : on error handling
%%%%
%%%%%%%%%%%%%%%%%%%%%%%%%%%%%%%%%%%%%%%%%%%%%%%%%%%%%%%%%%%%%%%%

\Level 0 {General discussion}

Sources or errors:
\begin{itemize}
\item Array indexing. See section~\ref{sec:stdvector}.
\item Null pointers
\item Division by zero and other numerical errors.
\end{itemize}

Guarding against errors.
\begin{itemize}
\item Check preconditions.
\item Catch results.
\item Check postconditions.
\end{itemize}

Error reporting:
\begin{itemize}
\item Message
\item Total abort
\item Exception
\end{itemize}

Assertions:
\begin{verbatim}
#include <cassert>
...
assert( bool )
\end{verbatim}
assertions are omitted with optimization

Function return values

\Level 0 {Exception handling}

\begin{block}{Exception throwing}
  \label{sl:exception-throw}
  \emph{Throwing} an \emph{exception}%
  \index{exception!throwing} is one way of signalling an error or
  unexpected behaviour:
\begin{verbatim}
void do_something() {
  if ( oops )
    throw(5);
}
\end{verbatim}
\end{block}

\begin{block}{Catching an exception}
  \label{sl:exception-catch}
  It now becomes possible to detect this unexpected behaviour by
  \emph{catching}\index{exception!catch}
  the exception:
\begin{verbatim}
throw {
  do_something();
} catch (int i) {
  cout << "doing something failed: error=" << i << endl;
}
\end{verbatim}
\end{block}

You can throw integers to indicate an error code, a string with an
actual error message. You could even make an error class:

\begin{block}{Exception classes}
  \label{sl:exception-class}
\begin{verbatim}
class MyError {
public :
  int error_no; string error_msg;
  MyError( int i,string msg )
  : error_no(i),error_msg(msg) {};
}

throw( MyError(27,"oops");

try {
  // something
} catch ( MyError &m ) {
  cout << "My error with code=" << m.error_no
    << " msg=" << m.error_msg << endl;
}
\end{verbatim}
\end{block}

\begin{block}{Multiple catches}
  \label{sl:exception-catches}
  You can multiple \n{catch} statements to catch different types of
  errors:
\begin{verbatim}
try {
  // something
} catch ( int i ) {
  // handle int exception
} catch ( std::string c ) {
  // handle string exception
}
\end{verbatim}
\end{block}

\begin{block}{Catch-all}
  \label{sl:exception-catchall}
  Catch all exceptions:
\begin{verbatim}
try {
  // something
} catch ( ... ) { // literally: three dots
  cout << "Something went wrong!" << endl;
}
\end{verbatim}
\end{block}

\begin{block}{More about exceptions}
  \label{sl:exception-more}
  \begin{itemize}
  \item Functions can define what exceptions they throw: 
\begin{verbatim}
void func() throw( MyError, std::string );
void funk() throw();
\end{verbatim}
\item Predefined exceptions: \indextermtt{bad_alloc},
  \indextermtt{bad_exception}, etc.
\item An exception handler can throw an exception; to rethrow the same
  exception use `\n{throw;}' without arguments.
\item Exceptions delete all stack data, but not \n{new} data.
  \end{itemize}
\end{block}
