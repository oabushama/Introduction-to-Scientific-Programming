% -*- latex -*-
%%%%%%%%%%%%%%%%%%%%%%%%%%%%%%%%%%%%%%%%%%%%%%%%%%%%%%%%%%%%%%%%
%%%%
%%%% This TeX file is part of the course
%%%% Introduction to Scientific Programming in C++/Fortran2003
%%%% copyright 2017-2020 Victor Eijkhout eijkhout@tacc.utexas.edu
%%%%
%%%% classf.tex : object oriented programming in Fortran
%%%%
%%%%%%%%%%%%%%%%%%%%%%%%%%%%%%%%%%%%%%%%%%%%%%%%%%%%%%%%%%%%%%%%

\Level 0 {Classes}
\label{sec:objectf}

\begin{block}{Classes and objects}
  \label{sl:fclass}
Fortran classes are based on \indextermfort{type} objects, a little like the
analogy between C++ \n{struct} and \n{class} constructs.

New syntax for specifying methods.
\end{block}

\begin{block}{Object is type with methods}
  \label{sl:fclass-prog}
  You define a type as before, with its data members, but now the type
  has a \indextermfort{contains}\index{contains!for class functions} for the
  methods:
  %
  \footnotesize
  \begin{multicols}{2}
  \verbatimsnippet{fmult1prog}
  \end{multicols}
\end{block}

\begin{block}{Method definition}
  \label{sl:fclass-method}
  \verbatimsnippet{fmult1method}
\end{block}

\begin{block}{Methods have object as argument}
  \label{sl:fclass-self}
  You define functions that accept the type as first argument, but
  instead of declaring the argument as \indextermfort{type}, you define it as
  \indextermfort{class}.

  The members of the class object have to be accessed through the
  \verb+%+~operator.
  %
  \verbatimsnippet{pointsetf}
\end{block}

\begin{block}{Class organization}
  \label{sl:fclass-org}
  \begin{itemize}
  \item You're pretty much forced to use \indextermfort{Module}
  \item A class is a \indextermfort{Type} with a \indextermfort{contains} clause\\
    followed by \indextermfort{procedure} declaration
  \item Actual methods go in the \indextermfort{contains} part of the module
  \item $\Rightarrow$ First argument of method is the object itself. $\Leftarrow$
  \end{itemize}
\end{block}

\begin{block}{Point program}
  \label{sl:fpoint-program}
  \footnotesize
  
  \begin{multicols}{2}
    \verbatimsnippet{pointexmod}
    \columnbreak
    \verbatimsnippet{pointexmain}
  \end{multicols}  
\end{block}

\begin{exercise}
  \label{ex:fclass-point-distance}
  Take the point example program and add a distance function:
\begin{lstlisting}
  Type(Point) :: p1,p2
  ! ... initialize p1,p2
  dist = p1%distance(p2)
  ! ... print distance
\end{lstlisting}
\end{exercise}

\begin{exercise}
  \label{ex:fclass-translate}
  Write a method \n{add} for the \n{Point} type:
\begin{lstlisting}
  Type(Point) :: p1,p2,sum
  ! ... initialize p1,p2
  sum = p1%add(p2)
\end{lstlisting}
  What is the return type of the function \n{add}?
\end{exercise}

\Level 1 {Final proceduares: destructors}

The Fortran equivalent of
\emph{destructors}\index{destructor!in Fortran|see{final procedure}}
is a \indextermsub{final}{procedure},
designated by the \indextermdef{final} keyword.

\verbatimsnippet{finaldecl}

A final procedure has a single argument of the type that it applies to:

\verbatimsnippet{finalproc}

The final procedure is invoked when a derived type object is deleted,
except at the conclusion of a program:

\verbatimsnippet{finaluse}

\Level 1 {Inheritance}

\begin{block}{More OOP}
  \label{sl:oopf}
Inheritance:
\begin{lstlisting}
type,extends(baseclas) :: derived_class
\end{lstlisting}
Pure virtual:
\begin{lstlisting}
type,abstract
\end{lstlisting}
\end{block}

\begin{block}{Further reading}
  \url{http://fortranwiki.org/fortran/show/Object-oriented+programming}
\end{block}

\endinput

\begin{block}{Use modules!}
  \label{sl:fclass-module}
   It is of course best to put the type definition and method
   definitions in a module, so that you can \indextermfort{use} it.

   Mark methods as \indextermfort{private} so that they can only be used as part
   of the \indextermfort{type}:

   \verbatimsnippet{classmodule}
\end{block}

