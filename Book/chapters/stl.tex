% -*- latex -*-
%%%%%%%%%%%%%%%%%%%%%%%%%%%%%%%%%%%%%%%%%%%%%%%%%%%%%%%%%%%%%%%%
%%%%
%%%% This TeX file is part of the course
%%%% Introduction to Scientific Programming in C++/Fortran2003
%%%% copyright 2017-9 Victor Eijkhout eijkhout@tacc.utexas.edu
%%%%
%%%% stl.tex : about the standard template library
%%%%
%%%%%%%%%%%%%%%%%%%%%%%%%%%%%%%%%%%%%%%%%%%%%%%%%%%%%%%%%%%%%%%%

The C++ language has a \indexterm{Standard Template Library} (STL),
which contains functionality that is considered standard, but that is
actualy implemented in terms of already existing language
mechanisms. The STL is enormous, so we just highlight a couple of
parts.

You have already seen
\begin{itemize}
\item
  arrays (chapter~\ref{ch:array}),
\item strings (chapter~\ref{ch:string}),
\item streams (chapter~\ref{ch:io}).
\end{itemize}

Using a template class typically involves
\begin{lstlisting}
#include <something>
using std::function;
\end{lstlisting}
see section~\ref{sec:usename}.

\Level 0 {Complex numbers}
\label{sec:stl-complex}

\emph{Complex numbers}\index{complex numbers|textbf} use templating to
set their precision.
\begin{lstlisting}
#include <complex>
complex<float> f;
f.re = 1.; f.im = 2.;

complex<double> d(1.,3.);
\end{lstlisting}
Math operator like \n{+,*} are defined, as are math functions.

\Level 0 {Containers}

Vectors (section~\ref{sec:stdvector}) and strings
(chapter~\ref{ch:string}) are special cases of a STL
\indextermdef{container}. Methods such as \n{push_back} and \n{insert}
apply to all containers.

\Level 1 {Maps: associative arrays}
\label{sec:map}

Arrays use an integer-valued index. Sometimes you may wish to use an
index that is not ordered, or for which the ordering is not relevant.
A~common example is looking up information by string, such as finding
the age of a person, given their name. This is sometimes called
`indexing by content', and the data structure that supports this is
formally known as an \indextermsub{associative}{array}.

In C++ this is implemented through a \indextermttdef{map}:
\begin{lstlisting}
#include <map>
using std::map;
map<string,int> ages;
\end{lstlisting}
is set of
pairs where the first item (which is used for indexing) is of type
\n{string}, and the second item (which is found) is of type \n{int}.

A map is made by inserting the elements one-by-one:
\begin{lstlisting}
#include <map>
using std::make_pair;
ages.insert(make_pair("Alice",29));
ages["Bob"] = 32;
\end{lstlisting}

You can range over a map:
\begin{lstlisting}
for ( auto person : ages )
  cout << person.first << " has age " << person.second << endl;
\end{lstlisting}
A more elegant solution uses structured bindings (section~\ref{sec:tuple}):
\begin{lstlisting}
for ( auto [person,age] : ages )
  cout << person << " has age " << age << endl;
\end{lstlisting}

Searching for a key gives either the iterator of the key/value pair,
or the \lstinline{end} iterator if not found:
%
\verbatimsnippet{intcountfind}

\Level 1 {Iterators}
\label{sec:iterator-class}

The container class has a subclass \indextermdef{iterator} that can be
used to iterate through all elements of a container. This was
discussed in section~\ref{sec:iterator}.

However, an \indextermtt{iterator} can be used outside of strictly iterating.
\begin{block}{Iterators outside a loop}
  \label{sl:vec-iterator}
  First, you can use them by themselves:
  %
  \snippetwithoutput{veciterator}{stl}{iter}
  %
  (Note: the \n{auto} actually stands for \n{vector::iterator})
\end{block}

Note that the star notation is an operator, no a
\indextermbus{pointer}{dereference}:
%
\verbatimsnippet{iterderef}

\Level 2 {Vector methods using iterators}

Vector methods such as \indextermtt{erase} and \indextermtt{insert} use
these iterators:
\begin{block}{Iterators in vector methods}
  \label{sl:vec-erase}
  Methods \n{erase} and \n{insert} indicate their range with begin/end
  iterators
  %
  \snippetwithoutput[iter]{vecerase}{stl}{erase}
  %
  Note: end is exclusive.
\end{block}

\Level 2 {Algorithms using iterators}
\label{sec:alg-iter}

Numerical \emph{reductions}\index{reduction} can be applied using iterators:
\begin{block}{Reduction operation}
  \label{sl:vec-accumulate}
  Default is sum reduction:
  %
  \snippetwithoutput[reduce]{vecaccumulate}{stl}{accumulate}
  %
\end{block}

\begin{block}{Reduction with supplied operator}
  \label{sl:vec-multiplies}
  Supply multiply operator:
  %
  \snippetwithoutput[reduce]{vecproduct}{stl}{product}  
\end{block}

\begin{block}{Max reduction}
  Specific for the max reduction is \indextermttdef{max_element}.
  This can be called without a comparator (for numerical max),
  or with a comparator for general maximum operations.

  Example: maximum relative deviation from a quantity:
\begin{lstlisting}
max_element(myvalue.begin(),myvalue.end(),
    [my_sum_of_squares] (double x,double y) -> bool {
        return fabs( (my_sum_of_squares-x)/x ) < fabs( (my_sum_of_squares-y)/y ); }
);
\end{lstlisting}
\end{block}

\Level 2 {Forming sub-arrays}

Iterators can be used to construct a \indextermtt{vector}. This can
for instance be used to create a
\emph{subvector}\index{vector!subvector}:
%
\snippetwithoutput[iter]{subvectorcopy}{iter}{subvector}

\Level 0 {Regular expression}

The header \indextermtt{regex} gives C++ the functionality for
\indexterm{regular expression} matching. For instance,
\indextermttdef{regex_match} returns whether or not an expression matches:

\snippetwithoutput{regexname}{regexp}{regexp}

You can do further analysis with \indextermttdef{regex_search}:
\begin{itemize}
\item the function itself evaluates to a \lstinline{bool};
\item there is a return parameter of type \indexterm{smatch} which can
  be queried:
\item the method \lstinline{smatch::position} states where the expression
  was matched, while \lstinline{smatch::str} returns the string that
  was matched.
\end{itemize}

\snippetwithoutput{regsearch}{regexp}{search}

\Level 0 {Tuples and structured bindings}
\label{sec:tuple}

Remember how in section~\ref{sec:pass-by-ref} we said that if you
wanted to return more than one value, you could not do that through a
return value, and had to use an \indextermsub{output}{parameter}?
Well, using the \ac{STL} there is a different solution.

You can make a \indextermdef{tuple}: an entity that comprises several
components, possibly of different type, and which unlike a
\indextermtt{struct} you do not need to define beforehand.

\lstset{style=reviewcode,language=C++}
\begin{block}{C++11 style tuples}
  \label{sl:tuple11}
  \begin{lstlisting}
    std::tuple<int,double> id = \
        std::make_tuple<int,double>(3,5.12);
    double result = std::get<1>(id);
    std::get<0>(id) += 1;
  \end{lstlisting}
\end{block}

This does not look terribly elegant. Fortunately,
\emph{C++17}\index{C++!C++17} can use denotations and the \n{auto}
keyword to make this considerably shorter. Consider the case of a
function that returns a tuple. You could use \n{auto} to deduce the
return type:
%
\verbatimsnippet{tuplemake}
%
but more interestingly, you can use a
\indextermbus{tuple}{denotation}:
%
\verbatimsnippet{tupledenote}

\begin{slide}{Function returning tuple}
  \label{sl:tuplefun}
  \begin{multicols}{2}
    Return type deduction:
    \verbatimsnippet{tuplemake}\columnbreak
    Alternative:
    \verbatimsnippet{tupledenote}
  \end{multicols}
\end{slide}

\begin{block}{Catching a returned tuple}
  \label{sl:catch-tuple}
  The calling code is particularly elegant:
  %
  \snippetwithoutput{tupleauto}{stl}{tuple}

  This is known as \indexterm{structured binding}.
\end{block}

An interesting use of structured bindings is iterating over a map (section~\ref{sec:map}):
\begin{lstlisting}
for ( const auto &[key,value] : mymap ) ....
\end{lstlisting}

\Level 0 {Union-like stuff: tuples, optionals, variants}

There are cases where you need a value that is one type or another,
for instance, a number if a computation succeeded, and an error
indicator if not.

The simplest solution is to have a function that returns both a bool
and a number:
%
\verbatimsnippet{rootorerror}

We will now consider some more idiomatically C++17 solutions to this.

\Level 1 {Tuples}

Using tuples (section~\ref{sec:tuple}) 
the solution to the above `a~number or an error' now becomes:
%
\verbatimsnippet{rootandvalid}

\Level 1 {Optional}
\label{sec:std-optional}

The most elegant solution to `a~number or an error' is to have a
single quantity that you can query whether it's valid.
For this, the
\emph{C++17}\index{C++!C++17} standard
introduced the concept of a \indextermsub{nullable}{type}:
a~type that can somehow convey that it's empty.

Here we discuss \lstinline{std::}\indextermtt{optional}.

\begin{lstlisting}
#include <optional>
using std::optional;
\end{lstlisting}

\begin{itemize}
\item You can create an optional quantity with a function that returns
  either a value of the indicated type, or \verb+{}+, which is a
  synonym for \indextermtt{nullopt}.
\begin{lstlisting}
optional<float> f {
  if (something) return 3.14;
  else return {};
}
\end{lstlisting}
\item You can test whether the optional quantity has a quantity with
  \indextermtt{has_value}, in which case you can extract the quantity
  with \indextermtt{value}:
\begin{lstlisting}
auto maybe_x = f();
if (f.has_value)
  // do something with f.value();
\end{lstlisting}
\end{itemize}

There is a function \indextermttdef{value_or} that gives the value, or
a default if the optional did not have a value.

\begin{slide}{Optional results (C++17)}
  \label{sl:optional-root}
  The most elegant solution to `a~number or an error' is to have a
  single quantity that you can query whether it's valid.
  %
  \verbatimsnippet{mayberootptr}
\end{slide}

\Level 1 {Variant}
\label{sec:stl-variant}

In~C, a~\indextermtt{union} is an entity that can be one of a number
of types. Just like that C~arrays do not know their size, a
\lstinline{union} does not know what type it is. The C++
\indextermttdef{variant} does not suffer from these limitations. The
function \indextermttdef{get_if} can retrieve a value by type.

\begin{lstlisting}
#include <variant>
using std::variant;
variant<int,double,string> union_ids {"Hello world"};
if ( auto union_int =  get_if<int>(&union_ids) ; union_int )
  cout << "Int: " << union_int.value() << endl;
\end{lstlisting}
(You can also use the dereference `star' or \lstinline{union_int}.)

If you know what type the variant is, you
can get it directly:
\begin{lstlisting}
  cout << "Int: " << get<int>( union_ids ) << endl;
\end{lstlisting}

\begin{exercise}
  Write a routine that computes the roots of the quadratic equation
  \[ ax^2+bx+c=0. \]
  The routine should return two roots, or one root, or an indication
  that the equation has no solutions.
\end{exercise}

In this exercise you can return a boolean to indicate `no roots', but
a boolean can have two values, and only one has meaning. For such
cases there is \lstinline{std::}\indextermttdef{monostate}.

\Level 1 {Any}
\label{sec:stl-any}

While \lstinline{variant} can be any of a number of prespecified
types, \lstinline{std::}\indextermttdef{any} can contain really any
type. Thus it is the equivalent of \lstinline{void*} in~C.

An \lstinline{any} object can be cast with \indextermttdef{any_cast}:
\begin{lstlisting}
std::any a{12};
std::any_cast<int>(a); // succeeds
std::any_cast<string>(a); // fails
\end{lstlisting}

\Level 0 {Algorithms}
\label{sec:algorithm}

Many simple algorithms on arrays, testing `there is' or `for all', no
longer have to be coded out in~C++. They can now be done with a single
function from \lstinline{std::}\indextermttdef{algorithm}.
\begin{itemize}
\item Test if any element satisfies a condition:
  \indextermttdef{any_of}
\item Test if all elements satisfy a condition:
  \indextermttdef{all_of}
\item Apply an operation to all elements: \indextermttdef{for_each}.
\end{itemize}

The object to which the function applies is not specified directly; 
rather, you have to specify a start
and end iterator. An examples applied to a 
\lstinline{std::vector}:
%
\snippetwithoutput{printeach}{stl}{printeach}

See section~\ref{sec:iterator-class} for iterators,
in particular section~\ref{sec:alg-iter} for algorithms with iterators,
and
section~\ref{sec:lambda} for the use of lambda expressions.

\Level 0 {Limits}
\label{sec:limits}

There used to be a header file \indextermtt{limits.h} that contained
macros such as \indextermtt{MAX_INT}. While this is still available,
the \ac{STL} offers a better solution in the
\indextermtt{numeric_limits} header.

\begin{block}{Templated functions for limits}
  \label{sl:stl-limits}
  Use header file \indextermtt{limits}:
\begin{lstlisting}
#include <limits>
using std::numeric_limits;

cout << numeric_limits<long>::max();
\end{lstlisting}
\end{block}

\begin{block}{Some limit values}
  \label{sl:ieee-limits}
  \snippetwithoutput{stllimits}{stl}{limits}
\end{block}

\begin{block}{Limits of floating point values}
  \label{sl:float-limits}
  \begin{itemize}
  \item \lstinline{min} is the smallest positive number;
  \item use \indextermdef{lowest} for `most negative'.
  \item There is an \indextermdef{epsilon} function for machine precision:
  \end{itemize}
  \snippetwithoutput{stllimitfloat}{stl}{eps}
\end{block}

\begin{exercise}
  \label{ex:big-factorial}
  Write a program to discover what the maximal~$n$ is so that~$n!$,
  that is, $n$-factorial, can be represented in an \n{int}, \n{long},
  or \n{long long}. Can you write this as a templated function?
\end{exercise}

Operations such as dividing by zero lead to floating point numbers
that do not have a valid value. For efficiency of computation, the
processor will compute with these as if they are any other floating
point number.

There are tests for detecting whether a number is \indextermtt{Inf} or
\indextermtt{NaN}. However, using these may slow a computation down.

\begin{block}{Detection of Inf and NaN}
  The functions \indextermtt{isinf} and \indextermtt{isnan} are
  defined for the floating point types (\n{float}, \n{double}, \n{long
    double}), returning a \n{bool}.
\begin{lstlisting}
#include <math.h>
isnan(-1.0/0.0);   // false
isnan(sqrt(-1.0)); // true
isinf(-1.0/0.0);   // true
isinf(sqrt(-1.0)); // false
\end{lstlisting}
\end{block}

\Level 1 {Storage}

\begin{itemize}
\item An \n{int} used to be 16 bits in the olden days, but these days
  32~bits is more common.
\item A \n{long int}\indextermtt{long} has at least the range of an
  \n{int}, and at least 32~bits.
\item A \n{long long int}\indextermtt{long long} has at least the
  range of a \n{long}, and at least 64~bits.
\end{itemize}

\Level 0 {Random numbers}
\label{sec:stl:random}

The \ac{STL} has a
\indextermbus{random number}{generator}
that is more general and more flexible than the C~version (section~\ref{sec:crand}).
\begin{itemize}
\item There are several generators that give uniformly distributed
  numbers;
\item then there are distributions that translate this to non-uniform
  or discrete distributions.
\end{itemize}

\begin{block}{Random number example}
  \label{sl:stl:rand16}
\begin{lstlisting}
// set the default generator
std::default_random_engine generator;

// distribution: ints 1..6
std::uniform_int_distribution<int> distribution(1,6);

// apply distribution to generator:
int dice_roll = distribution(generator);
  // generates number in the range 1..6 
\end{lstlisting}
\end{block}

\Level 0 {Time}

