% -*- latex -*-
%%%%%%%%%%%%%%%%%%%%%%%%%%%%%%%%%%%%%%%%%%%%%%%%%%%%%%%%%%%%%%%%
%%%%
%%%% This TeX file is part of the course
%%%% Introduction to Scientific Programming in C++/Fortran2003
%%%% copyright 2017 Victor Eijkhout eijkhout@tacc.utexas.edu
%%%%
%%%% stl.tex : about the standard template library
%%%%
%%%%%%%%%%%%%%%%%%%%%%%%%%%%%%%%%%%%%%%%%%%%%%%%%%%%%%%%%%%%%%%%

The C++ language has a \indexterm{Standard Template Library} (STL),
which contains functionality that is considered standard, but that is
actualy implemented in terms of already existing language
mechanisms. The STL is enormous, so we just highlight a couple of
parts.

You have already seen
\begin{itemize}
\item
  arrays (chapter~\ref{ch:array}),
\item strings (chapter~\ref{ch:string}),
\item streams (chapter~\ref{ch:io}).
\end{itemize}

Using a template class typically involves
\begin{verbatim}
#include <something>
using std::function;
\end{verbatim}
see section~\ref{sec:usename}.

\Level 0 {Containers}

Vectors (section~\ref{sec:stdvector}) and strings
(chapter~\ref{ch:string}) are special cases of a STL
\indextermdef{container}. Methods such as \n{push_back} and \n{insert}
apply to all containers.

\Level 1 {Iterators}

The container class has a subclass \indextermdef{iterator} that can be
used to iterate through all elements of a container. This was
discussed in section~\ref{sec:iterator}.

\Level 0 {Complex numbers}
\label{sec:stl-complex}

\emph{Complex numbers}\index{complex numbers|textbf} use templating to
set their precision.
\begin{verbatim}
#include <complex>
complex<float> f;
f.re = 1.; f.im = 2.;

complex<double> d(1.,3.);
\end{verbatim}
Math operator like \n{+,*} are defined, as are math functions.

\Level 0 {Random numbers}

\Level 0 {Time}
