% -*- latex -*-
%%%%%%%%%%%%%%%%%%%%%%%%%%%%%%%%%%%%%%%%%%%%%%%%%%%%%%%%%%%%%%%%
%%%%
%%%% This TeX file is part of the course
%%%% Introduction to Scientific Programming in C++/Fortran2003
%%%% copyright 2017-9 Victor Eijkhout eijkhout@tacc.utexas.edu
%%%%
%%%% const.tex : discussion of const
%%%%
%%%%%%%%%%%%%%%%%%%%%%%%%%%%%%%%%%%%%%%%%%%%%%%%%%%%%%%%%%%%%%%%

The keyword \n{const} can be used to indicate that various quantities
can not be changed. This is mostly for programming safety: if you
declare that a method will not change any members, and it does so
(indirectly) anyway, the compiler will warn you about this.

\begin{slide}{Why const?}
  \label{sl:why-const}
  \begin{itemize}
  \item Clean coding: express your intentions whether quantities are
    supposed to not alter.
  \item Functional style programming: prevent side effects.
  \item NOT for optimization: the compiler does not use this for `constant
    hoisting' (moving constant expression out of a loop).
  \end{itemize}
\end{slide}

\Level 0 {Const arguments}
\label{sec:constparam}

The use of const arguments is one way of protecting you against yourself.
If an argument is conceptually supposed to stay constant, the compiler
will catch it if you mistakenly try to change it.

\begin{block}{Constant arguments}
  \label{sl:const-arg}
  Function arguments marked \indextermtt{const} can not be altered by
  the function code. The following segment gives a compilation error:
  %
  \verbatimsnippet{constchange}
\end{block}

\Level 0 {Const references}
\label{sec:const-ref}

A more sophisticated use of const is the
\indextermbus{const}{reference}:
\begin{lstlisting}
void f( const int &i ) { .... }
\end{lstlisting}
This may look strange. After all, references, and the
pass-by-reference mechanism, were introduced in
section~\ref{sec:passing} to return changed values to the calling
environment. The const keyword negates that possibility of changing
the parameter.

\begin{slide}{Const ref parameters}
  \label{sl:const-ref}
\begin{lstlisting}
void f( const int &i ) { .... }
\end{lstlisting}
\begin{itemize}
\item Pass by reference: no copying, so cheap
\item Const: no accidental altering.
\end{itemize}

\end{slide}

But there is a second reason for using references. Parameters are
passed by value, which means that they are copied, and that includes
big objects such as \n{std::vector}. Using a reference to pass a
vector is much less costly in both time and space, but then there is the
possibility of changes to the vector propagating back to the calling
environment.

Consider a class that has methods that return an internal member by
reference, once as const reference and once not:
%
\snippetwithoutput{constref}{const}{constref}

We can make visible the difference between pass by value and pass by
const-reference if we define a class where the
\indextermsub{copy}{constructor} explicitly reports itself:
%
\verbatimsnippet{classwithcopy}
  
Now if we define two functions, with the two parameter passing
mechanisms, we see that passing by value invokes the copy constructor,
and passing by const reference does not:
%
\snippetwithoutput{constcopy}{const}{constcopy}

\Level 1 {Const references in range-based loops}

The same pass by value/reference issue comes up in range-based for
loops. The syntax
\begin{lstlisting}
for ( auto v : some_vector )
\end{lstlisting}
copies the vector elements to the \lstinline{v} variable, whereas
\begin{lstlisting}
for ( auto& v : some_vector )
\end{lstlisting}
makes a reference. To get the benefits of references (no copy cost)
while avoiding the pitfalls (inadvertant changes), you can also use a
const-reference here:
\begin{lstlisting}
for ( const auto& v : some_vector )
\end{lstlisting}

\Level 0 {Const methods}

We can distinguish two types of methods: those that alter internal
data members of the object, and those that don't. The ones that don't
can be marked \indextermtt{const}.
While this is in no way required, it contributes to a clean
programming style:

\begin{block}{Intention checking}
  Using \lstinline{const} will catch mismatches between the prototype
  and definition of the method. For instance,
  \begin{lstlisting}
    class Things {
      private:
      int var;
      public:
      f(int &ivar,int c) const {
        var += c; // typo: should be `ivar'
      }
    }
  \end{lstlisting}
\end{block}

Here, the use of \n{var} was a typo, should have been \n{ivar}. Since
the method is marked \n{const}, the compiler will generate an error.

\begin{block}{No side-effects}
  \label{sl:const-sideeffect}
  It encourages a functional style, in the sense that it makes
  \indexterm{side-effects} impossible:
\begin{lstlisting}
class Things {
private:
  int i;
public:
  int get() const { return i; }
  int inc() { return i++; } // side-effect!
  void addto(int &thru) const { thru += i; }
}
\end{lstlisting}
\end{block}

\begin{comment}
\item It allows the compiler to optimize your code. For instance:
  \begin{lstlisting}
    class Things {
      public:
      int f() const { /* ... */ };
      int g() const { /* ... */ };
    }
    ...
    Things t;
    int x,y,z;
    x = t.f();
    y = t.g();
    z = t.f();
  \end{lstlisting}
  Since the methods did not alter the object, the compiler can conclude
  that \n{x,z} are the same, and skip the calculation for~\n{z}.
\end{comment}

\Level 0 {Overloading on const}

\begin{block}{Const polymorphism}
  \label{sl:const-poly}
  A const method and its non-const variant are different enough that you
  can use this for overloading.
  %
  \snippetwithoutput{constatclass}{const}{constat}
\end{block}

\begin{exercise}
  \label{ex:const-poly}
  Explore variations on this example, and see which ones work and
  which ones not.
  \begin{enumerate}
  \item Remove the second definition of \lstinline{at}. Can you
    explain the error?
  \item
    Remove either of the \lstinline{const} keywords from the second
    \lstinline{at} method. What errors do you get?
  \end{enumerate}
\end{exercise}

\Level 0 {Const and pointers}

Let's declare a class \n{thing} to point to, and a class \n{has_thing}
that contains a pointer to a \n{thing}.

\begin{multicols}{2}
\verbatimsnippet{constpoint1}
\end{multicols}

If we define a method to return the pointer, we get a copy of the
pointer, so redirecting that pointer has no effect on the container:
%
\snippetwithoutput[constpoint]{constpoint2}{const}{constpoint2}

If we return the pointer by reference we can
change it. However, this requires the method not to be \n{const}.
On the other hand, with the \n{const} method earlier we can change the
object:
%
\snippetwithoutput[constpoint]{constpoint3}{const}{constpoint3}

If you want to prevent the pointed object from being changed, you can
declare the pointer as a \n{shared_ptr<const thing>}:
%
\begin{multicols}{2}
\verbatimsnippet{constpoint4}
\end{multicols}

\Level 0 {Mutable}

Typical class with non-const update methods,
and const readout of some computed quantity:
\begin{lstlisting}
class Stuff {
private:
  int i,j;
public:
  Stuff(int i,int j) : i(i),j(j) {};
  void seti(int inew) { i = inew; };
  void setj(int jnew) { j = jnew; };
  int result () const { return i+j; };
\end{lstlisting}
Attempt at caching the computation:
\begin{lstlisting}
class Stuff {
private:
  int i,j;
  int cache;
  void compute_cache() { cache = i+j; };
public:
  Stuff(int i,int j) : i(i),j(j) {};
  void seti(int inew) { i = inew; compute_cache(); };
  void setj(int jnew) { j = jnew; compute_cache(); };
  int result () const { return cache; };
\end{lstlisting}
But what if setting happens way more often than getting?
\begin{lstlisting}
class Stuff {
private:
  int i,j;
  int cache;
  bool cache_valid{false};
  void update_cache() {
    if (!cache_valid) {
      cache = i+j; cache_valid = true;
  };
public:
  Stuff(int i,int j) : i(i),j(j) {};
  void seti(int inew) { i = inew; cache_valid = false; };
  void setj(int jnew) { j = jnew; cache_valid = false; }
  int result () const {
    update_cache(); return cache; };
\end{lstlisting}
This does not compile, becuase \lstinline{result} is const, but it
calls a non-const function.

We can solve this by
declaring the cache variables to be \indextermttdef{mutable}.
Then the methods that conceptually don't change the object
can still stay \lstinline{const}, even while altering the state of the
object.
%
(It is better not to use \indextermtt{const_cast}.)

\begin{lstlisting}
class Stuff {
private:
  int i,j;
  mutable int cache;
  mutable bool cache_valid{false};
  void update_cache() const {
    if (!cache_valid) {
      cache = i+j; cache_valid = true;
  };
public:
  Stuff(int i,int j) : i(i),j(j) {};
  void seti(int inew) { i = inew; cache_valid = false; };
  void setj(int jnew) { j = jnew; cache_valid = false; }
  int result () const {
    update_cache(); return cache; };
\end{lstlisting}

\Level 0 {Compile-time constants}

Compilers have long been able to simplify expressions that only
contain constants:
\begin{lstlisting}
int i=5;
int j=i+4;
f(j)
\end{lstlisting}
Here the compiler will conclude that \lstinline{j} is~9, and that's
where the story stops. it also
becomes possible to let \lstinline{f(j)} be evaluated by the compiler,
if the function \lstinline{f} is simple enough.
\emph{C++17}\index{C++!C++17} added several more variants of
\indextermtt{constexpr} usage.

The \indextermtt{const} keyword can be used to indicate that a
variable can not be changed:
\begin{lstlisting}
const int i=5;
// DOES NOT COMPILE:
i += 2;
\end{lstlisting}

\begin{block}{Constexpr if}
  \label{sl:constexpr-if}
  The combination \lstinline{if constexpr} is useful with templates:
\begin{lstlisting}
template <typename T>
auto get_value(T t) {
  if constexpr (std::is_pointer_v<T>)
    return *t;
  else
    return t;
}
\end{lstlisting}
\end{block}

\begin{block}{Constant functions}
  \label{sl:constexpr}
  To declare a function to be constant, use
  \indextermttdef{constexpr}. The standard example is:
\begin{lstlisting}
constexpr double pi() {
  return 4.0 * atan(1.0); };
\end{lstlisting}
but also
\begin{lstlisting}
constexpr int factor(int n) {
  return n <= 1 ? 1 : (n*fact(n-1));
}
\end{lstlisting}
(Recursion in \emph{C++11}\index{C++!C++11},
loops and local variables in \emph{C++14}\index{C++!C++14}.)
\end{block}

\begin{itemize}
\item Can use conditionals, if testing on constant data;
\item can use loops, if number of iterations constant;
\item (\emph{C++20}\index{C++!C++20}) can allocate memory, if size constant.
\end{itemize}
