% -*- latex -*-
%%%%%%%%%%%%%%%%%%%%%%%%%%%%%%%%%%%%%%%%%%%%%%%%%%%%%%%%%%%%%%%%
%%%%
%%%% This TeX file is part of the course
%%%% Introduction to Scientific Programming in C++/Fortran2003
%%%% copyright 2017/8 Victor Eijkhout eijkhout@tacc.utexas.edu
%%%%
%%%% functionf.tex : functions and such in Fortran
%%%%
%%%%%%%%%%%%%%%%%%%%%%%%%%%%%%%%%%%%%%%%%%%%%%%%%%%%%%%%%%%%%%%%

\Level 0 {Procedures}

Programs can have subprograms: parts of code that for some reason you
want to separate from the main program. If you structure your code in a single
file, this is the recommended structure:

\begin{block}{Subprograms in contains clause}
  \label{sl:contains}
\begin{verbatim}
Program foo
  < declarations>
  < executable statements >
  Contains
    < subprogram definitions >
End Program foo
\end{verbatim}
\end{block}

That is, subprograms are placed after the main program statements,
separated by a \indextermfort{contains} clause.

In general, these are the  placements of subprograms:
\begin{itemize}
\item Internal: after the \n{Contains} clause of a program
\item In a \indextermfort{Module}; see section~\ref{sec:modulef}.
\item Externally: the subprogram is not internal to a \n{Program} or
  \n{Module}. In this case it's safest to declare it through an
  \indextermfort{Interface} specification; section~\ref{sec:finterface}.
\end{itemize}
\Level 1 {Subroutines and functions}

Fortran has two types of subprograms:
\begin{itemize}
\item Subroutines, which are somewhat like \n{void} functions in C++:
  they can be used to structure the code, and they can only return
  information to the calling environment through their parameters.
\item Functions, which are like C++ functions with a return type.
\end{itemize}

Both types have the same structure, which is roughly the same as of
the main program:
\begin{verbatim}
subroutine foo( <parameters> )
<variable declarations>
<executable statements>
end subroutine foo
\end{verbatim}

Exit from a procedure can happen two ways:
\begin{enumerate}
\item the flow of control reaches the end of the procedure body, or
\item execution is finished by an explicit \indextermtt{return}
  statement.
\begin{verbatim}
subroutine foo()
  print *,"foo"
  if (something) return
  print *,"bar"
end subroutine foo
\end{verbatim}
\end{enumerate}
The \n{return} statement is optional in the first case.

\begin{exercise}
  Rewrite the above function \n{foo} without a \n{return} statement.
\end{exercise}

A subroutine is invoked with a \indextermtt{call} statement:
\begin{verbatim}
call foo()
\end{verbatim}

\begin{slide}{Subroutines}
  \label{sl:subroutine}
\begin{verbatim}
subroutine foo()
  implicit none
  print *,"foo"
  if (something) return
  print *,"bar"
end subroutine foo
\end{verbatim}
\begin{itemize}
\item Looks much like a main program
\item Ends at the end, or when \n{return} is reached
\item Activated with 
\begin{verbatim}
  call foo()
\end{verbatim}
\end{itemize}
\end{slide}

\begin{block}{Fortran nomenclature}
  \label{sl:fortran-dummy}
  The term \indextermsub{dummy}{argument} is what Fortran calls there
  parameters in the subprogram definition. The arguments in the
  subprogram call are the \indextermsub{actual}{argument}s.
\end{block}

\begin{block}{Recursion}
  \label{sl:funcf:recursion}
  Declare function as \indextermfort{Recursive}~\n{Function}
\begin{verbatim}
Recursive Function factorial(n) Result(fact)
  if (n==0) then
    fact = 1
  else
    fact = n * factorial(n-1)
  end if
End Function Factorial
\end{verbatim}
\end{block}

\Level 1 {Return results}

While a \n{subroutine} can only return information through its parameters,
a~\indexterm{function} procedure returns an explicit result:
\begin{verbatim}
logical function test(x)
  implicit none
  real :: x

  test = some_test_on(x)
  return ! optional, see above
end function test
\end{verbatim}
You see that the result is not returned in the \n{return} statement,
but rather through assignment to the function name. The \n{return}
statement, as before, is optional and only indicates where the flow of
control ends.

A \indextermdef{function} in Fortran is a subprogram that return a
result to its calling program, much like a non-void function in~C++

\begin{block}{Function definition and usage}
  \label{sl:ffunction-def}
  \begin{itemize}
  \item Return type, keyword \n{function}, name, parameters
  \item Function body has statements
  \item Result is returned by assigning to the function name
  \item Use: \n{y = f(x)}
  \end{itemize}
\end{block}

\begin{block}{Function example}
  \label{sl:ffunction-ex}
  \snippetwithoutput{fplusone}{funcf}{plusone}
\end{block}

A function is not invoked with \n{call}, but rather through being used
in an expression:
\begin{verbatim}
if (test(3.0) .and. something_else) ...
\end{verbatim}
You now have the following cases to make the function known in the
main program:
\begin{itemize}
\item If the function is in a \n{contains} section, its type is known
  in the main program.
\item If the function is in a module (see section~\ref{sec:modulef}
  below), it becomes known through a \n{use} statement.
\end{itemize}

\begin{f77note}
  Without modules and \n{contains} sections, you need to declare the
  function type explitly in the calling program. The safe way is
  through using an \indextermtt{interface} specification.
\end{f77note}

\begin{exercise}
  \label{ex:freadpos}
  Write a program that asks the user for a positive number; negative
  input should be rejected.  Fill in the missing lines in this code
  fragment:
  \snippetwithoutput{readpos}{funcf}{readpos}
\end{exercise}

\begin{block}{Why a `contains' clause?}
  \label{sl:whycontain}
  \begin{multicols}{2}
    \verbatimsnippet{nocontain}
    Warning only, crashes.
    \vfill\columnbreak
    \verbatimsnippet{wrongcontain}
    Error, does not compile
  \end{multicols}
\end{block}

\begin{block}{Why a `contains' clause, take 2}
  \label{sl:whycontain_type}
  \snippetwithoutput{nocontaintype}{funcf}{nocontaintype}
  At best compiler warning if all in the same file\\
  For future reference: if you see very small floating point numbers,
  maybe you have made this error.
\end{block}

\Level 1 {Arguments}

\begin{block}{Subprogram arguments}
  \label{sl:farguments}
 Arguments are declared in subprogram body:
\begin{verbatim}
subroutine f(x,y,i)
  implicit none
  integer,intent(in) :: i
  real(4),intent(out) :: x
  real(8),intent(inout) :: y
  x = 5; y = y+6
end subroutine f
! and in the main program
call f(x,y,5)
\end{verbatim}
\end{block}

\begin{block}{Parameter passing}
  \label{sl:fpassing}
  \begin{itemize}
  \item Everything is passed by reference.
  \item Use \n{in}, \n{out}, \n{inout} qualifiers to clarify semantics
    to compiler.
  \item Terminology: Fortran talks about `dummy' and `actual'
    arguments. Dummy: in the definition; actual: in the calling program.
  \end{itemize}
\end{block}

\begin{block}{Intent checking}
  \label{sl:fintent}
  Compiler checks your intent against your implementation. This code
  is not legal:

  \verbatimsnippet{arginwrong}
\end{block}

\begin{block}{Why intent checking?}
\label{sl:intentwhy}
Allow compiler optimizations:

\begin{multicols}{2}
\begin{verbatim}
x = f()
call ArgOut(x)
print *,x
\end{verbatim}
Call to \n{f} removed
\vfill\columnbreak
\begin{verbatim}
do i=1,1000
  x = ! something
  y1 = .... x ....
  call ArgIn(x)
  y2 = ! same expression as y1
\end{verbatim}
\n{y2} is same as \n{y1} because \n{x} not changed
\end{multicols}

\end{block}

\Level 1 {Types of procedures}

Procedures that are in the main program (or another type of program
unit), separated by a \n{contains} clause, are known as
\indextermsub{internal}{procedures}. This is as opposed to
\indextermsub{module}{procedures}.

There are also \indexterm{statement functions}, which are
single-statement functions, usually to identify commonly used
complicated expressions in a program unit. Presumably the compiler
will \indexterm{inline} them for efficiency.

The \n{entry} statement is so bizarre that I refuse to discuss it.

\Level 1 {More about arguments}

\begin{block}{Keyword arguments}
  \label{sl:funcf:keyword}
  \begin{itemize}
  \item Use the name of the \indextermsub{formal}{parameter} as
    keyword.
  \item Keyword arguments have to come last.
  \end{itemize}
  \snippetwithoutput{sayxykw}{funcf}{keyword}
\end{block}

\begin{block}{Optional arguments}
  \label{sl:funcf:optional}
  \begin{itemize}
  \item Extra specifier: \indextermfort{Optional}
  \item Presence of argument can be tested with \indextermfort{Present}
  \end{itemize}
\end{block}

\Level 0 {Interfaces}
\label{sec:finterface}

If you want to use a subprogram in your main program, the compiler
needs to know the signature of the subprogram: how many arguments, of
what type, and with what intent. You have seen how the
\indextermtt{contains} clause can be used for this purpose if the
subprogram resides in the same file as the main program.

If the subprogram is in a separate file, the compiler does not see
definition and usage in one go. To allowed the compiler to do checking
on proper usage, we can use an \indextermttdef{interface} block. This
is placed at the calling site, declaring the signature of the
subprogram.

\begin{multicols}{2}
\textbf{Main program:}
\verbatimsnippet{interfacemain}
\vfill\columnbreak
\textbf{Subprogram:}
\verbatimsnippet{interfunc}
\end{multicols}

The \indextermtt{interface} block is not required (an older
\indextermtt{external} mechanism exists for functions), but is
recommended.
It is required if the function takes function arguments.

\Level 1 {Polymorphism}

The \indextermtt{interface} block can be used to define a generic
function:
\begin{verbatim}
interface f
function f1( ..... )
function f2( ..... )
end interface f
\end{verbatim}
where \n{f1},\n{f2} are functions that can be distinguished by their
argument types. The generic function~\n{f} then becomes either \n{f1}
or \n{f2} depending on what type of argument it is called with.

\Level 0 {Modules}
\label{sec:modulef}

A module is a container for definitions of subprograms and types, and
for data such as constants and variables. A~module is not a 
structure or object: there is only one instance.

What do you use a module for?
\begin{itemize}
\item Type definitions: it is legal to have the same type definition
  in multiple program units, but this is not a good idea. 
  Write the definition just once in a module and make it available
  that way.
\item Function definitions: this makes the functions available in multiple
  sources files of the same program, or in multiple programs.
\item Define constants: for physics simulations, put all constants in
  one module and use that, rather than spelling out the constants each
  time.
\item Global variables: put variables in a module if they do not fit
  an obvious scope.
  \begin{f77note}
  Modules are much cleaner than common blocks. Do not use those.
  \end{f77note}
\end{itemize}
%
\begin{block}{Module definition}
  \label{sl:modulef}
\verbatimsnippet{modulef}
\end{block}


Any routines come after the \n{contains}\index{contains!in modules}

A module is made available with the \indextermtt{use} keyword, which
needs to go before the \n{implicit none}.
%
\begin{block}{Module use}
  \label{sl:moduleuse}
  \verbatimsnippet{moduleuse}
  Also possible:
\begin{verbatim}
Use mymodule, Only: func1,func2
Use mymodule, func1 => new_name1
\end{verbatim}
\end{block}

By default, all the contents of a module is usable by a subprogram
that uses it. However, a keyword \indextermtt{private} make module
contents available only inside the module.
You can make the default behaviour explicit by using the
\indextermtt{public} keyword. Both \n{public,private} can be used as
attributes on definitions in the module.
There is a keyword \indextermtt{protected} for data members that
are public, but can not be altered by code outside the module.

If you compile a module, you will find a \n{.mod} file in your
directory. (This is little like a~\n{.h} file in~C++.)
If this file is not present, you can not \n{use} the module in another
program unit, so you need to compile the file containing the module
first.

\begin{exercise}
  \label{ex:fpointmod}
  Write a module \n{PointMod} that defines a type \n{Point} and a
  function \n{distance} to make this code work:
  %
  \verbatimsnippet{pointmainex}
  %
  Put the program and module in two separate files and compile thusly:
\begin{verbatim}
ifort -g -c pointmod.F90
ifort -g -c pointmain.F90
ifort -g  -o pointmain pointmod.o pointmain.o 
\end{verbatim}
\end{exercise}

\Level 1 {Polymorphism}

\begin{verbatim}
module somemodule

INTERFACE swap
MODULE PROCEDURE swapreal, swapint, swaplog, swappoint
END INTERFACE

  contains
  subroutine swapreal
  ...
  end subroutine swapreal
  subroutine swapint
  ...
  end subroutine swapint
\end{verbatim}

\Level 1 {Operator overloading}

\begin{verbatim}
MODULE operator_overloading
IMPLICIT NONE
...
INTERFACE OPERATOR (+)
MODULE PROCEDURE concat
END INTERFACE
\end{verbatim}

including the assignment operator:

\begin{verbatim}
INTERFACE ASSIGNMENT (=) 
subroutine_interface_body
END INTERFACE
\end{verbatim}

This mechanism can also be used for dot-operators:

\begin{verbatim}
INTERFACE OPERATOR (.DIST.)
MODULE PROCEDURE calcdist
END INTERFACE
\end{verbatim}

