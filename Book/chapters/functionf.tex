% -*- latex -*-
%%%%%%%%%%%%%%%%%%%%%%%%%%%%%%%%%%%%%%%%%%%%%%%%%%%%%%%%%%%%%%%%
%%%%
%%%% This TeX file is part of the course
%%%% Introduction to Scientific Programming in C++/Fortran2003
%%%% copyright 2017-9 Victor Eijkhout eijkhout@tacc.utexas.edu
%%%%
%%%% functionf.tex : functions and such in Fortran
%%%%
%%%%%%%%%%%%%%%%%%%%%%%%%%%%%%%%%%%%%%%%%%%%%%%%%%%%%%%%%%%%%%%%

Programs can have subprograms: parts of code that for some reason you
want to separate from the main program. The term for these is
\indextermdef{procedure}, but this is not a language keyword.

If you structure your code in a single
file, this is the recommended structure:

\begin{block}{Subprograms in contains clause}
  \label{sl:contains}
\begin{lstlisting}
Program foo
  < declarations>
  < executable statements >
  Contains
    < procedure definitions >
End Program foo
\end{lstlisting}
\end{block}

That is, procedure are placed after the main program statements,
separated by a \indextermfort{Contains} clause.

In general, these are the  placements of procedures:
\begin{itemize}
\item Internal: after the \indextermfort{Contains} clause of a program
\item In a \indextermfort{Module}; see section~\ref{sec:modulef}.
\item Externally: the procedure is not internal to a \indextermfort{Program} or
  \indextermfort{Module}. In this case it's safest to declare it through an
  \indextermfort{Interface} specification; section~\ref{sec:finterface}.
\end{itemize}

\Level 0 {Subroutines and functions}

Fortran has two types of procedures:
\begin{itemize}
\item Subroutines, which are somewhat like \n{void} functions in C++:
  they can be used to structure the code, and they can only return
  information to the calling environment through their parameters.
\item Functions, which are like C++ functions with a return value.
\end{itemize}

Both types have the same structure, which is roughly the same as of
the main program:
\begin{lstlisting}
subroutine foo( <parameters> )
<variable declarations>
<executable statements>
end subroutine foo
\end{lstlisting}
and
\begin{lstlisting}
returntype function foo( <parameters> )
<variable declarations>
<executable statements>
end subroutine foo
\end{lstlisting}

Exit from a procedure can happen two ways:
\begin{enumerate}
\item the flow of control reaches the end of the procedure body, or
\item execution is finished by an explicit \indextermfort{return}
  statement.
\begin{lstlisting}
subroutine foo()
  print *,"foo"
  if (something) return
  print *,"bar"
end subroutine foo
\end{lstlisting}
\end{enumerate}
The \indextermfort{return} statement is optional in the first case.
The \indextermfort{return} statement is different from C++ in that it does not
indicate the return result of a function.

\begin{exercise}
  Rewrite the above subroutine \n{foo} without a \indextermfort{return} statement.
\end{exercise}

A subroutine is invoked with a \indextermfort{call} statement:
\begin{lstlisting}
call foo()
\end{lstlisting}

\begin{slide}{Subroutines}
  \label{sl:subroutine}
\begin{lstlisting}
subroutine foo()
  implicit none
  print *,"foo"
  if (something) return
  print *,"bar"
end subroutine foo
\end{lstlisting}
\begin{itemize}
\item Looks much like a main program
\item Ends at the end, or when \indextermfort{return} is reached
\item Note: \indextermfort{return} does not return anything
\item Activated with 
\begin{lstlisting}
  call foo()
\end{lstlisting}
\end{itemize}
\end{slide}

\begin{block}{Subroutine with argument}
  \label{sl:fsubr-arg}
  \snippetwithoutput{fprintone}{funcf}{printone}
\end{block}

\begin{block}{Subroutine can change argument}
  \label{sl:fsubr-inout}
  \snippetwithoutput{faddone}{funcf}{addone}
  Parameters are always `by reference'!
\end{block}

\begin{block}{Recursion}
  \label{sl:funcf:recursion}
  Declare function as \indextermfort{Recursive}~\indextermfort{Function}
  %
  \snippetwithoutput{frecursf}{funcf}{fact}
  %
  Note the \indextermfort{result} clause. This prevents ambiguity.
\end{block}

\Level 0 {Return results}

While a \indextermfort{subroutine} can only return information through its parameters,
a~\indexterm{function} procedure returns an explicit result:
\begin{lstlisting}
logical function test(x)
  implicit none
  real :: x

  test = some_test_on(x)
  return ! optional, see above
end function test
\end{lstlisting}
You see that the result is not returned in the \indextermfort{return} statement,
but rather through assignment to the function name. The \indextermfort{return}
statement, as before, is optional and only indicates where the flow of
control ends.

A \indextermdef{function} in Fortran is a procedure that return a
result to its calling program, much like a non-void function in~C++

\begin{block}{Function definition and usage}
  \label{sl:ffunction-def}
  \begin{itemize}
  \item \lstinline$subroutine$ vs \lstinline$function$:\\
    compare \lstinline$void$ functions vs non-void in~C++.
  \item Return type, keyword \indextermfort{function}, name, parameters
  \item Function body has statements
  \item Result is returned by assigning to the function name
  \item Use: \n{y = f(x)}
  \end{itemize}
\end{block}

\begin{block}{Function example}
  \label{sl:ffunction-ex}
  \snippetwithoutput{fplusone}{funcf}{plusone}
\end{block}

A function is not invoked with \indextermfort{call}, but rather through being used
in an expression:
\begin{lstlisting}
if (test(3.0) .and. something_else) ...
\end{lstlisting}
You now have the following cases to make the function known in the
main program:
\begin{itemize}
\item If the function is in a \indextermfort{contains} section, its type is known
  in the main program.
\item If the function is in a module (see section~\ref{sec:modulef}
  below), it becomes known through a \indextermfort{use} statement.
\end{itemize}

\begin{f77note}
  Without modules and \indextermfort{contains} sections, you need to declare the
  function type explitly in the calling program. The safe way is
  through using an \indextermfort{interface} specification.
\end{f77note}

\begin{exercise}
  \label{ex:freadpos}
  Write a program that asks the user for a positive number; negative
  input should be rejected.  Fill in the missing lines in this code
  fragment:
  \snippetwithoutput{readpos}{funcf}{readpos}
\end{exercise}

\begin{block}{Why a `contains' clause?}
  \label{sl:whycontain}
  \begin{multicols}{2}
    \verbatimsnippet{nocontain}
    Warning only, crashes.
    \vfill\columnbreak
    \verbatimsnippet{wrongcontain}
    Error, does not compile
  \end{multicols}
\end{block}

\begin{block}{Why a `contains' clause, take 2}
  \label{sl:whycontain_type}
  \snippetwithoutput{nocontaintype}{funcf}{nocontaintype}
  At best compiler warning if all in the same file\\
  For future reference: if you see very small floating point numbers,
  maybe you have made this error.
\end{block}

\Level 0 {Arguments}

\begin{block}{Procedure arguments}
  \label{sl:farguments}
 Arguments are declared in procedure body:
\begin{lstlisting}
subroutine f(x,y,i)
  implicit none
  integer,intent(in) :: i
  real(4),intent(out) :: x
  real(8),intent(inout) :: y
  x = 5; y = y+6
end subroutine f
! and in the main program
call f(x,y,5)
\end{lstlisting}
declaring the `intent' is optional, but highly advisable.
\end{block}

\begin{block}{Parameter passing}
  \label{sl:fpassing}
  \begin{itemize}
  \item Everything is passed by reference.\\
    Don't worry about large objects being copied.
  \item Optional intent declarations:\\
    Use \indextermfort{in}, \indextermfort{out}, \indextermfort{inout} qualifiers to clarify semantics
    to compiler.
  \end{itemize}
\end{block}

\begin{block}{Fortran nomenclature}
  \label{sl:fortran-dummy}
  The term \indextermsub{dummy}{argument} is what Fortran calls the
  parameters in the procedure definition. The arguments in the
  procedure call are the \indextermsub{actual}{argument}s.
\end{block}

\begin{block}{Intent checking}
  \label{sl:fintent}
  Compiler checks your intent against your implementation. This code
  is not legal:

  \verbatimsnippet{arginwrong}
\end{block}

\begin{block}{Why intent checking?}
\label{sl:intentwhy}
Self-protection: if you state the intended behaviour of a routine, the
compiler can detect programming mistakes.

Allow compiler optimizations:

\begin{multicols}{2}
\begin{lstlisting}
x = f()
call ArgOut(x)
print *,x
\end{lstlisting}
Call to \n{f} removed
\vfill\columnbreak
\begin{lstlisting}
do i=1,1000
  x = ! something
  y1 = .... x ....
  call ArgIn(x)
  y2 = ! same expression as y1
\end{lstlisting}
\n{y2} is same as \n{y1} because \n{x} not changed
\end{multicols}
(May need further specifications, so this is not the prime justification.)
\end{block}

\begin{exercise}
  \label{ex:ffunc-sin-cos}
  Write a subroutine \n{trig} that takes a number~$\alpha$ as input
  and passes $\sin\alpha$ and $\cos\alpha$ back to the calling
  environment.
\end{exercise}

\Level 0 {Types of procedures}

Procedures that are in the main program (or another type of program
unit), separated by a \indextermfort{contains} clause, are known as
\indextermsub{internal}{procedures}. This is as opposed to
\indextermsub{module}{procedures}.

There are also \indexterm{statement functions}, which are
single-statement functions, usually to identify commonly used
complicated expressions in a program unit. Presumably the compiler
will \indexterm{inline} them for efficiency.

The \indextermfort{entry} statement is so bizarre that I refuse to discuss it.

\Level 0 {More about arguments}

\begin{block}{Keyword arguments}
  \label{sl:funcf:keyword}
  \begin{itemize}
  \item Use the name of the \indextermsub{formal}{parameter} as
    keyword.
  \item Keyword arguments have to come last.
  \end{itemize}
  \snippetwithoutput{sayxykw}{funcf}{keyword}
\end{block}

\begin{block}{Optional arguments}
  \label{sl:funcf:optional}
  \begin{itemize}
  \item Extra specifier: \indextermfort{Optional}
  \item Presence of argument can be tested with \indextermfort{Present}
  \end{itemize}
\end{block}

Fortran behaves slightly counterintuitive if you try to initialize
local variables: they initialization is only executed once, and the
second time you call the procedure it will have retained its previous
value. (Technically: the variable has an implicit \indextermfort{Save}
attribute.)

This may trip you up as the following example shows:

\begin{block}{Saved values}
  \label{sl:func-param-save}
  Local variable is initialized only once,\\
  second time it uses its retained value.
  
  \snippetwithoutput{unsafesave}{funcf}{save}
\end{block}


