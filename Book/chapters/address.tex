% -*- latex -*-
%%%%%%%%%%%%%%%%%%%%%%%%%%%%%%%%%%%%%%%%%%%%%%%%%%%%%%%%%%%%%%%%
%%%%
%%%% This TeX file is part of the course
%%%% Introduction to Scientific Programming in C++/Fortran2003
%%%% copyright 2017/8 Victor Eijkhout eijkhout@tacc.utexas.edu
%%%%
%%%% address.tex : about addresses and references
%%%%
%%%%%%%%%%%%%%%%%%%%%%%%%%%%%%%%%%%%%%%%%%%%%%%%%%%%%%%%%%%%%%%%

\Level 0 {Reference}
\label{sec:reference}

This section contains further facts about references, which you have
already seen as a mechanism for parameter passing;
section~\ref{sec:pass-by-ref}.
Make sure you study that material first.

Passing a variable to a routine passes the value; in the routine, the
variable is local.

\verbatimsnippet{localparm}

If you do want to make the change visible in the
%
\indexterm{calling environment}, use a reference:
%
\verbatimsnippet{refparm}
%
There is no change to the calling program. (Some people find this bad,
since you can not see from the use of a function whether it passes
%
\emph{by reference}\index{parameter!passing!by reference}
%
or \emph{by value}\index{parameter!passing!by value}.)

Arrays are always pass by reference:
%
\verbatimsnippet{arrayparm}
%
This is due to the fact that an array is really a pointer to memory;
see~\ref{sec:arraypointer}. On the other hand, a \n{std::vector} is
copied if you do not pass a reference.

The old-style way of doing things:
%
\verbatimsnippet{starparm}

\Level 0 {Pass by reference}

\begin{block}{Reference: change argument}
\label{sl:refarg-change}
\begin{verbatim}
void f( int &i ) { i += 1; };
int main() {
  int i = 2;
  f(i); // makes it 3
\end{verbatim}
\end{block}

\begin{block}{Reference: save on copying}
\label{sl:refarg-nocopy}
\begin{verbatim}
class BigDude {
private:
   vector<double> array(5000000);
}
int main() {
   BigDude big;
   f(big); // whole thing is copied
\end{verbatim}
Instead write:
\begin{verbatim}
void f( BigDude &thing ) { .... };
\end{verbatim}
Prevent changes:
\begin{verbatim}
void f( const BigDude &thing ) { .... };
\end{verbatim}
\end{block}

\Level 0 {Reference to class members}
\label{sec:class-ref}

Here is the naive way of returning a class member:
\begin{verbatim}
class Object {
private:
  SomeType thing;
public:
  SomeType get_thing() {
    return thing; };
};
\end{verbatim}
The problem here is that the return statement makes a copy of
\n{thing}, which can be expensive. Instead, it is better to return the
member by \emph{reference}\index{reference!to class member}:
\begin{verbatim}
  SomeType &get_thing() { 
    return thing; };
\end{verbatim}
The problem with this solution is that the calling program can now
alter the private member. To prevent that, use a
\emph{const reference}\index{reference!const!to class member}:
%
\verbatimsnippet{getsetref}

Output of this fragment:
\begin{verbatim}
Contained int is now: 4
Contained int is now: 17
\end{verbatim}

In the above example, the function giving a reference was used in the left-hand side of
an assignment. If you would use it on the right-hand side, you would
not get a reference. The result of an expression can not be a
reference.

Let's again make a class where we can get a reference to the
internals:
%
\verbatimsnippet{rhsrefclass}

Now we explore various ways of using that reference on the right-hand
side:
%
\snippetwithoutput{rhsref}{func}{rhsref}
%
You see that, despite the fact that the method \n{data} was defined as
returning a reference, you still need to indicate whether the
left-hand side is a reference.

See section~\ref{sec:constparam} for the interaction between \n{const}
and references.

\Level 0 {Reference to array members}
\label{sec:overloadbracket}

You can define various operator, such as \verb.+-*/. arithmetic
operators, to act on classes, with your own provided implementation;
see section~\ref{sec:operatordef}. You can also define the parentheses
and square brackets operators, so make your object look like a
function or an array respectively.

These mechanisms can also be used to provide safe acess to arrays
and/or vectors that are private to the object.

Suppose you have an object that contains an \n{int} array. You can
return an element by defining the subscript (square bracket) operator
for the class:
%
\verbatimsnippet{refindexcopy}
%
Note that \n{return array[i]} will return a copy of the array element,
so it is not possible to write
\begin{verbatim}
myobject[5] = 6;
\end{verbatim}
For this we need to return a reference to the array element:
%
\verbatimsnippet{refindexref}

Another reason for wanting to return a reference is to prevent the
\emph{copy of the return result}\index{return!makes copy}
that is induced by the \n{return} statement.
In this case, you may not want to be able to alter the object
contents, so you can return a \indextermsub{const}{reference}:
%
\verbatimsnippet{refindexconstref}

\Level 0 {rvalue references}

See the chapter about obscure stuff; section~\ref{sec:rvalue-ref}.
