% -*- latex -*-
%%%%%%%%%%%%%%%%%%%%%%%%%%%%%%%%%%%%%%%%%%%%%%%%%%%%%%%%%%%%%%%%
%%%%
%%%% This TeX file is part of the course
%%%% Introduction to Scientific Programming in C++/Fortran2003
%%%% copyright 2017-9 Victor Eijkhout eijkhout@tacc.utexas.edu
%%%%
%%%% pointer.tex : about pointers
%%%%
%%%%%%%%%%%%%%%%%%%%%%%%%%%%%%%%%%%%%%%%%%%%%%%%%%%%%%%%%%%%%%%%

The term pointer is used to denote a reference to a quantity.  This
chapter will explain pointers, and give some uses for them.  If you
don't already speak~C, skip to the first subsection.

If you do
already know~C, we remark that there is less need for pointers in C++
than there was in~C.

\begin{itemize}
\item To pass an argument
  \emph{by reference}\index{parameter!passing!by reference},
  use a \emph{reference}\index{reference!argument}.
  Section~\ref{sec:passing}.
\item Strings are done through \n{std::string}, not character arrays;
  see~\ref{ch:string}.
\item Arrays can largely be done through \n{std::vector}, rather than
  \indextermtt{malloc}; see~\ref{ch:array}.
\item Traversing arrays and vectors can be done with ranges;
  section~\ref{sec:arrayrange}.
\item Anything that obeys a scope should be created through a
  \indexterm{constructor}, rather than using \indextermtt{malloc}.
\end{itemize}

Legitimate needs:
\begin{itemize}
\item Linked lists and \acp{DAG}; see the example in section~\ref{sec:linklist}.
\item Objects on the heap.
\item Use \indextermtt{nullptr} as a signal.
\end{itemize}

\Level 0 {The `arrow' notation}

\begin{block}{Members from pointer}
  \begin{itemize}
  \item If \n{x} is object with member~\n{y}:\\ \n{x.y}
  \item If \n{xx} is pointer to object with member~\n{y}:\\ \n{xx->y}
  \item In class methods \n{this} is a pointer to the object, so:
\begin{lstlisting}
class x {
  int y;
  x(int v) {
    this->y = v; }
}
\end{lstlisting}
\item Arrow notation works with old-style pointers and new
  shared/unique pointers.
  \end{itemize}
\end{block}

\Level 0 {Making a shared pointer}
\label{sec:shared_ptr}

Smart pointers are used the same way as old-style pointers in~C.
If you have an
object \n{Obj X} with a member \n{y}, you access that with \n{X.y}; if you
have a pointer~\n{X} to such an object, you write \n{X->y}.

So what is the type of this latter~\n{X} and how did you create it?

\begin{block}{Creating a shared pointer}
  \label{sl:make-shared}
  Allocation and pointer in one:
\begin{lstlisting}
shared_ptr<Obj> X =
    make_shared<Obj>( /* constructor args */ );
  // or:
auto X = make_shared<Obj>( /* args */ );

X->method_or_member;
\end{lstlisting}
\end{block}

(It is also possible to cast a \indextermtt{new} object to shared
pointer, but that method is not recommended.)

Using shared pointers requires at the top of your file:
\begin{lstlisting}
#include <memory>
using std::shared_ptr;
using std::make_shared;
\end{lstlisting}

\begin{block}{Simple example}
  \label{sl:shared-ptr}
  \snippetwithoutput{pointx}{pointer}{pointx}
\end{block}

\begin{block}{Pointers to arrays}
  \label{sl:shared-vector}
  The constructor syntax is a little involved for vectors:
\begin{lstlisting}
auto x = make_shared<vector<double>>(vector<double>{1.1,2.2});
\end{lstlisting}
\end{block}

\Level 1 {Pointers and arrays}

\begin{slide}{Shared pointers for arrays}
This offers a safer version of \n{new}. But you should use a
\n{vector} anyway:
\begin{lstlisting}
shared_ptr<myobject> obj_ptr
    = make_shared<myobject>(x);
    // = shared_ptr<myobject>( new myobject(x) );
obj_ptr->mymethod(1.1);
cout << obj_ptr->member << endl;

auto array = shared_ptr<double>( new double[100] );
// ILLEGAL: array[1] = 2.14;
array->at(2) = 3.15;
\end{lstlisting}
\end{slide}

\begin{block}{Linked lists}
  \label{sl:linkedlistimpl}
  The prototypical example use of pointers is in linked lists. Let a
  class \n{Node} be given:
  %
  \begin{multicols}{2}
    \verbatimsnippet{simplenodeimpl}
    \columnbreak
    \verbatimsnippet{linkprintrecursive}
    \vfill\hbox{}
  \end{multicols}
\end{block}

\begin{block}{List usage}
\label{sl:linkedlistuse}
  Example use:
  %
  \answerwithoutput{linksimple1}{tree}{simple}
\end{block}

\begin{block}{Linked lists and recursion}
  \label{sl:linkedlistrecursive}
  Many operations on linked lists can be done recursively:
  %
  \verbatimsnippet{linklengthrecursive}
\end{block}

\begin{exercise}
  \label{ex:linkedlist1}
  Write a recursive \n{append} method that appends a node to the end
  of a list:
  %
  \answerwithoutput{linkappend}{tree}{append}
\end{exercise}

\begin{exercise}
  \label{ex:linkedlist2}
  Write a recursive \n{insert} method that inserts a node in a list, such that
  the list stays sorted:
  %
  \answerwithoutput{linkinsert}{tree}{insert}
\end{exercise}

\begin{exercise}
  \label{ex:linkedlist3}
  For a more sophisticated approach to linked lists, do the exercises
  in section~\ref{sec:linklist}.
\end{exercise}

\begin{exercise}
  If you are doing the prime numbers project (chapter~\ref{ch:prime})
  you can now do exercise~\ref{sec:streamsieve}.
\end{exercise}

\Level 1 {Smart pointers versus address pointers}

\begin{block}{Pointers don't go with addresses}
  \label{sl:shareaddress}

  The oldstyle \n{&y} address pointer can not be made smart:
  %
  \verbatimsnippet{shareaddress}
  %
  gives:
\begin{verbatim}
address(56325,0x7fff977cc380) malloc: *** error for object
    0x7ffeeb9caf08: pointer being freed was not allocated
\end{verbatim}
\end{block}


\Level 0 {Garbage collection}

The big problem with C-style pointers is the chance of a
\indexterm{memory leak}. If a pointer to a block of memory goes out of
scope, the block is not returned to the \ac{OS}, but it is no longer
accessible.
\begin{lstlisting}
// the variable `array' doesn't exist
{
  // attach memory to `array':
  double *array = new double[25];
  // do something with array
}
// the variable `array' does not exist anymore
// but the memory is still reserved.
\end{lstlisting}
Shared and unique
pointers do not have this problem: if they go out of scope, or are
overwritten, the destructor on the object is called, thereby releasing
any allocated memory.

An example.

\begin{block}{Reference counting illustrated}
  \label{sl:construct-destruct-trace}
  We need a class with constructor and destructor tracing:
  \verbatimsnippet{thingcall}
\end{block}

\begin{block}{Pointer overwrite}
  \label{sl:shared-ptr-overwrite}
  Let's create a pointer and overwrite it:
  %
  \snippetwithoutput{shareptr1}{pointer}{ptr1}
\end{block}

However, if a pointer is copied, there are two pointers to the same
block of memory, and only when both disappear, or point elsewhere, is
the object deallocated.

\begin{block}{Pointer copy}
  \label{sl:shared-ptr-copy}
  \snippetwithoutput{shareptr2}{pointer}{ptr2}
\end{block}

\Level 0 {More about pointers}

\Level 1 {Get the pointed data}

Most of the time, accessing the target of the pointer through the
arrow notation is enough. However, if you actually want the object,
you can get it with \indextermtt{get}. Note that this does not give
you the pointed object, but a traditional pointer.

\begin{block}{Getting the underlying pointer}
  \label{sl:pointer-get}
\begin{lstlisting}
X->y;
// is the same as
X.get()->y;
// is the same as
( *X.get() ).y;
\end{lstlisting}

\snippetwithoutput{pointy}{pointer}{pointy}
\end{block}

\Level 1 {Example: linked lists}

The standard example of pointer manipulation is `linked lists'. This
is discussed in some detail in section~\ref{sec:linklist}.

\Level 0 {Advanced topics}

\Level 1 {Unique pointers}

Shared pointers are fairly easy to program, and they come with lots of
advantages, such as the automatic memory management. However, they
have more overhead than strictly necessary because they have a
\indexterm{reference count} mechanism to support the memory
management. Therefore, there exists a \indextermsub{unique}{pointer},
\indextermtt{unique_ptr}, for cases where an object will only ever be
`owned' by one pointer. In that case, you can use a C-style
\indextermsub{bare}{pointer} for non-owning references.

\Level 1 {Shared pointer to `this'}

Inside an object method, the object is accessible as
\indextermtt{this}. This is a pointer in the classical sense. So what
if you want to refer to `this' but you need a shared pointer?

For instance, suppose you're writing a linked list code, and your
\n{node} class has a method \n{prepend_or_append} that gives a shared
pointer to the new head of the list.

\begin{slide}{Linked list code}
  \label{sl:share-ptr-node}  
\begin{lstlisting}
node *node::prepend_or_append(node *other) {
  if (other->value>this->value) {
    this->tail = other;
    return this;
  } else {
    other->tail = this;
    return other;
  }
};
\end{lstlisting}
Can we do this with shared pointers?
\end{slide}

Your code would start something
like this, handling the case where the new node is appended to the current:
\begin{lstlisting}
shared_pointer<node> node::append
    ( shared_ptr<node> other ) {
  if (other->value>this->value) {
    this->tail = other;
\end{lstlisting}
But now you need to return this node, as a shared pointer. But
\lstinline{this} is a \lstinline{node*}, not a \lstinline{shared_ptr<node>}.

\begin{slide}{A problem with shared pointers}
  \label{sl:share-ptr-node-sh}
\begin{lstlisting}
shared_pointer<node> node::prepend_or_append
    ( shared_ptr<node> other ) {
  if (other->value>this->value) {
    this->tail = other;
\end{lstlisting}
So far so good. However, \lstinline{this} is a \lstinline{node*}, not a
\lstinline{shared_ptr<node>}, so
\begin{lstlisting}
    return this;
\end{lstlisting}
returns the wrong type.
\end{slide}

The solution here is that you can return
\begin{lstlisting}
    return this->shared_from_this();
\end{lstlisting}
if you have defined your node class to inherit from what probably
looks like magic:
\begin{lstlisting}
class node : public enable_shared_from_this<node>
\end{lstlisting}

\begin{slide}{Solution: shared from this}
  \label{sl:share-ptr-node-from}
  It is possible to have a `shared pointer to this' if you
  define your node class with (warning, major magic alert):
\begin{lstlisting}
class node : public enable_shared_from_this<node> {
\end{lstlisting}
This allows you to write:
\begin{lstlisting}
    return this->shared_from_this();
\end{lstlisting}
\end{slide}

\Level 1 {Null pointer}

In C there was a macro \indexterm{NULL} that, only by convention, was
used to indicate
\emph{null pointers}\index{pointer!null}:
pointers that do not point to anything.
C++ has the \indextermtt{nullptr}, which is an object of type
\lstinline{std::}\indextermtt{nullptr_t}.

There are some scenarios where this is useful, for instance, with
polymorphic functions:
\begin{lstlisting}
void f(int);
void f(int*);
\end{lstlisting}
Calling \lstinline{f(ptr)} where the point is \lstinline{NULL}, the first function is
called, whereas with \lstinline{nullptr} the second is called.

\begin{slide}{Null pointer}
  \label{sl:cpp-nullptr}
  C++ has the \indextermtt{nullptr}, which is an object of type
  \lstinline{std::}\indextermtt{nullptr_t}.

\begin{lstlisting}
void f(int);
void f(int*);
  f(NULL);    // calls the int version
  f(nullptr); // calls the ptr version
\end{lstlisting}
Note: dereferencing is undefined behaviour; does not throw an exception.
\end{slide}

\Level 1 {Void pointer}

The need for \emph{void pointers}\index{pointer!void!in C++}
is a lot less in C++ than it was in~C. For
instance, contexts can often be modeled with captures in closures
(chapter~\ref{sec:lambda}). If you really need a pointer that does not
\textit{a~priori} knows what it points to, use \indextermtt{std::any},
which is usually smart enough to call destructors when needed.

\begin{slide}{Void pointer}
  \label{sl:void-ptr}
  Use \lstinline{std::any} instead of void pointers.
\end{slide}

\Level 1 {Pointers to non-objects}

In the introduction to this chapter we argued that many of the uses
for pointers that existed in~C have gone away in C++, and the main one
left is the case where multiple objects share `ownership' of some
other object.

You can still make shared pointers to scalar data, for instance to an
array of scalars. You then get the advantage of the memory management,
but you do not get the \lstinline{size} function and such that you would have
if you'd used a \lstinline{vector} object.

Here is an example of a pointer to a solitary double:
%
\snippetwithoutput{ptrdouble}{pointer}{ptrdouble}
%
It is possible to initialize that double:
%
\snippetwithoutput{ptrdoubleinit}{pointer}{ptrdoubleinit}

