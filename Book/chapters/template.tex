% -*- latex -*-
%%%%%%%%%%%%%%%%%%%%%%%%%%%%%%%%%%%%%%%%%%%%%%%%%%%%%%%%%%%%%%%%
%%%%
%%%% This TeX file is part of the course
%%%% Introduction to Scientific Programming in C++/Fortran2003
%%%% copyright 2017/8 Victor Eijkhout eijkhout@tacc.utexas.edu
%%%%
%%%% template.tex : on templates
%%%%
%%%%%%%%%%%%%%%%%%%%%%%%%%%%%%%%%%%%%%%%%%%%%%%%%%%%%%%%%%%%%%%%

Sometimes you want a function or a class based on more than one
different datatypes. For instance, in chapter~\ref{ch:array} you saw
how you could create an array of ints as \n{vector<int>} and
of doubles as \n{vector<double>}. Here you will learn the mechanism
for that.

\begin{block}{Templated type name}
  \label{sl:template-gen}
  If you have multiple routines that do `the same' for multiple types,
  you want the type name to be a variable. Syntax:
\begin{verbatim}
template <typename yourtypevariable>
// ... stuff with yourtypevariable ...
\end{verbatim}
\end{block}

Historically \n{typename} was \n{class} but that's confusing.

\Level 0 {Templated functions}

\begin{block}{Example: function}
  \label{sl:template-fun}
  Definition:
\begin{verbatim}
template<typename T>
void function(T var) { cout << var << end; }
\end{verbatim}
Usage:
\begin{verbatim}
int i; function(i);
double x; function(x);
\end{verbatim}
and the code will behave as if you had defined \n{function} twice,
once for \n{int} and once for \n{double}.
\end{block}

\begin{exercise}
  \label{ex:eps-template}
  Machine precision, or `machine epsilon', is sometimes defined as the
  smallest number~$\epsilon$ so that $1+\epsilon>1$ in computer
  arithmetic.

  Write a templated function \n{epsilon} so that the following code
  prints out the values of the machine precision for the \n{float} and
  \n{double} type respectively:
%
\snippetwithoutput{epsprint}{template}{eps}
\end{exercise}

\Level 0 {Templated classes}

The most common use of templates is probably to define templated
classes.
You have in fact seen this mechanism in action:
\begin{block}{Templated vector}
  \label{sl:template-vector}
  the \ac{STL} contains
  in effect
\begin{verbatim}
template<typename T>
class vector {
private:
  // data definitions omitted
public:
  T at(int i) { /* return element i */ };
  int size() { /* return size of data */ };
  // much more
}
\end{verbatim}
\end{block}

\Level 0 {Specific implementation}

\begin{verbatim}
template <typename T>
void f(T);
template <> 
void f(char c) { /* code with c */ };
template <>
void f(double d) { /* code with d */ };
\end{verbatim}

\Level 0 {Templating over non-types}

THESE EXAMPLES ARE NOT GOOD.

See:
\url{https://www.codeproject.com/Articles/257589/An-Idiots-Guide-to-Cplusplus-Templates-Part}

\begin{block}{Templating a value}
  Templeting over integral types, not double.

  The templated quantity is a value:
  %
  \verbatimsnippet{itemplate}
\end{block}

\begin{exercise}
  Write a class that contains an array. The length of the array should
  be templated.
\end{exercise}
