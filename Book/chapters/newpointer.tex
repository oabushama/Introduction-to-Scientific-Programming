% -*- latex -*-
%%%%%%%%%%%%%%%%%%%%%%%%%%%%%%%%%%%%%%%%%%%%%%%%%%%%%%%%%%%%%%%%
%%%%
%%%% This TeX file is part of the course
%%%% Introduction to Scientific Programming in C++/Fortran2003
%%%% copyright 2017/8 Victor Eijkhout eijkhout@tacc.utexas.edu
%%%%
%%%% newpointer.tex : about C++ proper pointers
%%%%
%%%%%%%%%%%%%%%%%%%%%%%%%%%%%%%%%%%%%%%%%%%%%%%%%%%%%%%%%%%%%%%%

\Level 0 {Safer pointers in C++}
\label{sec:shared_ptr}

The C~language had a pointer concept that was fraught with danger. For
instance, it was prone to
%
\emph{memory leaks}\index{memory leak}
if you didn't code very carefully. (See the example in 
section~\ref{sec:memleak}.)
%
The \indexterm{C++11} standard has mechanisms that can
help solve this problem\footnote{A mechanism along these lines already
  existed in the `weak pointer', but it was all but unusable.}.

A `shared pointer' is a pointer that keeps count of how many times the
object is pointed to. If one of these pointers starts pointing
elsewhere, the shared pointer decreases this `reference count'. If the
reference count reaches zero, the object is deallocated or destroyed.

\begin{block}{Shared pointers}
  \label{sl:shared-ptr}
Shared pointers look like regular pointers:
\begin{verbatim}
#include <memory>

std::shared_ptr<myobject> obj_ptr
    = std::make_shared<myobject>(x);
    // = std::shared_ptr<myobject>( new myobject(x) );
obj_ptr->mymethod(1.1);
cout << obj_ptr->member << endl;

auto array = std::shared_ptr<double>( new double[100] );
// ILLEGAL: array[1] = 2.14;
array->at(2) = 3.15;
\end{verbatim}
\end{block}

As an illustration, let's take a class that reports its construction
and destruction:
%
\begin{block}{Reference counting illustrated}
  \label{sl:construct-destruct-trace}
  We need a class with constructor and destructor tracing:
  \verbatimsnippet{thingcall}
\end{block}

%
and
\begin{block}{Pointer overwrite}
  \label{sl:shared-ptr-overwrite}
  Let's create a pointer and overwrite it:
  %
  \snippetwithoutput{shareptr1}{pointer}{ptr1}
\end{block}
%
You see that overwriting the pointer does not cause a
\indexterm{memory leak}, as would happen with C-style pointers, but
causes destruction of the object it points to.

Next we overwrite a pointer, but after having copied it. Now the
destructor will not be called until all references to the object
--~both the original pointer and the copy~-- have been overwritten.
%
\begin{block}{Pointer copy}
  \label{sl:shared-ptr-copy}
  \snippetwithoutput{shareptr2}{pointer}{ptr2}
\end{block}

\begin{remark}
  We can use the shared pointer mechanism for arrays. This is new to
  c++20 but I don't see the point.
\begin{block}{Shared pointer to vector}
    \label{sl:shared-ptr-doubles}
    There is not much point in a shared pointer to a simple array.
    %
    \snippetwithoutput{shareptr3a}{pointer}{ptr3a}
\end{block}
\begin{block}{Shared pointer to vector}
    \label{sl:shared-ptr-things}
    I don't see the point of this either
    %
    \snippetwithoutput{shareptr3b}{pointer}{ptr3b}
\end{block}
\end{remark}

\Level 1 {Null pointer}

In C there was a macro \indexterm{NULL} that, only by convention, was
used to indicate
\emph{null pointers}\index{pointer!null}:
pointers that do not point to anything.
C++ has the \indextermtt{nullptr}, which is an object of type
\n{std::}\indextermtt{nullptr_t}.

There are some scenarios where this is useful, for instance, with
polymorphic functions:
\begin{verbatim}
void f(int);
void f(int*);
\end{verbatim}
Calling \n{f(ptr)} where the point is \n{NULL}, the first function is
called, whereas with \n{nullptr} the second is called.

\Level 1 {Shared from this}

There is a problem with shared pointers in the case where a method has
to produce a pointer. For instance, to program
\emph{linked lists}\index{list!linked} you want a
method:
\begin{slide}{Linked list code}
  \label{sl:share-ptr-node}  
\begin{verbatim}
node *node::prepend_or_append(node *other) {
  if (other->value>this->value) {
    this->tail = other;
    return this;
  } else {
    other->tail = this;
    return other;
  }
};
\end{verbatim}
Can we do this with shared pointers?
\end{slide}

\begin{block}{A problem with shared pointers}
  \label{sl:share-ptr-node-sh}
\begin{verbatim}
shared_pointer<node> node:prepend_or_append
    ( shared_ptr<node> other ) {
  if (other->value>this->value) {
    this->tail = other;
\end{verbatim}
So far so good. However, \n{this} is a \n{node*}, not a
\n{shared_ptr<node>}, so
\begin{verbatim}
    return this;
\end{verbatim}
returns the wrong type.
\end{block}

\begin{block}{Solution: shared from this}
  \label{sl:share-ptr-node-from}
  It is possible to have a `shared pointer to this' if you
  define your node class with (warning, major magic alert):
\begin{verbatim}
class node : public enable_shared_from_this<node> {
\end{verbatim}
This allows you to write:
\begin{verbatim}
    return this->shared_from_this();
\end{verbatim}
\end{block}
