% -*- latex -*-
%%%%%%%%%%%%%%%%%%%%%%%%%%%%%%%%%%%%%%%%%%%%%%%%%%%%%%%%%%%%%%%%
%%%%
%%%% This TeX file is part of the course
%%%% Introduction to Scientific Programming in C++/Fortran2003
%%%% copyright 2017/8 Victor Eijkhout eijkhout@tacc.utexas.edu
%%%%
%%%% object.tex : we get down to OOP!
%%%%
%%%%%%%%%%%%%%%%%%%%%%%%%%%%%%%%%%%%%%%%%%%%%%%%%%%%%%%%%%%%%%%%

\Level 0 {What is an object?}
\label{sec:object}

You learned about \n{struct}s (chapter~\ref{ch:struct}) as a way of
abstracting from the elementary data types. The elements of a
structure were called its members.

You also saw that it is possible to write
functions that work on structures. Since these functions are really
tied to the definition of the \n{struct}, shouldn't there be a way to
make that tie explicitly?

That's what an object is:
\begin{itemize}
\item An object is like a structure in that it has data members.
\item An object has \emph{methods}\index{methods (of an object)} which
  are the functions that operate on that object.
\end{itemize}
C++ does not actually have a `object' keyword; instead you define a
class with the \indextermttdef{class} keyword, which describes all the objects
of that class.

First of all, you can make an object look pretty much like a
structure:
%
\snippetwithoutput{pointstruct}{geom}{pointstruct}

Observations about the above code snippet:
\begin{itemize}
\item Again we have separate definition of the class and declaration
  of the objects. You define the class only once, after which you can
  make as many objects of that class as you want.
\item There are data members. We will get to the \n{public} in a
  minute.
\item You make an object of that class by using the class name as the
  datatype.
\item The data members can be accessed with the period.
\end{itemize}

\begin{slide}{Classes look a bit like structures}
  \label{sl:class-struct}
  \snippetwithoutput{pointstruct}{geom}{pointstruct}

  Class definition versus object declaration.\\
  We'll get to that `\n{public}' in a minute.
\end{slide}

\Level 1 {Constructor}

Next we'll look at a syntax for creating class objects that is new. If
you create an object, you actually call a function that has the same
name as the class: the \indexterm{constructor}.
Above you had a declaration \n{Vector p1} which was not just a
declaration: it called the so-called \indextermsub{default}{constructor},
which has no arguments, and does nothing.

Usually you write your own constructor,
for instance to initialize data members:
%
%\snippetwithoutput{pointconstruct}{geom}{pointconstruct}
\verbatimsnippet{pointconstruct}
\verbatimsnippet{pointconstructuse}

\begin{slide}{Class initialization and use}
  \label{sl:class-construct}
  Use a \indexterm{constructor}: function with same name as the class.
  \verbatimsnippet{pointprivate}
  \verbatimsnippet{pointprivatedefine}
\end{slide}

\Level 1 {Public and private}

In the first example above, the data members of the \n{Vector} class were
declared \n{public}, meaning that they are accessable from the calling
(main) program. While this is initially convenient for coding, it is a bad idea
in the long term. For a variety of reasons it is good practice to
separate interface and implementation of a class.

\begin{block}{Example of accessor functions}
  \label{sl:pointaccess}
  Getting and setting of members values is done through accessor functions:
  \begin{multicols}{2}
    \verbatimsnippet{pointprivate}
    \verbatimsnippet{pointprivateset}
    \verbatimsnippet{pointprivatedefine}
  \end{multicols}
  Usage:
  %
  \verbatimsnippet{pointprivatesetuse}
\end{block}

\begin{block}{Public versus private}
  \label{sl:interfaceimpl}
  \begin{itemize}
  \item Implementation: data members, keep \n{private},
  \item Interface: \n{public} functions to get/set data.
  \item Protect yourself against inadvertant changes of object data.
  \item Possible to change implementation without rewriting calling code.
  \end{itemize}
\end{block}

\begin{block}{Private access gone wrong}
  \label{sl:privatenogood}
  We make a class with two members that sum to one.\\
  You don't want to be able to change just one of them!
\begin{verbatim}
class SumIsOne {
public: 
  float x,y;
  SumIsOne( double xx ) { x = xx; y = 1-x; };
}
int main() {
  SumIsOne pointfive(.5);
  pointfive.y = .6;
}
\end{verbatim}
In general: enforce predicates on the members.
\end{block}

\Level 1 {Initialization}

\begin{block}{Member default values}
  \label{sl:class-defval}
  Class members can have default values, just like ordinary variables:
\begin{verbatim}
class Point {
private:
  float x=3., y=.14;
private:
  // et cetera
}
\end{verbatim}
  Each object will have its members initialized to these values.
\end{block}

\begin{block}{Member initialization}
  \label{sl:class-init}
  Other syntax for initialization:
  \verbatimsnippet{classpointinit}
  Allows for reuse of names:
  \snippetwithoutput{classpointinitxy}{geom}{pointinitxy}

  \emph{Initializer lists}\index{initializer list} can be used as denotations.
  \verbatimsnippet{initcurlypoint}
\end{block}

If the data members follow a \indextermtt{public} directive, code
outside the class can access the data members, both for getting and
setting their values. This may be convenient for coding, but it's not
a clean coding style. It's better to make data members
\indextermtt{private}, and use \indexterm{accessor} functions to get
and set values.

\begin{block}{Private data}
  \label{sl:class-private}
  \verbatimsnippet{pointprivate}
\end{block}

\Level 1 {Accessor functions}

Remember the lines:
\begin{verbatim}
private:
  double vx,vy;
\end{verbatim}
This implies that \n{vx,vy} are not accessible from anything outside
the object. In order to change them we need functions, which we call
\index{accessor} functions:

\begin{block}{Accessor for setting private data}
  \label{sl:class-private-set}
  Class methods:
  \verbatimsnippet{pointprivateset}
\end{block}

\Level 1 {Methods}

With the accessors, you have just seen a first example of a class
\indextermdef{method}: a function that is only defined for objects of
that class, and that have access to the private data of that object.
Let's now look at more meaningful methods. For instance, for the
\n{Vector} class you can define functions such as \n{length} and
\n{angle}. 
%
\snippetwithoutput{pointfunc}{geom}{pointfunc}

By making these functions public, and the data members
private, you define an \acf{API} for the class:
\begin{itemize}
\item You are defining operations for that class; they are the only
  way to access the data of the object.
\item The methods can use the data of the object, or alter it. All
  data members, even when declared \n{private}, or global to the methods.
\item  Data members declared \n{private} are not accessible from outside the
  object.
\end{itemize}

\begin{slide}{Functions on objects}
  \label{sl:obj-func}
  \snippetwithoutput{pointfunc}{geom}{pointfunc}
  %\verbatimsnippet{pointfunc}
  We call such internal functions `methods'.\\
  Data members, even \n{private}, are global to the methods.
\end{slide}

So far you have seen methods that use the data members of an object to
return some quantity. It is also possible to alter the members. 
For instance, you may want to scale a vector by some amount:
%
  \snippetwithoutput{pointscaleby}{geom}{pointscaleby}

\begin{slide}{Methods that alter the object}
  \label{sl:obj-func-on}
  \snippetwithoutput{pointscaleby}{geom}{pointscaleby}
\end{slide}

\begin{block}{Direct alteration of internals}
  \label{sl:obj-return-ref}
  Return a reference to a private member:
\begin{verbatim}
class Vector {
private:
  double vx,vy;
public:
  double &x() { return vx; };
};
int main() {
  Vector v;
  v.x() = 3.1;
}
\end{verbatim}
\end{block}

\begin{block}{Reference to internals}
  \label{sl:obj-return-const-ref}
  Returning a reference saves you on copying.\\
  Prevent unwanted changes by using a `const reference'.
\begin{verbatim}
class Grid {
private:
  vector<Point> thepoints;
public:
  const vector<Point> &points() {
    return thepoints; };
};
int main() {
  Grid grid;
  cout << grid.points()[0];
  // grid.points()[0] = whatever ILLEGAL
}
\end{verbatim}
\end{block}

The methods you have seen so far only returned elementary
datatypes. It is also possible to return an object, even from the same
class. For instance, instead of scaling the members of a vector object, you
could create a new object based on the scaled members:
%
\snippetwithoutput{pointscale}{geom}{pointscale}

\begin{slide}{Methods that create a new object}
  \label{sl:obj-return}
  \snippetwithoutput{pointscale}{geom}{pointscale}
\end{slide}

\Level 1 {Default constructor}

One of the more powerful ideas in C++ is that there can be more than
one constructor. You will often be faced with this whether you want or
not. The following code looks plausible:
%
\verbatimsnippet{pointdef2}
%
but it will give an error message during compilation. The reason is
that 
\begin{verbatim}
Vector p;
\end{verbatim}
calls the default constructor. Now that you have defined your own
constructor, the default constructor no longer exists. So you need to
define it explicitly:
%
\verbatimsnippet{pointdef1}

\begin{slide}{Default constructor}
  \label{sl:obj-def-construct}
\small
  \verbatimsnippet{pointdef2}
  gives (g++; different for intel):
\begin{verbatim}
pointdefault.cxx: In function 'int main()':
pointdefault.cxx:32:21: error: no matching function for call to
                'Vector::Vector()'
   Vector p1(1.,2.), p2;
\end{verbatim}
The problem is with \n{p2}. How is it created? We need to define two constructors:
\verbatimsnippet{pointdef1}
\end{slide}

\Level 1 {Accessors}

It is a good idea to keep data members private, and use accessor
functions.

\begin{block}{Use accessor functions!}
  \label{sl:repr-independent}
\begin{verbatim}
class PositiveNumber { /* ... */ }
class Point {
private:
  // data members
public:
  Point( float x,float y ) { /* ... */ };
  Point( PositiveNumber r,float theta ) { /* ... */ };
  float get_x() { /* ... */ };
  float get_y() { /* ... */ };
  float get_r() { /* ... */ };
  float get_theta() { /* ... */ };
};
\end{verbatim}
  Functionality is independent of implementation.
\end{block}

\begin{comment}
  \begin{exercise}
    \label{ex:geom:twoconstruct}
    Write a \n{Point} class that has two constructors:
\begin{verbatim}
class Point {
private:
  // data members
public:
  Point( float x,float y ) { /* ... */ };
  Point( float r,float theta ) { /* ... */ };
  float get_x() { /* ... */ };
  float get_y() { /* ... */ };
  float get_r() { /* ... */ };
  float get_theta() { /* ... */ };
};
\end{verbatim}
Use $r,\theta$ for the private variables, do not store $x,y$.
  \end{exercise}
\end{comment}

\begin{exercise}
  If you are doing the prime project (chapter~\ref{ch:prime}),
  now is a good time to do exercise~\ref{ex:prime:sequence}.
\end{exercise}

\emph{The remainder of this section is advanced material. Make sure
  you have studied section~\ref{sec:class-ref}.}

It is a good idea to make the data in an object private,
to control outside access to it.
\begin{itemize}
\item Sometimes this private data is auxiliary, and there is no reason
  for outside access.
\item Sometimes you do want outside access, but you want to precisely
  control by whom.
\end{itemize}

Accessor functions:
\begin{verbatim}
class thing {
private:
 float x;
public:
 float get_x() { return x; };
 void set_x(float v) { x = v; }
};
\end{verbatim}
This has advantages:
\begin{itemize}
\item You can print out any time you get/set the value; great for
  debugging:
\begin{verbatim}
void set_x(float v) {
  cout << "setting: " << v << endl;
  x = v; }
\end{verbatim}
\item You can catch specific values: if \n{x} is always supposed to be
  positive, print an error (throw an exception) if nonpositive.
\end{itemize}

Having two accessors can be a little clumsy. Is it possible to use the
same accessor for getting and setting?

\begin{block}{Setting members through accessor}
  \label{sl:setmember}
  %
  Use a single accessor for getting and setting:

  \snippetwithoutput{objaccessref}{object}{accessref}
  %
  The function \n{the_x} returns a reference to the internal
  variable~\n{x}.
\end{block}

Of course you should only do this if you want the internal variable to
be directly changable!

This is a good time to do the exercises in section~\ref{ex:pointfunc}.

\Level 1 {Accessability}

\begin{slide}{Access levels}
  \label{sl:private-etc}
  Methods and data can be 
  \begin{itemize}
  \item private, because they are only used internally;
  \item public, because they should be usable from outside a class
    object, for instance in the main program;
  \item protected, because they should be usable in derived classes (see
    section~\ref{sec:derive-method}).
  \end{itemize}
\end{slide}

You can have multiple methods with the same name, as long as they can
be distinguished by their argument types. This is known as \indexterm{overloading}.

\Level 1 {Operator overloading}
\label{sec:operatordef}

Instead of writing 
\begin{verbatim}
myobject.plus(anotherobject)
\end{verbatim}
you can actually redefine the \n{+} operator so that
\begin{verbatim}
myobject + anotherobject
\end{verbatim}
is legal.

The syntax for this is
\begin{block}{Operator overloading}
  \label{sl:object-operator}
\begin{verbatim}
<returntype> operator<op>( <argument> ) { <definition> }
\end{verbatim}
For instance:
\begin{verbatim}
class Point {
private:
  float x,y;
public:
  Point operator*(float factor) {
    return Point(factor*x,factor*y);
  };
};
\end{verbatim}
Can even redefine equals and parentheses.
\end{block}

See section~\ref{sec:overloadbracket} for redefining the parentheses
and square brackets.

\begin{exercise}
  Write a \n{Fraction} class, and define the arithmetic operators on it.
\end{exercise}

\Level 1 {Copy constructor}

Just like the default constructor which is defined if you don't define
an explicit constructor, there is an implicitly defined
\indextermsub{copy}{constructor}. This constructor is invoked whenever
you do an obvious copy:
\begin{verbatim}
my_object x,y; // regular or default constructor
x = y;         // copy constructor
\end{verbatim}
Usually the copy constructor that is implicitly defined does the right
thing: it copies all data members. (If you want to define your own copy
constructor, you need to know its prototype. We will not go into that.)

Example of the copy constructor in action:
%
\verbatimsnippet{classwithcopy}
\snippetwithoutput{classwithcopyuse}{object}{copyscalar}

\begin{slide}{Copy constructor}
  \label{sl:class-copy}
  \begin{multicols}{2}
    \begin{itemize}
    \item
      Several default copy constructors are defined
    \item They copy an object:
      \begin{itemize}
      \item simple data, including pointers
      \item included objects recursively.
      \end{itemize}
    \item You can redefine them as needed, for instance for deep copy.
    \end{itemize}
    \vfill\columnbreak
    \verbatimsnippet{classwithcopy}
  \end{multicols}
  \snippetwithoutput{classwithcopyuse}{object}{copyscalar}
\end{slide}

\begin{block}{Copying is recursive}
  \label{sl:class-copy-vector}
  Class with a vector:
  \verbatimsnippet{classwithcopyvector}
  Copying is recursive, so the copy has its own vector:
  \snippetwithoutput{classwithcopyvectoruse}{object}{copyvector}
\end{block}

\Level 1 {Destructor}
\label{sec:destructor}

Just there is a constructor routine to create an object, there is a
\indextermdef{destructor} to destroy the object.
As with the case of the default constructor, there is a default
destructor, which you can replace with your own.

A destructor can be useful if your object contains dynamically created
data: you want to use the destructor to dispose of that dynamic data
to prevent a \indextermbus{memory}{leak}. Another example is closing
files for which the \indextermbus{file}{handle} is stored in the object.

The destructor is typically called without you noticing it. For
instance, any statically created object is destroyed when the control
flow leaves its scope.

Example:
%
\snippetwithoutput{destructor}{object}{destructor}

\begin{slide}{Destructor}
  \label{sl:class-destruct}
  \begin{itemize}
  \item Every class \n{myclass} has a \emph{destructor} \n{~myclass}
    defined by default.
  \item The default destructor does nothing:
\begin{verbatim}
~myclass() {};
\end{verbatim}
\item A destructor is called when the object goes out of scope.\\
  Great way to prevent memory leaks: dynamic data can be released
  in the destructor. Also: closing files.
\end{itemize}
\end{slide}

\begin{slide}{Destructor example}
  \label{sl:class-destruct-ex}
  Destructor called implicitly:
\snippetwithoutput{destructor}{object}{destructor}
\end{slide}


\begin{exercise}
  \label{ex:destruct-trace}
  Write a class
\begin{verbatim}
class HasInt {
private:
  int mydata;
public:
  HasInt(int v) { /* initialize */ };
  ...
}
\end{verbatim}
used as
\begin{verbatim}
{ HasInt v(5);
  v.set(6);
  v.set(-2);
}
\end{verbatim}
which gives output
\begin{verbatim}
**** creating object with 5 ****
**** setting object to 6 ****
**** setting object to -2 ****
**** object destroyed after 2 updates ****
\end{verbatim}
\end{exercise}

\begin{block}{Destructors and exceptions}
  \label{sl:exceptobj}
  The destructor is called when you throw an exception:
  %
  \snippetwithoutput{exceptobj}{object}{exceptobj}
\end{block}

% {Relations between classes}
%\SetBaseLevel 1
\input inheritance
%\SetBaseLevel 0

\Level 0 {More topics about classes}

\begin{block}{`this'}
  \label{sl:class-this}
  Inside an object, a \indexterm{pointer} to the object is available
  as \indextermtt{this}:
\begin{verbatim}
class Myclass {
private:
  int myint;
public:
  Myclass(int myint) {
    this->myint = myint;
  };
};
\end{verbatim}
This is not often needed. Typical use case:
you need to call a function inside a method that needs the object as argument)
\begin{verbatim}
class someclass;
void somefunction(const someclass &c) {
  /* ... */ }
class someclass {
// method:
void somemethod() {
  somefunction(*this);
};
\end{verbatim}
\end{block}

\Level 1 {Static members}

\begin{block}{Static class members}
  \label{sl:static-member}
  A static member acts like shared between all objects.
\begin{verbatim}
class MyClass {
private:
  static object_count;
public:
  MyClass() { object_count++; };
}
\end{verbatim}
Initialization has to be done elsewhere:
\begin{verbatim}
MyClass::object_count = 0;
\end{verbatim}
\end{block}

\Level 0 {Review question}

\begin{exercise}
  What specifies the capabilities of an object?
\end{exercise}

\begin{exercise}
  What determines the state of an object?
\end{exercise}

\begin{exercise}
  How many constructors do you need to specify in an object definition?
  \begin{itemize}
  \item Zero
  \item Zero or more
  \item One
  \item One or more
  \end{itemize}
\end{exercise}

\begin{exercise}
  Describe various ways to initialize the members of an object.
\end{exercise}
