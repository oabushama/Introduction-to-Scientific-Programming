% -*- latex -*-
%%%%%%%%%%%%%%%%%%%%%%%%%%%%%%%%%%%%%%%%%%%%%%%%%%%%%%%%%%%%%%%%
%%%%
%%%% This TeX file is part of the course
%%%% Introduction to Scientific Programming in C++/Fortran2003
%%%% copyright 2017-9 Victor Eijkhout eijkhout@tacc.utexas.edu
%%%%
%%%% object.tex : we get down to OOP!
%%%%
%%%%%%%%%%%%%%%%%%%%%%%%%%%%%%%%%%%%%%%%%%%%%%%%%%%%%%%%%%%%%%%%

\Level 0 {What is an object?}
\label{sec:object}

You learned about \n{struct}s (chapter~\ref{ch:struct}) as a way of
abstracting from the elementary data types. The elements of a
structure were called its members.

You also saw that it is possible to write
functions that work on structures. We now combine these two elements
to give a new level of abstraction:

\begin{block}{Definition of object}
  \label{sl:object-def}
  An object is an entity that you can request to do certain
  things. These actions are the
  \emph{methods}\index{object!methods}\index{methods|see{object, methods}}
  and to make these possible the object probably stores
  data, the
  \emph{members}\index{object!members}\index{members|see{object, members}}.

  When designing an object, first ask yourself: `what functionality
  should this support'.

\end{block}

\begin{block}{Objects and classes}
  \label{label:class-def}
  An object is a particular instance of a \indextermdef{class} of
  similar objects, see you need a \indextermbus{class}{definition},
  after which you define objects of that class.
\end{block}

\begin{block}{Object functionality}
  \label{sl:object-functionality}
  Small illustration: vector objects.
  %
  \snippetwithoutput{functionality}{object}{functionality}
  %
  Note the `dot' notation; in a \lstinline{struct} we use it for the
  data members; in an object we (also) use it for methods.
\end{block}

\begin{exercise}
  \label{ex:pointdesign}
  Thought exercise:\\
  What data does the object need to store to do this?\\
  Is there more than one possibility?
\end{exercise}

\begin{comment}
  First of all, you can make an object look pretty much like a
  structure:
  %
  \snippetwithoutput{pointstruct}{geom}{pointstruct}

  Observations about the above code snippet:
  \begin{itemize}
  \item Again we have separate definition of the class and declaration
    of the objects. You define the class only once, after which you can
    make as many objects of that class as you want.
  \item There are data members. We will get to the \n{public} in a
    minute.
  \item You make an object of that class by using the class name as the
    datatype.
  \item The data members can be accessed with the period.
  \end{itemize}

  \begin{slide}{Classes look a bit like structures}
    \label{sl:class-struct}
    \snippetwithoutput{pointstruct}{geom}{pointstruct}

    Class definition versus object declaration.\\
    We'll get to that `\n{public}' in a minute.
  \end{slide}
\end{comment}

\Level 1 {Constructor}

Next we'll look at a syntax for creating class objects that is new. If
you create an object, you actually call a function that has the same
name as the class: the \indexterm{constructor}.

Usually you write your own constructor,
for instance to initialize data members. This fragment of the \n{class
  Vector} shows the constructor and the data members:
%
\verbatimsnippet{functionalityconstruct}

\begin{slide}{Constructor}
  \label{sl:class-construct}
  Use a \indexterm{constructor}: function with same name as the
  class.\\
  Typically used to initialize data members.
  
  \verbatimsnippet{functionalityconstruct}

  The synxtax \n{x(x)} copies the argument to the data member.
\end{slide}

\begin{block}{Constructor and initialization}
  \begin{itemize}
  \item Keyword \indextermtt{private} indicates that data is internal
  \item keyword \indextermtt{public} indicates that the constructor
    function can be used in the program.
  \item Initialization \n{x(x)} should be read as
    \begin{quotation}
      \hbox{\sl membername(argumentname)}
    \end{quotation}
    Yes, having \n{x} twice is a little confusing.
  \end{itemize}
\end{block}

\begin{comment}
  Above you had a declaration \n{Vector v} which was not just a
  declaration: it called the so-called \indextermsub{default}{constructor},
  which has no arguments, and does nothing.
\end{comment}

\Level 1 {Methods}

Methods are things you can ask your class objects to do. For instance,
in the \n{Vector} class, you could ask a vector what it's length is,
or you could ask it to scale its length by some number.

\begin{block}{Class methods}
  \label{sl:method-define}
  We used methods \n{length} and \n{scaleby}. These are defined inside the
  class:
  %
  \verbatimsnippet{functionalitymethod}
  %
  \begin{itemize}
  \item They look like ordinary functions,
  \item except that they can use the data members of the class, for
    instance~\n{x};
  \item Methods can only be used on an object:
    \verbatimsnippet{functionalitymethoduse}
  \end{itemize}
\end{block}

\Level 1 {Public and private}

You may have noticed the keywords \indextermtt{public} and
\indextermtt{private}. We made the data members private, and the
methods public. Thinking back to structures, you could say that their
data members were (by default) public. Why don't we do that here?

Objects are really supposed to be accessed through their
functionality.

Another reason for making data members inaccessible is that we can
change the internals of a class, without affecting the code that uses
the class. Consider the scenario where you have a \n{Point} class, and
you store the $x,y$ coordinates. Then you change you mind and decide
to store $r,\theta$ coordinates instead. If your program used the fact
that it could directly access the $x,y$ coordinates, you would be in
for a lot of rewriting. On the other hand if you have a function
\n{getx()}, your program does not need any rewriting, and you would
only change how that function was written.

\begin{block}{No direct access to internals!}
  \label{sl:repr-independent}
\begin{lstlisting}
class PositiveNumber { /* ... */ }
class Point {
private:
  // data members
public:
  Point( float x,float y ) { /* ... */ };
  Point( PositiveNumber r,float theta ) { /* ... */ };
  float get_x() { /* ... */ };
  float get_y() { /* ... */ };
  float get_r() { /* ... */ };
  float get_theta() { /* ... */ };
};
\end{lstlisting}
  The `get' functionality is independent of what data members the
  \n{Point} class has.
\end{block}

\begin{block}{Public versus private}
  \label{sl:interfaceimpl}
  \begin{itemize}
  \item Implementation: data members, keep \n{private},
  \item Interface: \n{public} functions to get/set data.
  \item Protect yourself against inadvertant changes of object data.
  \item Possible to change implementation without rewriting calling code.
  \end{itemize}
\end{block}

You should not write access functions lightly: you should first think
about what elements of your class should conceptually be inspectable
or changeable by the outside world.  Consider for example a class
where a certain relation holds between members. In that case only
changes are allowed that maintain that relation.

\begin{block}{Private access gone wrong}
  \label{sl:privatenogood}
  We make a class for points on the unit circle\\
  You don't want to be able to change just one of \n{x,y}!
  %
  \verbatimsnippet{unitpointdef}
  %
  In general: enforce predicates on the members.
\end{block}

\begin{comment}
  In the first example above, the data members of the \n{Vector} class were
  declared \n{public}, meaning that they are accessable from the calling
  (main) program. While this is initially convenient for coding, it is a bad idea
  in the long term. For a variety of reasons it is good practice to
  separate interface and implementation of a class.

  \begin{block}{Example of accessor functions}
    \label{sl:pointaccess}
    Getting and setting of members values is done through accessor functions:
    \begin{multicols}{2}
      \verbatimsnippet{pointprivate}
      \verbatimsnippet{pointprivateset}
      \verbatimsnippet{pointprivateclose}
      \verbatimsnippet{pointprivatedefine}
    \end{multicols}
    Usage:
    %
    \verbatimsnippet{pointprivatesetuse}
  \end{block}
\end{comment}

\Level 1 {Initialization}

\begin{block}{Member default values}
  \label{sl:class-defval}
  Class members can have default values, just like ordinary variables:
\begin{lstlisting}
class Point {
private:
  float x=3., y=.14;
private:
  // et cetera
}
\end{lstlisting}
  Each object will have its members initialized to these values.
\end{block}

\begin{block}{Member initializer lists}
  \label{sl:class-init}
  Other syntax for initialization:
  \verbatimsnippet{classpointinit}
\end{block}

\begin{block}{Member initialization in the constructor}
  \label{sl:class-set}
\begin{lstlisting}
class Vector {
private:
  double r,theta;
public:
  Vector( double x,double y ) {
    r = sqrt(x*x+y*y);
    theta = atan(y/x);
  }
\end{lstlisting}
\end{block}

\begin{block}{Advantage of initializer list}
  \label{sl:class-init-why}
  Allows for reuse of names:
  %
  \snippetwithoutput{classpointinitxy}{geom}{pointinitxy}
  %
  Also saves on constructer invocation if the member is an object.
\end{block}

\begin{block}{Initializer lists}
  \label{sl:class-inlist}
  \emph{Initializer lists}\index{initializer list} can be used as denotations.
  \verbatimsnippet{initcurlypoint}  
\end{block}

\Level 1 {Methods}

With the accessors, you have just seen a first example of a class
\indextermdef{method}: a function that is only defined for objects of
that class, and that have access to the private data of that object.
Let's now look at more meaningful methods. For instance, for the
\n{Vector} class you can define functions such as \n{length} and
\n{angle}. 
%
\snippetwithoutput{pointfunc}{geom}{pointfunc}

\begin{exercise}
  \label{ex:vectorclass-print}
  Add a \n{print} function to the \n{Vector} class.

  How many parameters does it need?
\end{exercise}

By making these functions public, and the data members
private, you define an \acf{API} for the class:
\begin{itemize}
\item You are defining operations for that class; they are the only
  way to access the data of the object.
\item The methods can use the data of the object, or alter it. All
  data members, even when declared \n{private}, are global to the methods.
\item  Data members declared \n{private} are not accessible from outside the
  object.
\end{itemize}

\begin{slide}{Functions on objects}
  \label{sl:obj-func}
  \snippetwithoutput{pointfunc}{geom}{pointfunc}
  %\verbatimsnippet{pointfunc}
  We call such internal functions `methods'.\\
  Data members, even \n{private}, are global to the methods.
\end{slide}

So far you have seen methods that use the data members of an object to
return some quantity. It is also possible to alter the members. 
For instance, you may want to scale a vector by some amount:
%
  \snippetwithoutput{pointscaleby}{geom}{pointscaleby}

\begin{slide}{Methods that alter the object}
  \label{sl:obj-func-on}
  \snippetwithoutput{pointscaleby}{geom}{pointscaleby}
\end{slide}

\begin{block}{Direct alteration of internals}
  \label{sl:obj-return-ref}
  Return a reference to a private member:
\begin{lstlisting}
class Vector {
private:
  double vx,vy;
public:
  double &x() { return vx; };
};
int main() {
  Vector v;
  v.x() = 3.1;
}
\end{lstlisting}
\end{block}

\begin{block}{Reference to internals}
  \label{sl:obj-return-const-ref}
  Returning a reference saves you on copying.\\
  Prevent unwanted changes by using a `const reference'.
\begin{lstlisting}
class Grid {
private:
  vector<Point> thepoints;
public:
  const vector<Point> &points() {
    return thepoints; };
};
int main() {
  Grid grid;
  cout << grid.points()[0];
  // grid.points()[0] = whatever ILLEGAL
}
\end{lstlisting}
\end{block}

The methods you have seen so far only returned elementary
datatypes. It is also possible to return an object, even from the same
class. For instance, instead of scaling the members of a vector object, you
could create a new object based on the scaled members:
%
\snippetwithoutput{pointscale}{geom}{pointscale}

\begin{slide}{Methods that create a new object}
  \label{sl:obj-return}
  \snippetwithoutput{pointscale}{geom}{pointscale}
\end{slide}

\Level 1 {Default constructor}

One of the more powerful ideas in C++ is that there can be more than
one constructor. You will often be faced with this whether you want or
not. The following code looks plausible:
%
\verbatimsnippet{pointdef2}
%
but it will give an error message during compilation. The reason is
that 
\begin{lstlisting}
Vector v2;
\end{lstlisting}
calls the default constructor. Now that you have defined your own
constructor, the default constructor no longer exists. So you need to
define it explicitly:
%
\verbatimsnippet{pointdef1}

\begin{slide}{Default constructor}
  \label{sl:obj-def-construct}
\small
  \verbatimsnippet{pointdef2}
  gives (g++; different for intel):
\begin{lstlisting}
pointdefault.cxx: In function 'int main()':
pointdefault.cxx:32:21: error: no matching function for call to
                'Vector::Vector()'
   Vector v1(1.,2.), v2;
\end{lstlisting}
The problem is with \n{v2}. How is it created? We need to define two constructors:
\verbatimsnippet{pointdef1}
\end{slide}

\begin{comment}
  \Level 1 {Accessors}

  It is a good idea to keep data members private, and use accessor
  functions.

  \begin{exercise}
    \label{ex:geom:twoconstruct}
    Write a \n{Point} class that has two constructors:
    \begin{lstlisting}
      class Point {
        private:
        // data members
        public:
        Point( float x,float y ) { /* ... */ };
        Point( float r,float theta ) { /* ... */ };
        float get_x() { /* ... */ };
        float get_y() { /* ... */ };
        float get_r() { /* ... */ };
        float get_theta() { /* ... */ };
      };
    \end{lstlisting}
    Use $r,\theta$ for the private variables, do not store $x,y$.
  \end{exercise}
\end{comment}

\Level 1 {Examples}

\begin{block}{Classes for abstact objects}
  \label{sl:intstream}
  Objects can model fairly abstract things:
  %
  \snippetwithoutput{integerstream}{object}{stream}
\end{block}

\begin{exercise}
  If you are doing the prime project (chapter~\ref{ch:prime}),
  now is a good time to do exercise~\ref{ex:prime:sequence}.
\end{exercise}

\Level 1 {Advanced topics}

\emph{The remainder of this section is advanced material. Make sure
  you have studied section~\ref{sec:class-ref}.}

\Level 2 {Accessor functions}

It is a good idea to make the data in an object private,
to control outside access to it.
\begin{itemize}
\item Sometimes this private data is auxiliary, and there is no reason
  for outside access.
\item Sometimes you do want outside access, but you want to precisely
  control by whom.
\end{itemize}

Accessor functions:
\begin{lstlisting}
class thing {
private:
 float x;
public:
 float get_x() { return x; };
 void set_x(float v) { x = v; }
};
\end{lstlisting}
This has advantages:
\begin{itemize}
\item You can print out any time you get/set the value; great for
  debugging:
\begin{lstlisting}
void set_x(float v) {
  cout << "setting: " << v << endl;
  x = v; }
\end{lstlisting}
\item You can catch specific values: if \n{x} is always supposed to be
  positive, print an error (throw an exception) if nonpositive.
\end{itemize}

Having two accessors can be a little clumsy. Is it possible to use the
same accessor for getting and setting?

\begin{block}{Setting members through accessor}
  \label{sl:setmember}
  %
  Use a single accessor for getting and setting:

  \snippetwithoutput{objaccessref}{object}{accessref}
  %
  The function \n{xvalue} returns a reference to the internal
  variable~\n{x}.
\end{block}

Of course you should only do this if you want the internal variable to
be directly changable!

\begin{exercise}
  This is a good time to do the exercises in section~\ref{ex:pointfunc}.
\end{exercise}

\Level 2 {Accessability}

\begin{block}{Access levels}
  \label{sl:private-etc}
  Methods and data can be 
  \begin{itemize}
  \item private, because they are only used internally;
  \item public, because they should be usable from outside a class
    object, for instance in the main program;
  \item protected, because they should be usable in derived classes (see
    section~\ref{sec:derive-method}).
  \end{itemize}
\end{block}

You can have multiple methods with the same name, as long as they can
be distinguished by their argument types. This is known as \indexterm{polymorphism}.

\Level 2 {Operator overloading}
\label{sec:operatordef}

Instead of writing 
\begin{lstlisting}
myobject.plus(anotherobject)
\end{lstlisting}
you can actually redefine the \n{+} operator so that
\begin{lstlisting}
myobject + anotherobject
\end{lstlisting}
is legal. This is known as \indextermbusdef{operator}{overloading}.

The syntax for this is
\begin{block}{Operator overloading}
  \label{sl:object-operator}
\begin{lstlisting}
<returntype> operator<op>( <argument> ) { <definition> }
\end{lstlisting}
For instance:
\begin{lstlisting}
class Point {
private:
  float x,y;
public:
  Point operator*(float factor) {
    return Point(factor*x,factor*y);
  };
};
\end{lstlisting}
Can even redefine equals and parentheses.
\end{block}

\begin{exercise}
  Write a \n{Fraction} class, and define the arithmetic operators on it.
\end{exercise}

\Level 2 {Functors}

A special case of operator overloading is
\index{operator!overloading!of parentheses}%
\emph{overloading the parentheses}. This makes an object look like a
function; we call this a \indextermdef{functor}.

Simple example:
%
\snippetwithoutput{printfunctor}{object}{functor}

\begin{exercise}
  \label{ex:functor2}
  Extend that class as follows: instead of printing the argument
  directly, it should print it multiplied by a scalar. That scalar
  should be set in the constructor. Make the following code work:
  %
  \answerwithoutput{functortworun}{object}{functor2}
  %
  (The \n{for_each} is part of \indextermtt{algorithm})
\end{exercise}


\Level 2 {Copy constructor}

Just like the default constructor which is defined if you don't define
an explicit constructor, there is an implicitly defined
\indextermsub{copy}{constructor}. This constructor is invoked whenever
you do an obvious copy:
\begin{lstlisting}
my_object x,y; // regular or default constructor
x = y;         // copy constructor
\end{lstlisting}
Usually the copy constructor that is implicitly defined does the right
thing: it copies all data members. (If you want to define your own copy
constructor, you need to know its prototype. We will not go into that.)

Example of the copy constructor in action:
%
\verbatimsnippet{classwithcopy}
\snippetwithoutput{classwithcopyuse}{object}{copyscalar}

\begin{slide}{Copy constructor}
  \label{sl:class-copy}
  \begin{multicols}{2}
    \begin{itemize}
    \item
      Several default copy constructors are defined
    \item They copy an object:
      \begin{itemize}
      \item simple data, including pointers
      \item included objects recursively.
      \end{itemize}
    \item You can redefine them as needed, for instance for deep copy.
    \end{itemize}
    \vfill\columnbreak
    \verbatimsnippet{classwithcopy}
  \end{multicols}
\end{slide}

\begin{slide}{Copy constructor in action}
  \label{sl:class-copy-out}
  \snippetwithoutput{classwithcopyuse}{object}{copyscalar}
\end{slide}

\begin{block}{Copying is recursive}
  \label{sl:class-copy-vector}
  Class with a vector:
  \verbatimsnippet{classwithcopyvector}
  Copying is recursive, so the copy has its own vector:
  \snippetwithoutput{classwithcopyvectoruse}{object}{copyvector}
\end{block}

\Level 2 {Destructor}
\label{sec:destructor}

Just as there is a constructor routine to create an object, there is a
\indextermdef{destructor} to destroy the object.
As with the case of the default constructor, there is a default
destructor, which you can replace with your own.

A destructor can be useful if your object contains dynamically created
data: you want to use the destructor to dispose of that dynamic data
to prevent a \indextermbus{memory}{leak}. Another example is closing
files for which the \indextermbus{file}{handle} is stored in the object.

The destructor is typically called without you noticing it. For
instance, any statically created object is destroyed when the control
flow leaves its scope.

Example:
%
\snippetwithoutput{destructor}{object}{destructor}

\begin{slide}{Destructor}
  \label{sl:class-destruct}
  \begin{itemize}
  \item Every class \n{myclass} has a \emph{destructor} \n{~myclass}
    defined by default.
  \item The default destructor does nothing:
\begin{lstlisting}
~myclass() {};
\end{lstlisting}
\item A destructor is called when the object goes out of scope.\\
  Great way to prevent memory leaks: dynamic data can be released
  in the destructor. Also: closing files.
\end{itemize}
\end{slide}

\begin{slide}{Destructor example}
  \label{sl:class-destruct-ex1}
  Just for tracing, constructor and destructor do \n{cout}:
  %
  \verbatimsnippet{destructor}
\end{slide}

\begin{slide}{Destructor example}
  \label{sl:class-destruct-ex2}
  Destructor called implicitly:
\snippetwithoutput{destructoruse}{object}{destructor}
\end{slide}

\begin{exercise}
  \label{ex:destruct-trace}
  Write a class
\begin{lstlisting}
class HasInt {
private:
  int mydata;
public:
  HasInt(int v) { /* initialize */ };
  ...
}
\end{lstlisting}
used as
%
\answerwithoutput{destructexcall}{object}{destructexercise}
\end{exercise}

\begin{block}{Destructors and exceptions}
  \label{sl:exceptobj}
  The destructor is called when you throw an exception:
  %
  \snippetwithoutput{exceptdestruct}{object}{exceptdestruct}
\end{block}

% {Relations between classes}
%\SetBaseLevel 1
\input inheritance
%\SetBaseLevel 0

\Level 2 {`this' pointer}

\begin{block}{`this' pointer to the current object}
  \label{sl:class-this}
  Inside an object, a \indexterm{pointer} to the object is available
  as \indextermtt{this}:
\begin{lstlisting}
class Myclass {
private:
  int myint;
public:
  Myclass(int myint) {
    this->myint = myint;
  };
};
\end{lstlisting}
\end{block}
\begin{block}{`this' use}
  \label{sl:class-this-fun}
This is not often needed. Typical use case:
you need to call a function inside a method that needs the object as argument)
\begin{lstlisting}
class someclass;
void somefunction(const someclass &c) {
  /* ... */ }
class someclass {
// method:
void somemethod() {
  somefunction(*this);
};
\end{lstlisting}
\end{block}

\Level 2 {Static members}

\begin{block}{Static class members}
  \label{sl:static-member}
  A static member acts like shared between all objects.
\begin{lstlisting}
class MyClass {
private:
  static int object_count;
public:
  MyClass() { object_count++; };
}
\end{lstlisting}
Initialization has to be done elsewhere:
\begin{lstlisting}
MyClass::object_count = 0;
\end{lstlisting}
\end{block}

\Level 0 {Review question}

\begin{exercise}
  \label{ex:class-review1}
  Fill in the missing term
  \begin{itemize}
  \item The functionality of a class is determined by its\ldots
  \item The state of an object is determined by its\ldots
  \end{itemize}

  How many constructors do you need to specify in a class definition?
  \begin{itemize}
  \item Zero
  \item Zero or more
  \item One
  \item One or more
  \end{itemize}
\end{exercise}

\begin{exercise}
  \label{ex:class-review2}
  Describe various ways to initialize the members of an object.
\end{exercise}
