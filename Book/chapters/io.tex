
\begin{exercise}
  \label{ex:fixedpout}
  Use integer output to print fixed point numbers aligned on the
  decimal:
\begin{verbatim}
  1.345
 23.789
456.1234
\end{verbatim}
  Use four spaces for both the integer and fractional part.
\end{exercise}

\begin{block}{Hexadecimal}
  \label{sl:io-hex}
  Hex output is useful for pointers (chapter~\ref{ch:pointer}):
  %
  \verbatimsnippet{coutpoint}
  %
  Back to decimal:
\begin{verbatim}
cout << hex << i << dec << j;
\end{verbatim}
\end{block}

\Level 0 {Floating point output}

\begin{block}{Floating point precision}
  \label{sl:io-float}
  Use \n{setprecision} to set the number of digits before and after
  decimal point:
\begin{verbatim}
  x = 1.234567;
  for (int i=0; i<10; i++) {
    cout << setprecision(4) << x << endl;
    x *= 10;
  }
\end{verbatim}
\end{block}

\begin{block}{Output}
  \label{sl:io-float-out}
\begin{verbatim}
1.235
12.35
123.5
1235
1.235e+04
1.235e+05
1.235e+06
1.235e+07
1.235e+08
1.235e+09
\end{verbatim}
(Notice the rounding)
\end{block}

\begin{block}{Fixed point precision}
  \label{sl:io-fix}
  Fixed precision applies to fractional part:
\begin{verbatim}
cout << "Fixed precision applies to fractional part:" << endl;
x = 1.234567;
cout << fixed;
for (int i=0; i<10; i++) {
  cout << setprecision(4) << x << endl;
  x *= 10;
}
\end{verbatim}
\end{block}

\begin{block}{Output}
  \label{sl:io-fix-out}
\begin{verbatim}
1.2346
12.3457
123.4567
1234.5670
12345.6700
123456.7000
1234567.0000
12345670.0000
123456700.0000
1234567000.0000
\end{verbatim}
\end{block}

\begin{block}{Aligned fixed point output}
  \label{sl:io-align}
  Combine width and precision:
\begin{verbatim}
x = 1.234567;
cout << fixed;
for (int i=0; i<10; i++) {
  cout << setw(10) << setprecision(4) << x << endl;
  x *= 10;
}
\end{verbatim}
\end{block}

\begin{block}{Output}
  \label{sl:io-align-out}
\begin{verbatim}
    1.2346
   12.3457
  123.4567
 1234.5670
12345.6700
123456.7000
1234567.0000
12345670.0000
123456700.0000
1234567000.0000
\end{verbatim}
\end{block}

\begin{block}{Scientific notation}
  \label{sl:io-sci}
\begin{verbatim}
cout << "Combine width and precision:" << endl;
x = 1.234567;
cout << scientific;
for (int i=0; i<10; i++) {
  cout << setw(10) << setprecision(4) << x << endl;
  x *= 10;
}
\end{verbatim}
\end{block}

\begin{block}{Output}
  \label{sl:io-sci-out}
\begin{verbatim}
Combine width and precision:
1.2346e+00
1.2346e+01
1.2346e+02
1.2346e+03
1.2346e+04
1.2346e+05
1.2346e+06
1.2346e+07
1.2346e+08
1.2346e+09
\end{verbatim}
\end{block}

\Level 0 {Saving and restoring settings}

\begin{verbatim}
ios::fmtflags old_settings = cout.flags();
\end{verbatim}

\begin{verbatim}
cout.flags(old_settings);
\end{verbatim}

\begin{verbatim}
int old_precision = cout.precision();
\end{verbatim}

\begin{verbatim}
cout.precision(old_precision);
\end{verbatim}

\Level 0 {File output}

\begin{block}{Text output to file}
  \label{sl:io-file}
Streams are general: work the same for console out and file out.
\begin{verbatim}
#include <fstream>
\end{verbatim}
Use:
\verbatimsnippet{fio}
\end{block}

\begin{block}{Binary output}
  \label{sl:io-bin}
  \verbatimsnippet{fiobin}
\begin{verbatim}
  
\end{verbatim}
\end{block}

\Level 1 {Output your own classes}
\label{sec:lessless}

You have used statements like:
\begin{verbatim}
cout << ``My value is: `` << myvalue << endl;
\end{verbatim}
How does this work? The `double less' is an operator with a left
operand that is a stream, and a right operand for which output is
defined; the result of this operator is again a stream. Recursively,
this means you can chain any number of applications of~\verb+<<+
together.

If you want to output a class that you wrote yourself, you have to
define how the \n{<<} operator deals with your class.

\verbatimsnippet{classostream}

\Level 0 {Input}

\begin{block}{Better terminal input}
  \label{sl:getline}
  It is better to use \indextermtt{getline}. This returns a string,
  rather than a value, so you need to convert it with the following bit
  of magic:
  %
  \verbatimsnippet{readin}

  You can not use \n{cin} and \n{getline} in the same program.

  More info:
  \url{http://www.cplusplus.com/forum/articles/6046/}.

\end{block}

\Level 1 {Input streams}

Test, mostly for file streams: \indextermtt{is_eof} \indextermtt{is_open}

