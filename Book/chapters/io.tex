% -*- latex -*-
%%%%%%%%%%%%%%%%%%%%%%%%%%%%%%%%%%%%%%%%%%%%%%%%%%%%%%%%%%%%%%%%
%%%%
%%%% This TeX file is part of the course
%%%% Introduction to Scientific Programming in C++/Fortran2003
%%%% copyright 2017 Victor Eijkhout eijkhout@tacc.utexas.edu
%%%%
%%%% io.tex : input and output
%%%%
%%%%%%%%%%%%%%%%%%%%%%%%%%%%%%%%%%%%%%%%%%%%%%%%%%%%%%%%%%%%%%%%

\Level 0 {Formatted output}

\begin{block}{Formatted output}
  \label{sl:cformat}
  \begin{itemize}
  \item \n{cout} uses default formatting
  \item Possible: pad a number, use limited precision, format as hex/base2, etc
  \item 
    Many of these output modifiers need
  \end{itemize}
\begin{verbatim}
#include <iomanip>
\end{verbatim}
\end{block}

\begin{slide}{Default unformatted output}
  \label{sl:unformat}
  \snippetwithoutput{cunformat}{io}{cunformat}
\end{slide}

Normally, output of numbers takes up precisely the space that it needs:
\snippetwithoutput{cunformat}{io}{cunformat}

\begin{block}{Reserve space}
  \label{sl:io-setw}
  You can specify the number of positions, and the output is right
  aligned in that space by default:
  \snippetwithoutput{formatwidth6}{io}{width}
\end{block}

\begin{block}{Padding character}
  \label{sl:io-fill}
  Normally, padding is done with spaces, but you can specify other characters:
  \snippetwithoutput{formatpad}{io}{formatpad}
Note: single quotes denote characters, double quotes denote strings.
\end{block}

\begin{block}{Left alignment}
  \label{sl:io-left}
  Instead of right alignment you can do left:
  \snippetwithoutput{formatleft}{io}{formatleft}
\end{block}

\begin{block}{Number base}
  \label{sl:io-base}
  Finally, you can print in different number bases than~10:
  \snippetwithoutput{format16}{io}{format16}
\end{block}

\begin{exercise}
  \label{ex:leadzero}
  Make the above output more nicely formatted:
\begin{verbatim}
00 01 02 03 04 05 06 07 08 09 0a 0b 0c 0d 0e 0f 
10 11 12 13 14 15 16 17 18 19 1a 1b 1c 1d 1e 1f 
20 21 22 23 24 25 26 27 28 29 2a 2b 2c 2d 2e 2f 
etc
\end{verbatim}
\end{exercise}

\begin{exercise}
  \label{ex:fixedpout}
  Use integer output to print fixed point numbers aligned on the
  decimal:
\begin{verbatim}
  1.345
 23.789
456.1234
\end{verbatim}
  Use four spaces for both the integer and fractional part.
\end{exercise}

\begin{block}{Hexadecimal}
  \label{sl:io-hex}
  Hex output is useful for pointers (chapter~\ref{ch:pointer}):
  %
  \verbatimsnippet{coutpoint}
  %
  Back to decimal:
\begin{verbatim}
cout << hex << i << dec << j;
\end{verbatim}
\end{block}

\Level 0 {Floating point output}

\begin{block}{Floating point precision}
  \label{sl:io-float}
  Use \n{setprecision} to set the number of digits before and after
  decimal point:
  %
  \snippetwithoutput{formatfloat}{io}{formatfloat}
  %
  (Notice the rounding)
\end{block}

\begin{block}{Fixed point precision}
  \label{sl:io-fix}
  Fixed precision applies to fractional part:
\begin{verbatim}
cout << "Fixed precision applies to fractional part:" << endl;
x = 1.234567;
cout << fixed;
for (int i=0; i<10; i++) {
  cout << setprecision(4) << x << endl;
  x *= 10;
}
\end{verbatim}
\end{block}

\begin{block}{Output}
  \label{sl:io-fix-out}
\begin{verbatim}
1.2346
12.3457
123.4567
1234.5670
12345.6700
123456.7000
1234567.0000
12345670.0000
123456700.0000
1234567000.0000
\end{verbatim}
\end{block}

\begin{block}{Aligned fixed point output}
  \label{sl:io-align}
  Combine width and precision:
\begin{verbatim}
x = 1.234567;
cout << fixed;
for (int i=0; i<10; i++) {
  cout << setw(10) << setprecision(4) << x << endl;
  x *= 10;
}
\end{verbatim}
\end{block}

\begin{block}{Output}
  \label{sl:io-align-out}
\begin{verbatim}
    1.2346
   12.3457
  123.4567
 1234.5670
12345.6700
123456.7000
1234567.0000
12345670.0000
123456700.0000
1234567000.0000
\end{verbatim}
\end{block}

\begin{block}{Scientific notation}
  \label{sl:io-sci}
\begin{verbatim}
cout << "Combine width and precision:" << endl;
x = 1.234567;
cout << scientific;
for (int i=0; i<10; i++) {
  cout << setw(10) << setprecision(4) << x << endl;
  x *= 10;
}
\end{verbatim}
\end{block}

\begin{block}{Output}
  \label{sl:io-sci-out}
\begin{verbatim}
Combine width and precision:
1.2346e+00
1.2346e+01
1.2346e+02
1.2346e+03
1.2346e+04
1.2346e+05
1.2346e+06
1.2346e+07
1.2346e+08
1.2346e+09
\end{verbatim}
\end{block}

\Level 0 {Saving and restoring settings}

\begin{verbatim}
ios::fmtflags old_settings = cout.flags();
\end{verbatim}

\begin{verbatim}
cout.flags(old_settings);
\end{verbatim}

\begin{verbatim}
int old_precision = cout.precision();
\end{verbatim}

\begin{verbatim}
cout.precision(old_precision);
\end{verbatim}

\Level 0 {File output}

\begin{block}{Text output to file}
  \label{sl:io-file}
Streams are general: work the same for console out and file out.
\begin{verbatim}
#include <fstream>
\end{verbatim}
Use:
\verbatimsnippet{fio}
\end{block}

\begin{block}{Binary output}
  \label{sl:io-bin}
  \verbatimsnippet{fiobin}
\begin{verbatim}
  
\end{verbatim}
\end{block}

\Level 1 {Output your own classes}
\label{sec:lessless}

You have used statements like:
\begin{verbatim}
cout << ``My value is: `` << myvalue << endl;
\end{verbatim}
How does this work? The `double less' is an operator with a left
operand that is a stream, and a right operand for which output is
defined; the result of this operator is again a stream. Recursively,
this means you can chain any number of applications of~\verb+<<+
together.

If you want to output a class that you wrote yourself, you have to
define how the \n{<<} operator deals with your class.

\verbatimsnippet{classostream}

\Level 0 {Input}

\begin{block}{Better terminal input}
  \label{sl:getline}
  It is better to use \indextermtt{getline}. This returns a string,
  rather than a value, so you need to convert it with the following bit
  of magic:
  %
  \verbatimsnippet{readin}

  You can not use \n{cin} and \n{getline} in the same program.

  More info:
  \url{http://www.cplusplus.com/forum/articles/6046/}.

\end{block}

\Level 1 {Input streams}

Test, mostly for file streams: \indextermtt{is_eof} \indextermtt{is_open}

