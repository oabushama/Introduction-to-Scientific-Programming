% -*- latex -*-
%%%%%%%%%%%%%%%%%%%%%%%%%%%%%%%%%%%%%%%%%%%%%%%%%%%%%%%%%%%%%%%%
%%%%
%%%% This TeX file is part of the course
%%%% Introduction to Scientific Programming in C++/Fortran2003
%%%% copyright 2017-9 Victor Eijkhout eijkhout@tacc.utexas.edu
%%%%
%%%% parampassing.tex : section on parameter passing
%%%%
%%%%%%%%%%%%%%%%%%%%%%%%%%%%%%%%%%%%%%%%%%%%%%%%%%%%%%%%%%%%%%%%

C++~functions resemble mathematical functions: you have seen that a
function can have an input and an output. In fact, they can have
multiple inputs, separated by commas, but they have only one
output\footnote{Unless you use more sophisticated mechanisms.}.
\[ a = f(x,y,i,j) \]

We start by studying functions that look like these mathematical
functions. They involve a \indextermbus{parameter}{passing} mechanism
called
\emph{passing by value}\index{parameter!passing by value}.
%
Later we will then look at
\emph{passing by reference}\index{parameter!passing by reference}.

\Level 1 {Pass by value}
\index{pass by value|see{parameter, passing by value}}

The following style of programming is very much inspired by
mathematical functions, and is known as \indextermdef{functional
  programming}\footnote {There is more to functional programming. For
  instance, strictly speaking your whole program needs to be based on
  function calling; there is no other code than function definitions
  and calls.}.
\begin{itemize}
\item A function has one result, which is returned through a return
  statement. The function call then looks like
\begin{lstlisting}
y = f(x1,x2,x3);
\end{lstlisting}
\item The definition of the C++ parameter passing mechanism says that
  input arguments are copied to the function, meaning that they don't
  change in the calling program:
  %
  \snippetwithoutput{passvalue}{func}{passvalue}

\end{itemize}

We say that the input argument is
\emph{passed by value}\index{parameter!pass by value|textbf}:
its value is copied into the
function.  In this example, the function parameter \n{x} acts as a
local variable in the function, and it is initialized with a copy of
the value of \n{number} in the main program.

Later we will worry about the cost of this copying.

\begin{slide}{Mathematical type function}
  \label{sl:func-functional}
  Pretty good design:
  \begin{itemize}
  \item pass data into a function,
  \item return result through \n{return} statement.
  \item Parameters are copied into the function. (Cost of copying?)
  \item \indexterm{pass by value}
  \item `functional programming'
  \end{itemize}
\end{slide}

\begin{slide}{Pass by value example}
  \label{sl:func-functional-ex}
  Note that the function alters its parameter \n{x}:
  %
  \snippetwithoutput{passvalue}{func}{passvalue}
  %
  but the argument in the main program is not affected.
\end{slide}

Passing a variable to a routine passes the value; in the routine, the
variable is local. So, in this example
the value of the argument is not changed:

\snippetwithoutput{localparm}{func}{localparm}

\begin{exercise}
If you are doing the prime project (chapter~\ref{ch:prime}) you can
now do exercise~\ref{ex:prime:func}.
\end{exercise}

\begin{block}{Background Square roots through Newton}
  \label{sl:newton-root}
  Early computers had no hardware for computing a square
  root. Instead, they used \indexterm{Newton's method}. Suppose you
  have a value~$y$ and you want want to compute
  $x=\sqrt{y}$. This is equivalent to finding the zero of
  \[ f(x) = x^2-y \] where $y$~is fixed. To indicate this dependence
  on~$y$, we will write~$f_y(x)$. Newton's method then finds the zero by
  evaluating
  \[ x_{\mathrm{next}}=x-f_y(x)/f_y'(x) \]
  until the guess is accurate enough, that is, until $f_y(x)\approx0$.
\end{block}

\begin{exercise}
  \label{ex:newton-root}
  \begin{itemize}
  \item Write functions \n{f(x,y)} and \n{deriv(x,y)}, that compute
    $f_y(x)$ and $f_y'(x)$ for the definition of $f_y$ above.
  \item Read a value $y$ and iterate until $|\n{f(x,y)}|<10^{-5}$. Print~$x$.
  \item Second part: write a function \n{newton_root} that computes~$\sqrt{y}$.
  \end{itemize}
\end{exercise}

\Level 1 {Pass by reference}
\label{sec:pass-by-ref}
\index{pass by reference|see{parameter, passing by reference}}
  
Having only one output is a limitation on functions. Therefore there
is a mechanism for altering the input parameters and returning
(possibly multiple) results that way. You do this by not copying
values into the function parameters, but by turning the function
parameters into aliases of the variables at the place where the
function is called.

We need the concept of a \indextermdef{reference}:
\begin{block}{Reference}
  \label{sl:cpp-reference}
  A reference is indicated with an ampersand in its definition, and it
  acts as an alias of the thing it references.
  %
  \snippetwithoutput{refint}{basic}{ref}
  %
  (You will not use references often this way.)
\end{block}

You can make a function parameter into a reference of a variable in
the main program:

\begin{block}{Parameter passing by reference}
  \label{sl:pass-by-ref}
The function parameter \n{n} becomes a reference to the variable \n{i}
in the main program:
\begin{lstlisting}
void f(int &n) {
  n = /* some expression */ ;
};
int main() {
  int i;
  f(i);
  // i now has the value that was set in the function
}
\end{lstlisting}
\end{block}

Using the ampersand, the parameter is
\emph{passed by reference}\index{parameter!passing by reference|textbf}:
instead of copying the value, the function receives a \emph{reference}:
information where the variable is stored in memory.

\begin{remark}
  The \emph{pass by reference mechanism in C}%
  \index{parameter!passing by reference! in C}
  was different and should not be used in~C++. In fact it was not a
  true pass by reference, but passing an address by value.
\end{remark}

We also the following terminology for function parameters:
\begin{itemize}
\item \emph{input}\index{parameter!input|textbf} parameters: passed by
  value, so that it only functions as input to the function, and no
  result is output through this parameter;
\item \emph{output}\index{parameter!output|textbf} parameters: passed
  by reference so that they return an `output' value to the program.
\item \emph{throughput}\index{parameter!throughput|textbf} parameters:
  these are passed by reference, and they have an initial value when
  the function is called. In C++, unlike Fortran, there is no real
  separate syntax for these.
\end{itemize}

\begin{slide}{Results other than through return}
  \label{sl:func-err-return}
  Also good design:
  \begin{itemize}
  \item Return no function result,
  \item or return \indexterm{return status} (0~is success, nonzero various
    informative statuses), and
  \item return other information by changing the parameters.
  \item \emph{pass by reference}
  \item Parameters are sometimes classified `input', `output', `throughput'.
  \end{itemize}
\end{slide}

\begin{block}{Pass by reference example 1}
  \label{sl:pass-reference1}
  \snippetwithoutput{setbyref}{basic}{setbyref}
  Compare the difference with leaving out the reference.
\end{block}

As an example, consider a function that tries to read a value from a
file. With anything file-related, you always have to worry about the
case of the file not existing and such. So our function returns:
\begin{itemize}
\item a boolean value to indicate whether the read succeeded, and
\item the actual value if the read succeeded.
\end{itemize}
The following is a common idiom, where the success value is returned
through the \n{return} statement, and the value through a parameter.

\begin{block}{Pass by reference example 2}
  \label{sl:pass-reference2}
\begin{lstlisting}
bool can_read_value( int &value ) {
  // this uses functions defined elsewhere
  int file_status = try_open_file();
  if (file_status==0) 
    value = read_value_from_file();
  return file_status==0;
}

int main() {
  int n;
  if (!can_read_value(n)) {
    // if you can't read the value, set a default
    n = 10;
  }
  ..... do something with 'n' ....
\end{lstlisting}
\end{block}

\begin{exercise}
  \label{ex:swap}
  Write a \n{void} function \n{swapij} of two parameters that
  exchanges the input values:
\begin{lstlisting}
int i=2,j=3;
swapij(i,j);
// now i==3 and j==2
\end{lstlisting}
\end{exercise}

\begin{exercise}
  \label{ex:div-remain}
  Write a divisibility function that takes a number and a divisor, and gives:
  \begin{itemize}
  \item a \n{bool} return result indicating that the number is
    divisible, and
  \item a remainder as output parameter.
  \end{itemize}

{\small
\begin{lstlisting}
int number,divisor,remainder;
// read in the number and divisor
if ( is_divisible(number,divisor,remainder) )
  cout << number << " is divisible by " << divisor << endl;
else
  cout << number << "/" << divisor <<
      " has remainder " << remainder << endl;
\end{lstlisting}
}
\end{exercise}

\begin{exercise}
  If you are doing the geometry project, you should now do the exercises
  in section~\ref{sec:geom-basic}.
\end{exercise}

\begin{block}{Const parameters}
  \label{sl:pass-constref}
  You can prevent local changes to the function parameter:

  \verbatimsnippet{readonlyparm}

  This is mostly to protect you against yourself.\\
  Const ref has no overhead, no danger of changes:

  \verbatimsnippet{constrefarray}
\end{block}

\begin{block}{Parameter passing summary}
  \label{sl:pass-summary}
  \begin{itemize}
  \item Standard mechanism: call by value, copying.
  \item Using \lstinline{Type &var}:
    \begin{itemize}
    \item call by reference,\item no copy, \item data in calling
      environment can be altered.
    \end{itemize}
  \item Using \lstinline{const Type &var}: 
    \begin{itemize}
    \item const-reference, \item by reference so no copy, \item but
      data in calling environment can not be changed.
    \end{itemize}
  \end{itemize}
\end{block}
