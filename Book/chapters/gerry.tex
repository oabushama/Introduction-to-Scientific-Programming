%% -*- latex -*-
%%%%%%%%%%%%%%%%%%%%%%%%%%%%%%%%%%%%%%%%%%%%%%%%%%%%%%%%%%%%%%%%
%%%%
%%%% This TeX file is part of the course
%%%% Introduction to Scientific Programming in C++/Fortran2003
%%%% copyright 2018 Victor Eijkhout eijkhout@tacc.utexas.edu
%%%%
%%%% gerry.tex : exercises about redistricting
%%%%
%%%%%%%%%%%%%%%%%%%%%%%%%%%%%%%%%%%%%%%%%%%%%%%%%%%%%%%%%%%%%%%%

In this project you can explore `redistricting', the strategic drawing
of districts to give a minority population a majority.

\Level 0 {Basic concepts}

We are dealing with the following concepts:
\begin{itemize}
\item A state is divided into census districts, which are given. Based
  on circumstantial factors (income, ethnicity, median age) one can
  usually make a good guess as to the overall voting in such a
  district. Next,
\item there is a predetermined number of congressional districts, that
  consist of census districts. A~congressional district is not a
  random collection: the census districts have to be contiguous.
\end{itemize}
Within these limits there is considerable freedom: by shifting the
boundaries of the (congressional) districts it is possible to give a
population that is in the minority a majority of districts.

As a simplifying assumption for this project we will assume a
one-dimensional state. This is enough to bring out the essence of the problem:
\begin{quotation}
  Consider a state of five voters, and we designate their votes as
  \n{AAABB}. Assigning them to three (contiguous) districts can be
  done as \n{AAA|B|B}, which has one `A'~district and two
  `B'~districts.
\end{quotation}

\Level 0 {Basic functions}

\Level 1 {Voters}

We dispense with census districts, expressing everything in terms of
voters instead.

\begin{exercise}
  Implement a \n{Voter} class. You could for instance let $\pm1$ stand
  for \n{A/B}, and 0 for undecided.
  %
  \snippetwithoutput{voters}{gerry}{voters}
  \verbatimsnippet{voterneg}
  \verbatimsnippet{voterwrong}
\end{exercise}

\Level 1 {Populations}

\begin{exercise}
  Implement a \n{District} class that models a group of voters.
  \begin{itemize}
    \item You probably want to create a district out of a single
      voter, or a vector of them
      %
      \snippetwithoutput{district}{gerry}{district}
    \item Write methods \n{majority} to give the exact majority or
      minority, and \n{lean} that evaluates whether the district
      overall counts as A~part or B~party.
  \item Write a \n{sub} method to creates subsets.
  \item For debugging and reporting it may be a good idea to have a method
\begin{verbatim}
string print();
\end{verbatim}
  \end{itemize}
\end{exercise}

\begin{exercise}
  Implement a \n{Population} class that will initially model a whole state.
  %
  \snippetwithoutput{populationexample}{gerry}{populationexample}

  In addition to an explicit creation, also write a constructor that
  specifies how many people and what the majority is:
  %
  \verbatimsnippet{randompopulation}
  %
  Use a random number generator to achieve precisely the indicated majority.
\end{exercise}

\Level 1 {Districting}

The next level of complication is to have a set of districts.
Since we
will be creating this incrementally, we need some methods for
extending it.

\begin{exercise}
  Write a class \n{Districting} that stores a \n{vector} of
  \n{District} objects. Write \n{size} and \n{lean} methods:
  %
  \snippetwithoutput{gerryempty}{gerry}{gerryempty}
\end{exercise}

\begin{exercise}
  Write methods to extend a \n{Districting}:
  %
  \verbatimsnippet{gerryextend}
\end{exercise}

\Level 0 {Strategy}

Now we need a method for districting a population:
\begin{verbatim}
Districting Population::minority_rules( int ndistricts );
\end{verbatim}
Rather than generating all possible partitions of the population, we
take an incremental approach (this is related to the solution strategy
called \indextermsub{dynamic}{programming}):
\begin{itemize}
\item The basic question is to divide a population optimally over $n$
  districts;
\item We do this recursively by first solving a division of a subpopulation over $n-1$
  districts,
  \item and extending that with the remaining population as one district.
\end{itemize}
Schematically:
\begin{itemize}
\item For all $p=0,\ldots n-1$ considering splitting the state into
  $0,\ldots,p-1$ and $p,\ldots,n-1$.
\item Use the best districting of the first group, and make the last
  group into a single district.
\item Keep the districting that gives the strongest minority rule,
  over all values of~$p$.
\end{itemize}

You can now realize the above simple example:
\begin{verbatim}
AAABB => AAA|B|B
\end{verbatim}

\begin{exercise}
  Implement the above scheme.
  %
  \snippetwithoutput{district5}{gerry}{district5}

  Note: the range for $p$ given above is not quite correct: for instance,
  the initial part of the population needs to be big enough to
  accomodate $n-1$ voters.
\end{exercise}

\begin{exercise}
  Test multiple population sizes; how much majority can you give
  party~B while still giving party~A a majority.
\end{exercise}

\Level 1 {Efficiency}

If you think about the algorithm you just implemented, you may notice
that the districtings of the initial parts get recomputed quite a bit.

\begin{exercise}
  Improve your implementation by storing and reusing results for the
  initial sub-populations.
\end{exercise}

\Level 0 {Extensions}

The project so far has several simplifying assumptions.
\begin{itemize}
\item Congressional districts need to be approximately the same
  size. Can you put a limit on the ratio between sizes? Can the
  minority still gain a majority?
\end{itemize}

\begin{exercise}
  The biggest assumption is of course that we considered a
  one-dimensional state. With two dimensions you have more degrees of
  freedom of shaping the districts. Implement a two-dimensional
  scheme; use a completely square state, where the census districts
  form a regular grid.
\end{exercise}
