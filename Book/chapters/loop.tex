
Another mechanism to alter the control flow in a loop is the
\indextermtt{continue} statement. If this is encountered, execution
skips to the start of the next iteration.

\begin{block}{Skip iteration}
  \label{sl:for-cont}
\begin{verbatim}
for (int var=low; var<N; var++) {
  statement;
  if (some_test) {
    statement;
    statement;
  }
}
\end{verbatim}
Alternative:
\begin{verbatim}
for (int var=low; var<N; var++) {
  statement;
  if (!some_test) continue;
  statement;
  statement;
}
\end{verbatim}
\end{block}

\begin{block}{While loop}
  \label{sl:while}
  The other possibility is a \indextermttdef{while} loop, which repeats until a
  condition is met.

  Syntax:
\begin{verbatim}
while ( condition ) {
  statements;
}
\end{verbatim}
or
\begin{verbatim}
do {
  statements;
} while ( condition );
\end{verbatim}
The while loop does not have a counter or an update statement; if you
need those, you have to create them yourself.
\end{block}

The two while loop variants can be described as `pre-test' and
`post-test'. The choice between them entirely depends on context. Here
is an example in which the second syntax is more appropriate.

\begin{block}{While syntax 1}
  \label{sl:while2}
  \verbatimsnippet{whiledo}

  Problem: code duplication.
\end{block}

\begin{block}{While syntax 2}
  \label{sl:while3}
  \verbatimsnippet{dowhile}

  More elegant.
\end{block}

\begin{exercise}
  \label{ex:interest}
  One bank account has 100 dollars and earns a 5~percent per year interest
  rate. Another account has 200 dollars but earns only 2~percent per
  year. In both cases the interest is deposited into the account.
  
  After how many years will the amount of money in the first account
  be more than in the second?
\end{exercise}

\Level 0 {Exercises}

\begin{exercise}
  \label{ex:pythagoras}
  Find all triples of integers $u,v,w$ under 100 such that
  $u^2+v^2=w^2$. Make sure you omit duplicates of solutions you have
  already found.
\end{exercise}

\begin{exercise}
  \label{ex:collatz}
  The integer sequence
  \[ u_{n+1} = 
  \begin{cases}
    u_n/2&\hbox{if $u_n$ is even}\\
    3u_n+1&\hbox{if $u_n$ is odd}\\
  \end{cases}
  \]
  leads to the Collatz conjecture: no matter the starting guess~$u_1$,
  the sequence $n\mapsto u_n$ will always terminate at~1.

  { \small
  \[ 5\rightarrow 16\rightarrow 8\rightarrow 4\rightarrow 2\rightarrow 1\]
  \[ 7\rightarrow 22\rightarrow 11\rightarrow 34\rightarrow
  17\rightarrow 52\rightarrow 26\rightarrow 13\rightarrow
  40\rightarrow 20\rightarrow 10\rightarrow 5\cdots \]
  }

  Try all starting values $u_1=1,\ldots,1000$ to find the values that
  lead to the longest sequence: every time you find a sequence that is
  longer than the previous maximum, print out the starting number.
\end{exercise}

\begin{exercise}
  Large integers are often printed with a comma (US~usage) or a period
  (European usage) between all triples of digits. Write a program that
  accepts an integer such as $2542981$ from the user, and prints it as
  \n{2,542,981}.
\end{exercise}

\begin{exercise}
  \label{ex:rootfind}
  \textbf{Root finding.}
  %
  For many functions~$f$, finding their zeros, that is, the values~$x$
  for which~$f(x)=0$, can not be done analytically. You then have to
  resort to numerical \indexterm{root finding} schemes. In this
  exercise you will implement a simple scheme.

  Suppose $x_-,x_+$ are such that 
  \[ x_-<x_+,\qquad f(x_-)\cdot f(x_+)<0,\]
  that is, the function values are of opposite sign. Then
  there is a zero in the interval~$(x_-,x_+)$. Try to find this zero
  by looking at the halfway point, and based on the function value
  there, considering the left or right interval.
  \begin{itemize}
  \item How do you find $x_-,x_+$? This is tricky in general; if you
    can find them in the interval~$[-10^6,+10^6]$, halt the program.
  \item Finding the zero exactly may also be tricky or maybe
    impossible. Stop your program if $|x_--x_+|<10^{-10}$.
  \end{itemize}
\end{exercise}

\Level 1 {Further practice}

The website
\url{http://www.codeforwin.in/2015/06/for-do-while-loop-programming-exercises.html}
