% -*- latex -*-
%%%%%%%%%%%%%%%%%%%%%%%%%%%%%%%%%%%%%%%%%%%%%%%%%%%%%%%%%%%%%%%%
%%%%
%%%% This TeX file is part of the course
%%%% Introduction to Scientific Programming in C++/Fortran2003
%%%% copyright 2017-9 Victor Eijkhout eijkhout@tacc.utexas.edu
%%%%
%%%% inheritance.tex : about the relation between classes
%%%%
%%%%%%%%%%%%%%%%%%%%%%%%%%%%%%%%%%%%%%%%%%%%%%%%%%%%%%%%%%%%%%%%

\Level 0 {Inclusion relations between classes}
\label{sec:hasa}

The data members of an object can be of elementary datatypes, or they
can be objects. For instance, if you write software to manage courses,
each \n{Course} object will likely have a \n{Person} object,
corresponding to the teacher.
\begin{lstlisting}
class Person { /* ... */ }
class Course {
private:
  Person the_instructor;
  int year;
}
class Person {
  string name;
  ....
}
\end{lstlisting}

Designing objects with relations between them
is a great mechanism for writing structured code,
as it makes the objects in code behave like objects in the real world.
The relation where one object contains another, is called a
\indexterm{has-a relation} between classes.

\begin{slide}{Has-a relationship}
  \label{sl:obj-hasa}
  A class usually contains data members. These can be simple types or
  other classes. This allows you to make structured code.
\begin{lstlisting}
class Course {
private:
  Person the_instructor;
  int year;
}
class Person {
  string name;
  ....
}
\end{lstlisting}
  This is
  called the \indexterm{has-a relation}.  
\end{slide}

\Level 1 {Accessors and other methods}

Sometimes a class can behave as if it includes an object of another
class, while not actually doing so. For instance, a line segment can
be defined from two points
\begin{lstlisting}
class Segment {
private:
  Point starting_point,ending_point;
}
  ...
int main() {
  Segment somesegment;
  Point somepoint = somesegment.get_the_end_point();
\end{lstlisting}
or from one point, and a distance and angle:
\begin{lstlisting}
class Segment {
private:
  Point starting_point;
  float length,angle;
}
\end{lstlisting}
In both cases the code using the object is written as if the segment
object contains two points.
This illustrates how object-oriented programming can decouple the
\ac{API} of classes from their actual implementation.

\begin{slide}{Literal and figurative has-a}
  \label{sl:obj-hasa-ish}
\small
  A line segment has a starting point and an end point.
\begin{multicols}{2}
  A \n{Segment}
  class can store those points:
\begin{lstlisting}
class Segment {
private:
  Point starting_point,ending_point;
public:
  Point get_the_end_point() {
    return ending_point; };
}
  ...
  Segment somesegment;
  Point somepoint =
    somesegment.get_the_end_point();
\end{lstlisting}
\columnbreak
or store one and derive the other:
\begin{lstlisting}
class Segment {
private:
  Point starting_point;
  float length,angle;
public:
  Point get_the_end_point() {
    /* some computation from the
       starting point */ };
}
\end{lstlisting}
\vfill\hbox{}
\end{multicols}
Implementation vs API: implementation can be very different from user interface.
\end{slide}

Related to this decoupling, a class can also have two very different constructors.
\begin{lstlisting}
class Segment {
private:
 // up to you how to implement!
public:
  Segment( Point start,float length,float angle )
    { .... }
  Segment( Point start,Point end ) { ... }
\end{lstlisting}
Depending on how you actually implement the class, the one constructor
will simply store the defining data, and the other will do some
conversion from the given data to the actually stored data.

This is another strength of object-oriented programming: you can
change your mind about the implementation of a class without having
to change the program that uses the class.

\begin{slide}{Polymorphism in constructors}
  \label{sl:obj-poly-construct}
  You have to decide what to store and what to derive, but you can
  construct two ways:
\begin{lstlisting}
class Segment {
private:
 // up to you how to implement!
public:
  Segment( Point start,float length,float angle )
    { .... }
  Segment( Point start,Point end ) { ... }
\end{lstlisting}
Advantage: with a good API you can change your mind about the
implementation without bothering the user.
\end{slide}

\begin{exercise}
  If you are doing the geometry project, this is a good time to
  do the exercises in section~\ref{sec:poly-rectangle}.
\end{exercise}

\Level 0 {Inheritance}
\label{sec:inheritance}

In addition to the has-a relation, there is the \indexterm{is-a
  relation}, also called \indextermdef{inheritance}. Here one class is
a special case of another class.
Typically the object of the \indextermsub{derived}{class} (the special
case) then also inherits the data and methods of the
\indextermsub{base}{class} (the general case).
\begin{lstlisting}
class General {
protected: // note!
 int g;
public:
 void general_method() {};
};

class Special : public General {
public:
  void special_method() { ... g ... };
};
\end{lstlisting}

\begin{slide}{General case, special case}
  \label{sl:obj-case}
  You can have classes where an object of one class is a special case of
  the other class. You declare that as
\begin{lstlisting}
class General {
protected: // note!
 int g;
public:
 void general_method() {};
};

class Special : public General {
public:
  void special_method() { g = ... };
};

int main() {
  Special special_object;
  special_object.general_method();
  special_object.special_method();
}
\end{lstlisting}
\end{slide}

How do you define a derived class?
\begin{itemize}
\item You need to indicate what the base class is:
\begin{lstlisting}
class Special : public General { .... }
\end{lstlisting}
\item The base class needs to declare its data members as
  \indextermttdef{protected}: this is similar to private, except that
  they are visible to derived classes
\item The methods of the derived class can then refer to data of the
  base class;
\item Any method defined for the base class is available as a method
  for a derived class object.
\end{itemize}

\begin{slide}{Inheritance: derived classes}
  \label{sl:obj-derive}
  \emph{Derived}\index{class!derived} class \n{Special}
  \emph{inherits}\index{inheritance} methods and data from
  \indextermsub{base}{class} \n{General}:
\begin{lstlisting}
int main() {
  Special special_object;
  special_object.general_method();
\end{lstlisting}
Members and methods need to be \n{protected}, not \n{private}, to be inheritable.
\end{slide}

The derived class has its own constructor, with the same name as the
class name, but when it is invoked, it also calls the constructor of
the base class. This can be the default constructor, but often you
want to call the base constructor explicitly, with parameters that are
describing how the special case relates to the base case. Example:
\begin{lstlisting}
class General {
public:
  General( double x,double y ) {};
};
class Special : public General {
public:
  Special( double x ) : General(x,x+1) {};
};
\end{lstlisting}

\begin{slide}{Constructors}
  \label{sl:obj-derive-construct}
  When you run the special case constructor, usually the general case
  needs to run too. By default the `default constructor', but:
\begin{lstlisting}
class General {
public:
  General( double x,double y ) {};
};
class Special : public General {
public:
  Special( double x ) : General(x,x+1) {};
};
\end{lstlisting}
\end{slide}

\Level 1 {Methods of base and derived classes}
\label{sec:derive-method}

\begin{block}{Overriding methods}
  \label{sl:obj-method-override}
  \begin{itemize}
  \item A derived class can inherit a method from the base class.
  \item A derived class can define a method that the base class does
    not have.
  \item A derived class can \emph{override}\index{method!overriding} a
    base class method:
\begin{lstlisting}
class Base {
public:
  virtual f() { ... };
};
class Deriv : public Base {
public:
  virtual f() override { ... };
};
\end{lstlisting}
  \end{itemize}
\end{block}

\begin{exercise}
  If you are doing the geometry project, 
  you can now do the exercises in section~\ref{sec:geom-isa}.
\end{exercise}

\Level 1 {Virtual methods}

\begin{block}{Base vs derived methods}
  \begin{itemize}
  \item Method defined in base class: can be used in any derived class.
  \item Method define in derived class: can only be used in that
    particular derived class.
  \item Method defined both in base and derived class, marked
    \indextermtt{override}: derived class method replaces (or extends)
    base class method.
  \item Virtual method: base class only declares that a routine has to
    exist, but does not give base implementation.\\ A class is called
    \indextermsub{abstract}{class} if it has virtual methods; pure
    virtual if all methods are virtual.\\ You can not make abstract objects.
  \end{itemize}  
\end{block}

\begin{block}{Abstract classes}
  Special syntax for \indextermsub{abstract}{method}:
\begin{lstlisting}
class Base {
public:
  virtual void f() = 0;
};
class Deriv {
public:
  virtual void f() { ... };
};
\end{lstlisting}
\end{block}

\begin{block}{Example: using virtual class}
  \small
  \begin{multicols}{2}
    \verbatimsnippet{virtfuncvirt}
    \vfill\columnbreak
    Suppose \n{DenseVector} derives from \n{VirtualVector}:\\
    \verbatimsnippet{virtfuncmain}
  \end{multicols}
\end{block}

\begin{block}{Implementation}
  \footnotesize
  \begin{multicols}{2}
    \verbatimsnippet{virtfuncdense}
    \vfill\columnbreak
    \verbatimsnippet{virtfuncfunc}
  \end{multicols}
\end{block}

\Level 1 {Advanced topics in inheritance}

\begin{block}{More}
  \label{sl:obj-more}  
  \begin{itemize}
  \item  Multiple inheritance: an~X is-a~A, but also is-a~B.\\
    This mechanism is somewhat dangerous.
  \item Virtual base class: you don't actually define a function in
    the base class, you only say `any derived class has to define this
    function'.
  \item Friend classes:
\begin{lstlisting}
class A;
class B {
  friend class A;
private:
  int i;
};
class  A {
public: 
  void f(B b) { b.i; };
};
\end{lstlisting}
  A \indextermtt{friend} class can access private data and methods
  even if there is no inheritance relationship.
  \end{itemize}
\end{block}

