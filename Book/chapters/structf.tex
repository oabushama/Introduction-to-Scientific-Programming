% -*- latex -*-
%%%%%%%%%%%%%%%%%%%%%%%%%%%%%%%%%%%%%%%%%%%%%%%%%%%%%%%%%%%%%%%%
%%%%
%%%% This TeX file is part of the course
%%%% Introduction to Scientific Programming in C++/Fortran2003
%%%% copyright 2017-9 Victor Eijkhout eijkhout@tacc.utexas.edu
%%%%
%%%% structf.tex : types Fortran
%%%%
%%%%%%%%%%%%%%%%%%%%%%%%%%%%%%%%%%%%%%%%%%%%%%%%%%%%%%%%%%%%%%%%

Fortran has structures for bundling up data, but there is no
\n{struct} keyword: instead you declare both the structure type and
variables of that \indextermsub{derived}{type} with the 
\indextermfort{type} keyword. 

\begin{slide}{Structures: \noexpand\texttt{type}}
  \label{sl:ftype}
  \begin{itemize}
  \item Fortran has structures too.
  \item Structures are a \indextermsub{derived}{type}: you can create
    variables of that type, but it's not a built-in type.
  \item  Fortran derived types are declared in a \indextermfort{type}
    definition;
  \item the \indextermfort{type} keyword is also used for making
    structure variables.    
  \end{itemize}
\end{slide}

Now you need to
\begin{itemize}
\item Define the type to describe what's in it;
\item Declare variables of that type; and
\item use those variables, but setting the type members or using their
  values.
\end{itemize}

\begin{block}{Type declaration}
  \label{sl:ftype-def}
  \n{Type name} / \n{End Type} block.\\
  Component declarations inside the block:
\begin{lstlisting}
type mytype
  integer :: number
  character :: name
  real(4) :: value
end type mytype
\end{lstlisting}
\end{block}

\begin{block}{Creating a type structure}
  \label{sl:ftype-set}
  Declare a type object in the main program:
\begin{lstlisting}
Type(mytype) :: object1,object2
\end{lstlisting}
 Initialize with type name:
\begin{lstlisting}
object1 = mytype( 1, 'my_name', 3.7 )
\end{lstlisting}
Copying:
\begin{lstlisting}
object2 = object1
\end{lstlisting}

\end{block}

\begin{block}{Member access}
  \label{sl:ftype-access}
  Access structure members with \verb+%+
\begin{lstlisting}
Type(mytype) :: typed_object
typed_object%member = ....  
\end{lstlisting}
\end{block}

\begin{block}{Example}
  \label{sl:ftype-ex}
  \begin{multicols}{2}
    \verbatimsnippet{ftypedef}
    \columnbreak
    \verbatimsnippet{ftypeuse}
  \end{multicols}
  Type definitions can go in the main program\\
  (or use a \indextermfort{module}; see later)
\end{block}

You can have arrays of types:
\begin{lstlisting}
type(my_struct) :: data
type(my_struct),dimension(1) :: data_array
\end{lstlisting}

\begin{block}{Structures as subprogram argument}
  \label{sl:ftype-pass}
  Structures can be passed as subprogram argument, just like any other
  datatype. This example:
  \begin{itemize}
  \item Takes a structure of \lstinline{type(point)} as argument; and
  \item returns a \lstinline{real(4)} result.
  \item The structure is declared as \lstinline{intent(in)}.
  \end{itemize}

  \begin{multicols}{2}
    Function with structure argument:
    \verbatimsnippet{ftypepass}
    \columnbreak
    Function call
    \verbatimsnippet{ftypecall}
  \end{multicols}

\end{block}

\begin{exercise}
  \label{ex:ftype-rect}
  Write a program that has the following:
  \begin{itemize}
  \item A type \n{Point} that contains real numbers~\n{x,y};
  \item a type \n{Rectangle} that contains two \n{Point}s,
    corresponding to the lower left and upper right point;
  \item a function \n{area} that has one argument: a \n{Rectangle}.
  \end{itemize}
  Test your program.
\end{exercise}
