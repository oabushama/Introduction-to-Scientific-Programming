
\begin{exercise}
  \label{ex:modulus}
  Write a program that ask for two integer numbers \n{n1,n2}.
  \begin{itemize}
  \item Assign the integer ratio $n_1/n_2$ to a variable.
  \item Can you use this variable to compute the modulus
    \[ n_1\mod n_2 \]
    (without using the \n{\char`\%} modulus operator!)\\
    Print out the value you get.
  \item Also print out the result from using the modulus operator:\n{\char`\%}.
  \end{itemize}
\end{exercise}

Complex numbers exist, see section~\ref{sec:stl-complex}.

\Level 0 {Library functions}

Some functions, such as \indexterm{abs} can be included through \indextermtt{cmath}:
\begin{verbatim}
#include <cmath>
using std::abs;
\end{verbatim}
Others, such as \indexterm{max}, are in the less common \indextermtt{algorithm}:
\begin{verbatim}
#include <algorithm>
using std::max;
\end{verbatim}

\Level 0 {Review questions}

\begin{exercise}
  \label{ex:cpp-mod}
What is the output of:
\begin{verbatim}
int m=32, n=17;
cout << n%m << endl;
\end{verbatim}
\end{exercise}

\begin{exercise}
  \label{ex:cpp-cube}
  Given
\begin{verbatim}
int n;
\end{verbatim}
write code that
uses elementary mathematical operators to compute n-cubed: $n^3$.
Do you get the correct result for all~$n$? Explain.
\end{exercise}
