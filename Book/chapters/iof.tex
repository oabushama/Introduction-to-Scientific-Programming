% -*- latex -*-
%%%%%%%%%%%%%%%%%%%%%%%%%%%%%%%%%%%%%%%%%%%%%%%%%%%%%%%%%%%%%%%%
%%%%
%%%% This TeX file is part of the course
%%%% Introduction to Scientific Programming in C++/Fortran2003
%%%% copyright 2017-9 Victor Eijkhout eijkhout@tacc.utexas.edu
%%%%
%%%% iof.tex : input and output in Fortran
%%%%
%%%%%%%%%%%%%%%%%%%%%%%%%%%%%%%%%%%%%%%%%%%%%%%%%%%%%%%%%%%%%%%%

\Level 0 {Types of I/O}

Fortran can deal with input/output in ASCII format, which is called
\emph{formatted}\index{I/O!formatted} I/O, and binary, or
\emph{unformatted}\index{I/O!unformatted}~I/O.  Formatted I/O can use
default formatting, but you can specify detailed formatting
instructions, in the I/O statement itself, or separately in a
\n{Format} statement.

Fortran I/O can be described as \indextermsub{list-directed}{I/O}:
both input and output commands take a list of item, possibly with
formatting specified.

\begin{block}{I/O commands}
  \label{sl:fio-commands}
  \begin{itemize}
  \item \indextermfort{Print} simple output to terminal
  \item \indextermfort{Write} output to terminal or file (`unit')
  \item \indextermfort{Read} input from terminal or file
  \item \indextermfort{Open}, \indextermfort{Close} for files and
    streams
  \item \indextermfort{Format} format specification that can be used
    in multiple statements.
  \end{itemize}
\end{block}

\begin{block}{Formatted and unformatted I/O}
  \label{sl:fio-types}
  \begin{itemize}
  \item Formatted: ascii output. This is good for reporting, but not
    for numeric data storage.
  \item Unformatted: binary output. Great for further processing of
    output data.
  \item Beware: binary data is machine-dependent. Use \indexterm{hdf5}
    for portable binary.
  \end{itemize}
\end{block}

\Level 0 {Print to terminal}

The simplest command for outputing data is \indextermtt{print}.
\begin{lstlisting}
print *,"The result is",result
\end{lstlisting}
In its easiest form, you use the star to indicate that you don't care
about formatting; after the comma you can then have any number of
comma-separated strings and variables.

\Level 1 {Print on one line}

The statement
\begin{lstlisting}
print *,item1,item2,item3
\end{lstlisting}
will print the items on one line, as far as the line length allows.

\begin{slide}{Simple print}
  \label{sl:fio-print}
  All on one line:
\begin{lstlisting}
print *,"The result is",result
print *,item1,item2,item3
\end{lstlisting}
\end{slide}

\begin{block}{Implicit do loops}
  \label{sl:print-implicit-loop}
  Parametrized printing with an \indextermsub{implicit}{do loop}:
\begin{lstlisting}
print *,( i*i,i=1,n)
\end{lstlisting}
All values will be printed on the same line.
\end{block}

\Level 1 {Printing arrays}

If you print a variable that is an array, all its elements will be
printed, in \indexterm{column-major} ordering if the array is
multi-dimensional.

You can also control the printing of an array by using an
\indextermsub{implicit}{do loop}:
\begin{lstlisting}
print *,( A(i),i=1,n)
\end{lstlisting}

\begin{slide}{Array printing}
  \label{sl:fprint-array}
  \begin{itemize}
  \item \n{print *,A} prints whole array, column-major
  \item Implicit do loops:
\begin{lstlisting}
print *,( A(i,i),i=1,n)
\end{lstlisting}
  Can also be nested.
  \end{itemize}
\end{slide}

\Level 1 {Formats}

The default formatting uses quite a few positions for what can be
small numbers. To indicate explicitly the formatting, for instance
limiting the number of positions used for a number, or the whole and
fractional part of a real number, you can use a format string.

For instance, you can use a letter followed by a digit to control the
formatting width:
%
\snippetwithoutput{i4}{iof}{i4}
%
(In the last line, the format specifier was not wide enough for the
data, so an
\emph{asterisk}\index{asterisk!in Fortran formatted I/O}
was used as output.)

If you want to display items of the same type, you can use a repeat
count:
%
\snippetwithoutput{ii}{iof}{ii}

You can mix variables of different types, as well as literal string,
by separating them with commas. And you can group them with
parentheses, and put a repeat count on those groups again:
%
\snippetwithoutput{ij}{iof}{ij}

To be precise, the format specifier is inside single quotes and parentheses, and
consists of comma-separated specifications for a single item:
\begin{itemize}
\item `\n{a}$n$' specifies a string of $n$~characters. If the actual
  string is longer, it is truncated in the output.
\item `\n{i}$n$' specifies an integer of up to~$n$ digits. If the actual
  number takes more digits, it is rendered with asterisks.
  To use the minimum number of digits, use~\n{i}$0$.
\item `\n{f}$m.n$' specifies a fixed point representation of a real
  number, with $m$~total positions (including the decimal point)
  and $n$~digits in the fractional part.
\item `\n{e}$m.n$' Exponent representation.
\item Strings can go into the format:
\begin{lstlisting}
print '("Result:",3f5.3)',x,y,z
\end{lstlisting}
\item `\n{x}' for a space, `\n{/}'~for newline
\end{itemize}
Putting a number in front of a single specifier indicates that it is
to be repeated.

If the data takes more positions than the format specifier allows, a
string of asterisks is printed:
%
\snippetwithoutput{fasterisk}{fio}{asterisk}

If you find yourself using the same format a number of times, you can
give it a \indexterm{label}:
\begin{lstlisting}
  print 10,"result:",x,y,z
10 format('(a6,3f5.3)')
\end{lstlisting}

\url{https://www.obliquity.com/computer/fortran/format.html}

\begin{slide}{Formats}
  \label{sl:formats}
  \begin{itemize}
  \item
    Fine control of input/output.
  \item
    Direct use in print statement:  
\begin{lstlisting}
print '(a6,3f5.3)',"Result",x,y,z
print '("Result:",3f5.3)',x,y,z
\end{lstlisting}
\item Format statement:
  \end{itemize}
\begin{lstlisting}
  print 10,"result:",x,y,z
10 format('(a6,3f5.3)')
\end{lstlisting}
\end{slide}

\begin{slide}{Format specifiers}
  \label{sl:formatspec}
  \begin{itemize}
  \item `\n{a}$n$' specifies a string of $n$~characters. If the actual
    string is longer, it is truncated in the output.
  \item `\n{i}$n$' specifies an integer of up to~$n$ digits. If the actual
    number takes more digits, it is rendered with asterisks.
  \item `\n{f}$m.n$' specifies a fixed point representation of a real
    number, with $m$~total positions (including the decimal point)
    and $n$~digits in the fractional part.
  \item `\n{e}$m.n$' Exponent representation.
  \item Strings can go into the format:
\begin{lstlisting}
print '("Result:",3f5.3)',x,y,z
\end{lstlisting}
\item `\n{x}' for a space, `\n{/}'~for newline
  \end{itemize}
\end{slide}

\begin{block}{Format repetitions}
  \label{sl:fformat-rep}
\begin{lstlisting}
print '( 3i4 )', i1,i2,i3
print '( 3(i2,":",f7.4) )', i1,r1,i2,r2,i3,r2
\end{lstlisting}
\end{block}

\begin{block}{Repeats and line breaks}
  \label{sl:formatrepeat}
  \begin{itemize}
  \item If \n{abc} is a format string, then \n{10(abc)} gives 10
    repetitions. There is no line break.
  \item If there is more data than specified in the format, the format
    is reused in a new print statement. This causes line breaks.
  \item The \n{/} (slash) specifier causes a line break.
  \item There may be a 80 character limit on output lines.
  \end{itemize}
\end{block}

\begin{exercise}
  \label{ex:f99}
  Use formatted I/O to print the number $0\cdots99$ as follows:
\begin{lstlisting}
 0  1  2  3  4  5  6  7  8  9
10 11 12 13 14 15 16 17 18 19
20 21 22 23 24 25 26 27 28 29
30 31 32 33 34 35 36 37 38 39
40 41 42 43 44 45 46 47 48 49
50 51 52 53 54 55 56 57 58 59
60 61 62 63 64 65 66 67 68 69
70 71 72 73 74 75 76 77 78 79
80 81 82 83 84 85 86 87 88 89
90 91 92 93 94 95 96 97 98 99
\end{lstlisting}
\end{exercise}

\Level 0 {File and stream I/O}

If you want to send output anywhere else than the terminal screen, you
need the \indextermfort{write} statement, which looks like:
\begin{lstlisting}
write (unit,format) data
\end{lstlisting}
where \n{format} and \n{data} are as described above. The new element
is the \indexterm{unit}, which is a numerical indication of an output
device, such as a file.

\Level 1 {Units}

\begin{lstlisting}
Open(11)
\end{lstlisting}
will result in a file with a name typically \n{fort11}.
\begin{lstlisting}
Open(11,FILE="filename")
\end{lstlisting}
Many other options for error handling, new vs old file, etc.

After this:
\begin{lstlisting}
Write (11,fmt) data
\end{lstlisting}
Again options for errors and such.

\Level 1 {Other write options}

\begin{lstlisting}
write(unit,fmt,ADVANCE="no") data
\end{lstlisting}
will not issue a newline.

\indextermfort{open} \indextermfort{Close}

\Level 0 {Unformatted output}
\label{sec:rawdataf}

So far we have looked at ascii output, which is nice to look at for a
human , but is not the right medium to communicate data to another
program.
\begin{itemize}
\item Ascii output requires time-consuming conversion.
\item Ascii rendering leads to loss of precision.
\end{itemize}
Therefore, if you want to output data that is later to be read by a
program, it is best to use \indextermsub{binary}{output} or
\indextermsub{unformatted}{output}, sometimes also called
\indextermsub{raw}{output}.

\begin{block}{Unformatted output}
  Indicated by lack of format specification:
\begin{lstlisting}
write (*) data
\end{lstlisting}
  Note: may not be portable between machines.
\end{block}
