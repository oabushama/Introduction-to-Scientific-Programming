
\Level 0 {Example: linked lists}

\begin{block}{Linked list}
  \begin{itemize}
  \item Linear data structure
  \item more flexible for insertion~/ deletion
  \item \ldots~but slower in access
  \end{itemize}
\end{block}

One of the standard examples of using pointers is the
\indexterm{linked list}. This is a dynamic one-dimensional structure
that is more flexible than an array. Dynamically extending an array
would require re-allocation, while in a list an element can be
inserted.

\begin{exercise}
  Using a linked list may be more flexible than using an array.
  On the other hand, accessing an element in a linked list is
  more expensive, both absolutely and as order-of-magnitude in the size
  of the list.

  Make this argument precise.
\end{exercise}

\begin{block}{Linked list datatypes}
  \begin{itemize}
  \item Node: value field, and pointer to next node.
  \item List: pointer to head node.
  \end{itemize}
  \verbatimsnippet{linklistf}
\end{block}

A list is based on a simple data structure, a node, which contains a
value field and a pointer to another node.

By way of example, we create a dynamic list of integers, sorted by
size. To maintain the sortedness, we need to append or insert nodes,
as required.

Here are the basic definitions of a node, and a list which is
basically a repository for the head node:
%
\verbatimsnippet{linklistf}

\begin{block}{List initialization}
  First element becomes the list head:

\verbatimsnippet{listheadf}
  
\end{block}

Initially, the list is empty, meaning that the `head' pointer is
un-associated. The first time we add an element to the list, we create
a node and assign it as the head of the list:
%
\verbatimsnippet{listheadf}

\begin{block}{Attaching a node}
  Keep the list sorted: new largest element attached at the end.
  
  \verbatimsnippet{listattachf}
\end{block}

Adding a value to a list can be done two ways. If the new element is
larger than all elements in the list, a new node needs to be appended
to the last one. Let's assume we have managed
to let \n{current} point at the last node of the list,
then here is how to 
attaching a new node from it:
%
\verbatimsnippet{listattachf}

\begin{block}{Inserting 1}
  Find the insertion point:
\verbatimsnippet{listfindf}
\end{block}

Inserting an element in the list is harder.
First of all, you need to find the two nodes,
\n{previous} and \n{current}, between which to insert the new node:
%
\verbatimsnippet{listfindf}

\begin{block}{Inserting 2}
  The actual insertion requires rerouting some pointers:
  %
  \verbatimsnippet{listinsertf}
\end{block}
