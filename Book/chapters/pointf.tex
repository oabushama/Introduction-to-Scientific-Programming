% -*- latex -*-
%%%%%%%%%%%%%%%%%%%%%%%%%%%%%%%%%%%%%%%%%%%%%%%%%%%%%%%%%%%%%%%%
%%%%
%%%% This TeX file is part of the course
%%%% Introduction to Scientific Programming in C++/Fortran2003
%%%% copyright 2017/8 Victor Eijkhout eijkhout@tacc.utexas.edu
%%%%
%%%% pointf.tex : arrays in Fortran
%%%%
%%%%%%%%%%%%%%%%%%%%%%%%%%%%%%%%%%%%%%%%%%%%%%%%%%%%%%%%%%%%%%%%

Pointers in C++ were largely the same as memory addresses (until you
got to smart pointers). Fortran pointers on the other hand, are more
abstract.

\Level 0 {Basic pointer operations}

\begin{block}{Pointers are aliases}
  \label{sl:fpoint}
  \begin{itemize}
  \item Pointer points at an object
  \item Access object through pointer
  \item You can change what object the pointer points at.
  \end{itemize}
\begin{verbatim}
real,pointer :: point_at_real
\end{verbatim}
\end{block}

Pointers could also be called `aliases': they act like an alias for an
object of elementary or derived data type. You can access the object
through the alias. The difference with actually using the object, is
that you can decide what object the pointer points at.

\begin{block}{C++ vs Fortran pointers}
  \label{sl:cpoint-vs-fpoint}
  Fortran pointers are automatically
  \emph{dereferenced}\index{pointer!dereferencing}: if you print a
  pointer you print the object it references, not some representation
  of the pointer.
\end{block}

The \indextermtt{pointer} definition
\begin{verbatim}
real,pointer :: point_at_real
\end{verbatim}
defined a pointer that can point at a real variable.

\begin{block}{Setting the pointer}
  \label{sl:fpoint-set}
  \begin{itemize}
  \item You have to declare that a variable is pointable:
\begin{verbatim}
real,target :: x
\end{verbatim}
\item Set the pointer with \verb+=>+ notation:
\begin{verbatim}
point_at_real => x
\end{verbatim}
\item Now using \n{point_at_real} is the same as using~\n{x}.
  \end{itemize}
\end{block}

Pointers can not just point at anything: the thing pointed at needs to
be declared as \indextermtt{target}
\begin{verbatim}
real,target :: x
\end{verbatim}
and you use the \verb+=>+ operator to let a pointer point at a target:
\begin{verbatim}
point_at_real => x
\end{verbatim}

If you use a pointer, for instance to print it
\begin{verbatim}
print *,point_at_real
\end{verbatim}
it behaves as if you were using the value of what it points at.

\begin{block}{Pointer example}
  \label{sl:fpoint-ex}
  \verbatimsnippet{pointatreal}
  %
  \begin{enumerate}
  \item The pointer points at~\n{x}, so the value of \n{x} is printed.
  \item The pointer is set to point at~\n{y}, so its value is printed.
  \item The value of \n{y} is changed, and since the pointer still
    points at~\n{y}, this changed value is printed.
  \end{enumerate}
\end{block}

\begin{block}{Assign pointer from other pointer}
  \label{sl:fpoint-copy}
\begin{verbatim}
real,pointer :: point_at_real,also_point
point_at_real => x
also_point => point_at_real  
\end{verbatim}
  Now you have two pointers that point at~\n{x}.

  \textbf{Very important to use the \n{=>}, otherwise strange
    memory errors}
\end{block}

If you have two pointers
\begin{verbatim}
real,pointer :: point_at_real,also_point
\end{verbatim}
you can make the target of the one to also be the target of the other:
\begin{verbatim}
also_point => point_at_real  
\end{verbatim}
This is not a pointer to a pointer: it assigns the target of the
right-hand side to be the target of the left-hand side.

\textbf{Using ordinary assignment does not work, and will give strange
  memory errors.}

\begin{exercise}
  \label{ex:fpoint-fun}
  Write a routine that accepts an array and a pointer, and on return
  has that pointer pointing at the largest array element:
  %
  \verbatimsnippet{arraypointmain}
\end{exercise}

\begin{block}{Pointer status}
  \label{sl:fpoint-stat}
  \begin{itemize}
  \item \indextermfort{Nullify}: zero a pointer
  \item \indextermfort{Associated}: test whether assigned
  \end{itemize}  
\end{block}

\begin{block}{Dynammic allocation}
  \label{sl:fpoint-dynamic}
  Associate unnamed memory:
\begin{verbatim}
Integer,Pointer,Dimension(:) :: array_point
Allocate( array_point(100) )
\end{verbatim}  
\end{block}

\Level 0 {Pointers and arrays}

You can set a pointer to an array:
\begin{verbatim}
real(8),dimension(10),target :: array
real(8),pointer,dimension(:) :: array_ptr

array_ptr => array
\end{verbatim}
More surprising, you can set pointers to array slices:
\begin{verbatim}
array_ptr => array(2:)
array_ptr => array(1:size(array):2)
\end{verbatim}

In case you're wondering, this does not create temporary arrays, but
the compiler adds descriptions to the pointers, to translate code
automatically to strided indexing.

\Level 0 {Example: linked lists}
\index{list!linked!in Fortran|(}

\begin{block}{Linked list}
  \label{sl:flink1}
  \begin{itemize}
  \item Linear data structure
  \item more flexible for insertion~/ deletion
  \item \ldots~but slower in access
  \end{itemize}
\end{block}

One of the standard examples of using pointers is the
\emph{linked list}. This is a dynamic one-dimensional structure
that is more flexible than an array. Dynamically extending an array
would require re-allocation, while in a list an element can be
inserted.

\begin{exercise}
  Using a linked list may be more flexible than using an array.
  On the other hand, accessing an element in a linked list is
  more expensive, both absolutely and as order-of-magnitude in the size
  of the list.

  Make this argument precise.
\end{exercise}

\begin{block}{Linked list datatypes}
  \label{sl:flink2}
  \begin{itemize}
  \item Node: value field, and pointer to next node.
  \item List: pointer to head node.
  \end{itemize}
  \verbatimsnippet{linklistf}
\end{block}

A list is based on a simple data structure, a node, which contains a
value field and a pointer to another node.

By way of example, we create a dynamic list of integers, sorted by
size. To maintain the sortedness, we need to append or insert nodes,
as required.

Here are the basic definitions of a node, and a list which is
basically a repository for the head node:
%
\verbatimsnippet{linklistf}

\begin{block}{List initialization}
  \label{sl:flink3}
  First element becomes the list head:

\verbatimsnippet{listheadf}
  
\end{block}

Initially, the list is empty, meaning that the `head' pointer is
un-associated. The first time we add an element to the list, we create
a node and assign it as the head of the list:
%
\verbatimsnippet{listheadf}

\begin{block}{Attaching a node}
  \label{sl:flink4}
  Keep the list sorted: new largest element attached at the end.
  
  \verbatimsnippet{listattachf}
\end{block}

Adding a value to a list can be done two ways. If the new element is
larger than all elements in the list, a new node needs to be appended
to the last one. Let's assume we have managed
to let \n{current} point at the last node of the list,
then here is how to 
attaching a new node from it:
%
\verbatimsnippet{listattachf}

\begin{block}{Inserting 1}
  \label{sl:flink5}
  Find the insertion point:
\verbatimsnippet{listfindf}
\end{block}

Inserting an element in the list is harder.
First of all, you need to find the two nodes,
\n{previous} and \n{current}, between which to insert the new node:
%
\verbatimsnippet{listfindf}

\begin{block}{Inserting 2}
  \label{sl:flink6}
  The actual insertion requires rerouting some pointers:
  %
  \verbatimsnippet{listinsertf}
\end{block}

\index{list!linked!in Fortran|)}

