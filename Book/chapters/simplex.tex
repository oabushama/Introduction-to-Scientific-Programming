
\Level 0 {Object oriented exercises}

\begin{exercise}
  Why is it a good idea to use an accessor function for the data
  members of a class, rather than
  declaring data members \n{public} and accessing them directly?
\end{exercise}

\begin{exercise}
  You are programming a video game. There are moving elements, and you
  want to have an object for each. Moving elements need to have a
  method \n{move} with an argument that indicates a time duration, and
  this method updates the position of the element, using the speed of
  that object and the duration.

  Supply the missing bits of code.
\begin{verbatim}
class position {
  /* ... */
public:
  position() {};
  position(int initial) { /* ... */ };
  void move(int distance) { /* ... */ };
};
class actor {
protected:
  int speed;
  position current;

public:
  actor() { current = position(0); };
  void move(int duration) {
    /* THIS IS THE EXERCISE: */
    /* write the body of this function */
  };
};
class human : public actor {
public:
  human() // EXERCISE: write the constructor
};
class airplane : public actor {
public:
  airplane() // EXERCISE: write the constructor
};

int main() {
  human Alice;
  airplane Seven47;
  Alice.move( 5 );
  Seven47.move( 5 );
\end{verbatim}
\end{exercise}

\begin{exercise}
  Let a \n{Point} class be given:
\begin{verbatim}
class Point {
private: 
  double x,y;
public:
  Point( double px,double py ) { x = px; y = py; };
  // maybe some more methods
}
\end{verbatim}
How would you design a \n{Set} class so that you could write
\begin{verbatim}
Point p1,p2,p3;
Set pointset;
pointset.add(p1); pointset.add(p2);
\end{verbatim}
%  \verbatimsnippet{setminusset}
\end{exercise}

