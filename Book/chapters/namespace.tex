% -*- latex -*-
%%%%%%%%%%%%%%%%%%%%%%%%%%%%%%%%%%%%%%%%%%%%%%%%%%%%%%%%%%%%%%%%
%%%%
%%%% This TeX file is part of the course
%%%% Introduction to Scientific Programming in C++/Fortran2003
%%%% copyright 2017-9 Victor Eijkhout eijkhout@tacc.utexas.edu
%%%%
%%%% namespace.tex : about namespaces
%%%%
%%%%%%%%%%%%%%%%%%%%%%%%%%%%%%%%%%%%%%%%%%%%%%%%%%%%%%%%%%%%%%%%

\Level 0 {Solving name conflicts}
\label{sec:usename}

In section~\ref{sec:stdvector} you saw that the C++ library comes with
a \indexterm{vector} class, that implements dynamic arrays. You say
\begin{lstlisting}
std::vector<int> bunch_of_ints;
\end{lstlisting}
and you have an object that can store a bunch of ints. And if you use
such vectors often, you can save yourself some typing by having
\begin{lstlisting}
using namespace std;
\end{lstlisting}
somewhere high up in your file, and write
\begin{lstlisting}
vector<int> bunch_of_ints;
\end{lstlisting}
in the rest of the file.

More safe:
\begin{lstlisting}
using std::vector;
\end{lstlisting}

But what if you are writing a geometry package, which includes a
vector class? Is there confusion with the \ac{STL} vector class?
There would be if it weren't for the phenomenon
\indextermdef{namespace}, which acts as as disambiguating prefix for
classes, functions, variables.

\begin{slide}{You have already seen namespaces}
  \label{sl:vec-namespace}
  Safest:
\begin{lstlisting}
#include <vector>
int main() {
  std::vector<stuff> foo;
}
\end{lstlisting}
\begin{multicols}{2}
  Drastic:
\begin{lstlisting}
#include <vector>
using namespace std;
int main() {
  vector<stuff> foo;
}
\end{lstlisting}
\vfill\columnbreak
Prudent:
\begin{lstlisting}
#include <vector>
using std::vector;
int main() {
  vector<stuff> foo;
}
\end{lstlisting}
\end{multicols}
\end{slide}

You have already seen namespaces in action when you wrote
\n{std::vector}: the `\n{std}' is the name of the namespace.

\begin{block}{Defining a namespace}
  \label{sl:namespace-def}
  You can make your own namespace by writing
\begin{lstlisting}
namespace a_namespace {
  // definitions
  class an_object { 
  };
|
\end{lstlisting}
\end{block}

so that you can write
\begin{block}{Namespace usage}
  \label{sl:namespace-use}
\begin{lstlisting}
a_namespace::an_object myobject();
\end{lstlisting}
or
\begin{lstlisting}
using namespace a_namespace;
an_object myobject();
\end{lstlisting}
or
\begin{lstlisting}
using a_namespace::an_object;
an_object myobject();
\end{lstlisting}
\end{block}

\Level 1 {Namespace header files}

If your namespace is going to be used in more than one program, you
want to have it in a separate source file, with an accompanying header
file:
%
\verbatimsnippet{nameinclude}

The header would contain the normal function and class headers, but
now inside a named namespace:
%
\verbatimsnippet{nameheader}

and the implementation file would have the implementations, in a
namespace of the same name:
%
\verbatimsnippet{nameimpl}

\Level 0 {Best practices}

In this course we advocated
pulling in functions explicitly:
\begin{lstlisting}
#include <iostream>
using std::cout;
\end{lstlisting}

It is also possible to use
\begin{lstlisting}
#include <iostream>
using namespace std;
\end{lstlisting}

The problem with this
is that it may pull in unwanted functions. For instance:

\begin{block}{Why not `using namespace std'?}
  \label{sl:namespace-std-harm}
  \begin{multicols}{2}
    This compiles, but should not:
    \verbatimsnippet{swapname}
    \vfill\columnbreak
    This gives an error:
    \verbatimsnippet{swapusing}
  \end{multicols}
\end{block}

Even if you use \n{using namespace}, you only do this in a source
file, not in a header file. Anyone using the header would have no idea what
functions are suddenly defined.

\begin{slide}{Big namespace no-no}
  \label{sl:h-no-using}
  Do not put \n{using} in a header file that a user may include.
\end{slide}
