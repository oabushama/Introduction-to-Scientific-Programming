% -*- latex -*-
%%%%%%%%%%%%%%%%%%%%%%%%%%%%%%%%%%%%%%%%%%%%%%%%%%%%%%%%%%%%%%%%
%%%%
%%%% This TeX file is part of the course
%%%% Introduction to Scientific Programming in C++/Fortran2003
%%%% copyright 2017-9 Victor Eijkhout eijkhout@tacc.utexas.edu
%%%%
%%%% elementsf.tex : basic language elements of Fortran
%%%%
%%%%%%%%%%%%%%%%%%%%%%%%%%%%%%%%%%%%%%%%%%%%%%%%%%%%%%%%%%%%%%%%

Fortran is an old programming language, dating back to the 1950s, and
the first `high level programming language' that was widely used.
In a way, the fields of programming language design and compiler
writing started with Fortran, rather than this language being based on
established fields. Thus, the design of Fortran has some
idiosyncracies that later designed languages have not adopted. Many of
these are now `deprecated' or simply inadvisable. Fortunately, it is
possible to write Fortran in a way that is every bit as modern and
sophisticated as other current languages.

In this part of our book, we will teach you safe practices for
writing Fortran. Occasionally we will not mention practices that you
will come across in old Fortran codes, but that we would not advise
you taking up. While our exposition of Fortran can stand on its own,
we will in places point out explicitly differences with~C++.

For secondary reading, this is a good course on modern Fortran:
\url{http://www.pcc.qub.ac.uk/tec/courses/f77tof90/stu-notes/f90studentMIF_1.html}

\Level 0 {Source format}

Fortran started in the era when programs were stored on
\indexterm{punch cards}. Those had 80 columns, so a line of Fortran
source code could not have more than 80 characters. Also, the first 6
characters had special meaning. This is referred to as \emph{fixed
  format}\index{source!format!fixed}. However, starting with
\indextermbus{Fortran}{90} it became possible to have \emph{free
  format}\index{source!format!free}, which allowed longer lines
without special meaning for the initial columns.

There are further differences between the two formats (notably
continuation lines) but we will only discuss free format in this course.

Many compilers have a convention for indicating the source format by
the file name extension:
\begin{itemize}
\item \n{f} and \n{f90} are the extensions for old-style fixed format;
  and
\item \n{F} and \n{F90} are the extensions for new free format.
\end{itemize}
The postfix \n{90} indicates that the \indextermbus{C}{preprocessor}
is applied to the file.
For this course we will use the \indextermtt{F90} extension.

\Level 0 {Compiling Fortran}

For Fortran programs, the compiler is \indextermtt{gfortran} for the
GNU compiler, and \indextermtt{ifort} for Intel.

The minimal Fortran program is:
%
\verbatimsnippet{emptyf}

\begin{exercise}
  Add the line
\begin{lstlisting}
print *,"Hello world!"
\end{lstlisting}
to the empty program, and compile and run it.
\end{exercise}

Fortran ignores \emph{case}\index{Fortran!case ignores}.
Both keywords such as \n{Begin} or \n{Program} can just as well be written as
\n{bEGIN} or \n{PrOgRaM}.
%% \footnote{Fortran also ignores spaces. Please
%%   don't use that fact; it only makes your program confusing to read.}.

A program optionally has a \indextermfort{stop} statement, which can
return a message to the \ac{OS}.
%
\snippetwithoutput{stopf}{basicf}{stop}

\Level 0 {Main program}

Fortran does not use curly brackets to delineate blocks, instead you
will find \indextermtt{end} statements. The very first one appears
right when you start writing your program:
a~Fortran program needs to start with a \n{Program} line, and end with
\n{End Program}. The program needs to have a name on both lines:
%
\verbatimsnippet{emptyf}
%
and you can not use that name for any entities in the program.

\Level 1 {Program structure}

Unlike C++, Fortran can not mix variable declarations and executable
statements, so both the main program and any subprograms have roughly
a
structure:
\begin{lstlisting}
Program foo
  < declarations >
  < statements >
End Program foo
\end{lstlisting}
(The \indexterm{emacs} editor will supply the block type and name if
you supply the `end' and hit the TAB or RETURN key; see
section~\ref{sec:editor-mode}.)

\begin{slide}{Program structure}
  \label{sl:programf}
Structure of your program file:
\begin{lstlisting}
Program foo
  < declarations >
  < statements >
End Program foo
\end{lstlisting}
\end{slide}

\begin{block}{Things to note}
  \label{sl:programfnotes}
  \begin{itemize}
  \item No includes before the program
  \item Program has a name (emacs tip: type \n{end<TAB>})
  \item There is an \lstinline{End}, rather than curly braces.
  \item Declarations first, not interspersed.
  \end{itemize}
\end{block}

\Level 1 {Statements}

Let's say a word about layout. Fortran has a `one line, one statement'
principle.
\begin{itemize}
\item As long as a statement fits on one line, you don't have to
  terminate it explicitly with something like a semicolon:
\begin{lstlisting}
x = 1
y = 2
\end{lstlisting}
\item If you want to put two statements on one line, you have to
  terminate the first one:
\begin{lstlisting}
x = 1; y = 2
\end{lstlisting}
\item If a statement spans more than one line, all but the first line
  need to have an explicit \indexterm{continuation character}, the ampersand:
\begin{lstlisting}
x = very &
  long &
  expression
\end{lstlisting}
\end{itemize}

\begin{slide}{Statements}
  \label{sl:fstatement}
  \begin{itemize}
  \item One line, one statement
\begin{lstlisting}
x = 1
y = 2
\end{lstlisting}
(historical reasons)
\item semicolon to separate multiple statements per line
\begin{lstlisting}
x = 1; y = 2
\end{lstlisting}
\item Continuation of a line
\begin{lstlisting}
x = very &
  long &
  expression
\end{lstlisting}
  \end{itemize}
\end{slide}

\Level 1 {Comments}

Fortran knows only single-line
\emph{comments}\index{Fortran!comments},
indicated by an exclamation point:
\begin{lstlisting}
x = 1 ! set x to one
\end{lstlisting}
Everything from the exclamation point onwards is ignored.

Maybe not entirely obvious: you can have a comment after a
continuation character:
\begin{lstlisting}
x = f(a) & ! term1 
  + g(b)   ! term2
\end{lstlisting}

\begin{slide}{Comments}
  \label{sl:fcomment}
  \begin{itemize}
  \item Ignore to end of line
\begin{lstlisting}
x = 1 ! set x to one
\end{lstlisting}
\item comment after continuation
\begin{lstlisting}
x = f(a) & ! term1 
  + g(b)   ! term2
\end{lstlisting}
  \item No multi-line comments.
  \end{itemize}
\end{slide}

\Level 0 {Variables}

Unlike in C++, where you can declare a variable right before you need
it, Fortran wants its variables declared near the top of the program
or subprogram:
\begin{lstlisting}
Program YourProgram
  implicit none
  ! variable declaration
  ! executable code
End Program YourProgram
\end{lstlisting}
A variable declaration looks like:
\begin{lstlisting}
type [ , attributes ] :: name1 [ , name2, .... ]
\end{lstlisting}
where
\begin{itemize}
\item we use the common grammar shorthand that \n{[ something ]}
  stands for an optional `something';
\item \textit{type} is most commonly \n{integer}, \n{real(4)}, \n{real(8)},
  \n{logical}. See below; section~\ref{sec:ftype}.
\item \textit{attributes} can be \n{dimension}, \n{allocatable},
  \n{intent}, \n{parameters} et cetera.
\item \textit{name} is something you come up with. This has to start
  with a letter.
\end{itemize}

\begin{slide}{Variable declarations}
  \label{sl:fvars}
  \begin{itemize}
  \item Variable declarations at the top of the program unit,\\
    before any executable statements.
  \item declaration
\begin{lstlisting}
type, attributes :: name1, name2, ....
\end{lstlisting}
where
\begin{itemize}
\item \textit{type} is most commonly \n{integer}, \n{real(4)}, \n{real(8)},
  \n{logical}
\item \textit{attributes} can be \n{dimension}, \n{allocatable},
  \n{intent}, \n{parameters} et cetera.
\end{itemize}
\item Keywords and variables are
  \emph{case-insensitive}\index{Fortran!case-insensitive}
\end{itemize}
\end{slide}

\begin{block}{Data types}
  \label{sl:ftypes}
  \begin{itemize}
  \item Numeric: \indextermfort{Integer}, \indextermfort{Real},
    \indextermfort{Complex}. 
  \item Logical: \indextermfort{Logical}.
  \item Character: \indextermfort{Character}. Strings are realized as
    arrays of characters.
  \end{itemize}  
\end{block}

\begin{slide}{Strings}
  \label{sl:fchar20}
  \begin{lstlisting}
    character*20 :: prompt
    prompt = "up to 20 characters"
  \end{lstlisting}
\end{slide}

Further specifications for numerical
    precision are discussed in section~\ref{sec:fprecision}.
Strings are discussed in chapter~\ref{ch:stringf}.

\Level 1 {Declarations}
\label{sec:ftype}

Fortran has a somewhat unusual treatment of data types: if you don't
specify what data type a variable is, Fortran will deduce it from
a~default or user rules. This is a very dangerous practice, so we
advocate putting a line
\begin{lstlisting}
implicit none
\end{lstlisting}
immediately after any program or subprogram header.

\begin{slide}{Implicit typing}
  \label{sl:fimplicit}
  Fortran does not need variable declarations (like python):\\
  variables have a type that is determined by name.\\
  This is \textbf{very dangerous}. Use \lstinline{implicit none}
  in every program unit.
\begin{lstlisting}
Program myprogram
  implicit none
  integer :: i
  real :: x
  ! more stuff
End Program myprogram
\end{lstlisting}
\end{slide}

You can the number of bytes for numerical types in the declaration:
%
\verbatimsnippet{fstorage}

\begin{slide}{Floating point types}
  \label{sl:ffloat}
  Indicate number of bytes:
  \verbatimsnippet{fstorage}
\end{slide}

\Level 1 {Precision}
\label{sec:fprecision}

In Fortran you can actually ask for a type with specified precision.
\begin{itemize}
\item For integers you can specify the number of decimal digits with
  \indextermttdef{selected_int_kind}\n{($n$)}.
\item For floating point numbers can specify the number of
  significant digits, and optionally the decimal exponent range with
  \indextermttdef{selected_real_kind}\n{($p$[,$r$])}.
  of significant digits.
\item To query the type of a variable, use the function
  \indextermttdef{kind}, which returns an integer.
\end{itemize}

Likewise, you can specify the precision of a constant.
Writing \n{3.14} will usually be a single precision
real. 

\begin{block}{Single/double precision constants}
  \label{sl:fsingledouble}
  \snippetwithoutput{f0const}{basicf}{e0}
\end{block}

You can query how many bytes a data type takes with
\indextermttdef{kind}.

\begin{block}{Numerical precision}
  \label{sl:fprecision48}
  Number of bytes determines numerical precision:
  \begin{itemize}
  \item Computations in 4-byte have relative error $\approx 10^{-6}$
  \item Computations in 8-byte have relative error $\approx 10^{-15}$
  \end{itemize}
  Also different exponent range: max $10^{\pm 50}$ and $10^{\pm 300}$ respectively.
\end{block}

\begin{block}{Storage size}
  F08: \indextermtt{storage_size} reports number of bits.

  F95: \indextermtt{bit_size} works on integers only.

  \indextermtt{c_sizeof} reports number of bytes, requires
  \indextermtt{iso_c_binding} module.
\end{block}

Force a constant to be \n{real(8)}:
\begin{block}{Double precision constants}
  \label{sl:fdouble}
\begin{lstlisting}
real(8) :: x,y
x = 3.14d0
y = 6.022e-23
\end{lstlisting}
  \begin{itemize}
  \item Use a compiler flag such as \n{-r8} to force all reals to be 8-byte.
  \item Write \n{3.14d0}
  \item \verb+x = real(3.14, kind=8)+
  \end{itemize}
\end{block}

\begin{block}{Complex}
  \label{sl:fcomplex}
  Complex constants are written as a pair of reals in parentheses.\\
  There are some basic operations.
  %
  \snippetwithoutput{fcomplex}{basicf}{complex}
\end{block}

\Level 1 {Initialization}

Variables can be initialized in their declaration:
\begin{lstlisting}
integer :: i=2
real(4) :: x = 1.5
\end{lstlisting}

That this is done at compile time, leading to a common error:
\begin{lstlisting}
subroutine foo()
  implicit none
  integer :: i=2
  print *,i
  i = 3
end subroutine foo
\end{lstlisting}
On the first subroutine call \n{i} is printed with its initialized
value, but on the second call this initialization is not repeated, and
the previous value of \n{3} is remembered.

\Level 0 {Input/Output, or I/O as we say}
\label{sec:fio}

\begin{block}{Simple I/O}
  \label{sl:frw}
  \begin{itemize}
  \item Input: 
\begin{lstlisting}
READ *,n
\end{lstlisting}
\item Output:
\begin{lstlisting}
PRINT *,n
\end{lstlisting}
  \end{itemize}
  There is also \n{WRITE}.

  The `star' indicates that default formatting is used.\\
  Other syntax for read/write with files and formats.
\end{block}

\Level 0 {Expressions}
\label{sec:fexpr}

\begin{block}{Arithmetic expressions}
  \label{sl:farith}
  \begin{itemize}
  \item Pretty much as in C++
  \item Exception: \n{r**a} for power $r^a$.
  \item Modulus is a function: \lstinline{MOD(7,3)}.
  \end{itemize}
\end{block}

\begin{block}{Boolean expressions}
  \label{sl:fbool}
  \begin{itemize}
  \item 
    Long form:\\
    \n{.and. .not. .or.}\\
    \n{.lt. .le. .eq. .ne. .ge. .gt.}\\
    \n{.true. .false.}
  \item Short form:\\
    \verb+< <= == /= > >=+
  \end{itemize}
\end{block}

\begin{block}{Conversion and casting}
  Conversion is done through functions.
  \begin{itemize}
  \item \n{INT}: truncation; \n{NINT} rounding
  \item \n{REAL}, \n{FLOAT}, \n{SNGL}, \n{DBLE}
  \item \n{CMPLX}, \n{CONJG}, \n{AIMIG}
  \end{itemize}
\url{http://userweb.eng.gla.ac.uk/peter.smart/com/com/f77-conv.htm}
\end{block}

\begin{block}{Complex}
  Complex numbers exist
\end{block}

\begin{block}{Strings}
  Strings are delimited by single or double quotes.

  For more, see chapter~\ref{ch:stringf}.
\end{block}

\Level 0 {Review questions}

\begin{exercise}
  What is the output for this fragment, assuming \n{i,j} are integers?
  %
  \verbatimsnippet{idiv}
\end{exercise}

\begin{exercise}
  What is the output for this fragment, assuming \n{i,j} are integers?
  %
  \verbatimsnippet{fdiv}
\end{exercise}

\begin{exercise}
  \label{ex:f-elt-rev1}
  In declarations
\begin{lstlisting}
real(4) :: x
real(8) :: y
\end{lstlisting}
what do the \n{4} and \n{8} stand for?

What is the practical implication of using the one or the other?
\end{exercise}

\begin{exercise}
  \label{ex:read-writet3np1}
  Write a program that :
  \begin{itemize}
  \item displays the message \n{Type a number},
  \item accepts an integer number from you (use~\n{Read}),
  \item makes another variable that is three times that integer plus one,
  \item and then prints out the second variable.
  \end{itemize}
\end{exercise}

\begin{exercise}
  \label{ex:f-elt-rev2}
  In the following code, if \n{value} is nonzero, what do expect about
  the output?
  %
  \verbatimsnippet{d0}
\end{exercise}
