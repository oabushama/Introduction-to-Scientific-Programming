% -*- latex -*-
%%%%%%%%%%%%%%%%%%%%%%%%%%%%%%%%%%%%%%%%%%%%%%%%%%%%%%%%%%%%%%%%
%%%%
%%%% This TeX file is part of the course
%%%% Introduction to Scientific Programming in C++/Fortran2003
%%%% copyright 2017/8 Victor Eijkhout eijkhout@tacc.utexas.edu
%%%%
%%%% scope.tex : scope issues
%%%%
%%%%%%%%%%%%%%%%%%%%%%%%%%%%%%%%%%%%%%%%%%%%%%%%%%%%%%%%%%%%%%%%

\Level 0 {Scope rules}

The concept of \indextermdef{scope} answers the question `when is the
binding between a name (read: variable) and the internal entity valid'.

\Level 1 {Lexical scope}

C++, like Fortran and most other modern languages, uses
\indextermsub{lexical}{scope} rule. This means that you can textually
determine what a variable name refers to.
\begin{verbatim}
int main() {
  int i;
  if ( something ) {
    int j;
    // code with i and j
  }
  int k;
  // code with i and k
}
\end{verbatim}
\begin{itemize}
\item The lexical scope of the variables \n{i,k} is the main program
  including any blocks in it, such as the conditional, from the point
  of definition onward. You can think that the variable in memory is
  created when the program execution reaches that statement, and after
  that it can be refered to by that name.
\item The lexical scope of \n{j} is limited to the true branch of the
  conditional. The integer quantity is only created if the true branch
  is executed, and you can refer to it during that execution. After
  execution leaves the conditional, the name ceases to exist, and so
  does the integer in memory.
\item In general you can say that any
  %
  \emph{use}\index{definition vs use}
  of a name has be in the lexical scope of that variable, and after
  its \emph{definition}.
\end{itemize}

\begin{slide}{Lexical scope}
  \label{sl:lexical}
  Visibility of variables
\begin{verbatim}
int main() {
  int i;
  if ( something ) {
    int j;
    // code with i and j
  }
  int k;
  // code with i and k
}
\end{verbatim}  
\end{slide}

\Level 1 {Shadowing}

Scope can be limited by an occurrence of a variable by the same name:
%
\snippetwithoutput{shadowtrue}{basic}{shadowtrue}
%
The first variable \n{i} has lexical scope of the whole program, minus
the two conditionals. While its
\emph{lifetime}\index{variable!lifetime} is the whole program, it is
unreachable in places because it is
\emph{shadowed}\index{variable!shadowing} by the variables \n{i} in the conditionals.

\begin{slide}{Shadowing}
  \label{sl:scope-shadow}
\begin{verbatim}
int main() {
  int i = 3;
  if ( something ) {
    int i = 5;
  }
  cout << i << endl; // gives 3
  if ( something ) {
    float i = 1.2;
  }
  cout << i << endl; // again 3
}
\end{verbatim}
Variable \n{i} is shadowed: invisible for a while.\\
After the lifetime of the shadowing variable, its value is unchanged
from before.
\end{slide}

This is independent of dynamic / runtime behaviour!

\begin{exercise}
  \label{ex:shadowfalse}
  What is the output of this code?
  %
  \verbatimsnippet{shadowfalse}
\end{exercise}

\begin{exercise}
  \label{ex:loopinitvar}
  What is the output of this code?
\begin{verbatim}
for (int i=0; i<2; i++) {
  int j; cout << j << endl;
  j = 2; cout << j << endl;
}
\end{verbatim}
\end{exercise}

\Level 1 {Lifetime versus reachability}

The use of functions introduces a complication in the lexical scope story:
a variable can be present in memory, but may not be textually accessible:
\begin{verbatim}
void f() {
  ...
}
int main() {
  int i;
  f();
  cout << i;
}
\end{verbatim}
During the execution of \n{f}, the variable \n{i} is present in
memory, and it is unaltered after the execution of~\n{f},
but it is not accessible.

\begin{slide}{Life time vs reachability}
  \label{sl:scope-lifetime}
  Even without shadowing, a variable can exist but be unreachable.
\begin{verbatim}
void f() {
  ...
}
int main() {
  int i;
  f();
  cout << i;
}
\end{verbatim}
\end{slide}

A special case of this is recursion:
\begin{verbatim}
void f(int i) {
  int j = i;
  if (i<100)
    f(i+1);
}
\end{verbatim}
Now each incarnation of \n{f} has a local variable~\n{i}; during a
recursive call the outer~\n{i} is still alive, but it is inaccessible.

\Level 1 {Scope subtleties}

\Level 2 {Mutual recursion}

If you have two functions \n{f,g} that call each other,
\begin{verbatim}
int f(int i) { return g(i-1); }
int g(int i) { return f(i+1); }
\end{verbatim}
you need a
%
\emph{forward declaration}\index{forward declaration!of functions}
%
\begin{verbatim}
int g(int);
int f(int i) { return g(i-1); }
int g(int i) { return f(i+1); }
\end{verbatim}
since the use of name~\n{g} has to come after its declaration.

There is also forward declaration of
%
\emph{classes}\index{forward declaration!of classes}.
\begin{verbatim}
class B;
class A {
private:
  shared_ptr<B> myB;
};
class B {
private:
  int myint;
}
\end{verbatim}

\Level 2 {Closures}

The use of 
%
lambdas\index{lambda|see{closure}}
or
%
\emph{closures}\index{closure} (section~\ref{sec:lambda}) comes with
another exception to general scope rules. Read about `capture'.

\Level 0 {Static variables}
\label{sec:static-scope}

Variables in a function have \indextermsub{lexical}{scope} limited to
that function. Normally they also have \indextermsub{dynamic}{scope}
limited to the function execution: after the function finishes they
completely disappear. (Class objects have their
%
\emph{destructor}\index{destructor!at end of scope}
called.)

There is an exception: a \indextermsub{static}{variable} persists
between function invocations.
\begin{verbatim}
void fun() {
  static int remember;
}
\end{verbatim}
For example
\begin{verbatim}
int onemore() {
  static int remember++; return remember;
}
int main() {
  for ( ... )
    cout << onemore() << end;
  return 0;
}
\end{verbatim}
gives a stream of integers.
\begin{exercise}
  The static variable in the \n{onemore} function is never
  initialized. Can you find a mechanism for doing so?
  Can you do it with a default argument to the function?
\end{exercise}

\Level 0 {Scope and memory}

The notion of scope is connected to the fact that variables correspond
to objects in memory. Memory is only reserved for an entity during the
dynamic scope of the entity. This story is clear in simple cases:
\begin{verbatim}
int main() {
  // memory reserved for 'i'
  if (true) {
    int i; // now reserving memory for integer i
    ... code ...
  }
  // memory for `i' is released
}
\end{verbatim}
Recursive functions offer a complication:
\begin{verbatim}
int f(int i) {
  int itmp;
  ... code with `itmp' ...
  if (something)
    return f(i-1);
  else return 1;
}
\end{verbatim}
Now each recursive call of \n{f} reserves space for its own
incarnation of~\n{itmp}.

In both of the above cases the variables are said to be on the
\indexterm{stack}: each next level of scope nesting or recursive
function invocation creates new memory space, and that space is
released before the enclosing level is released.

Objects behave like variables as described above: their memory is
released when they go out of scope. However, in addition, a
\indexterm{destructor} is called on the object, and on all its
contained objects:
%
\snippetwithoutput{destructor}{object}{destructor}

\Level 0 {Review questions}

\begin{exercise}
  \label{ex:cpp-scope1}
  Is this a valid program?
\begin{verbatim}
void f() { i = 1; }
int main() {
  int i=2;
  f();
  return 0;
}
\end{verbatim}
If yes, what does it do; if no, why not?
\end{exercise}

\begin{exercise}
  \label{ex:cpp-scope2}
  What is the output of:
\begin{verbatim}
#include <iostream>
using std::cout;
using std::endl;
int main() {
  int i=5;
  if (true) { i = 6; }
  cout << i << endl;
  return 0;
}
\end{verbatim}
\end{exercise}

\begin{exercise}
  \label{ex:cpp-scope3}
  What is the output of:
\begin{verbatim}
#include <iostream>
using std::cout;
using std::endl;
int main() {
  int i=5;
  if (true) { int i = 6; }
  cout << i << endl;
  return 0;
}
\end{verbatim}
\end{exercise}

\begin{exercise}
  \label{ex:cpp-scope4}
  What is the output of:
\begin{verbatim}
#include <iostream>
using std::cout;
using std::endl;
int main() {
  int i=2;
  i += /* 5;
  i += */ 6;
  cout << i << endl;
  return 0;
}
\end{verbatim}
\end{exercise}

\endinput

Global variables, local variables.

When you defined a function for primality testing, you placed it
outside the main program, and the main program was able to use
it. There are other things than functions that can be defined outside
the main program, such as \indextermsub{global}{variables}.

Here is a program that uses a global variable:
\begin{verbatim}
int i=5;
int main() {
  i = i+3;
  cout << i << endl;
  return 0;
}
\end{verbatim}

