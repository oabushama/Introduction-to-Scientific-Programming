% -*- latex -*-
%%%%%%%%%%%%%%%%%%%%%%%%%%%%%%%%%%%%%%%%%%%%%%%%%%%%%%%%%%%%%%%%
%%%%
%%%% This TeX file is part of the course
%%%% Introduction to Scientific Programming in C++/Fortran2003
%%%% copyright 2017-9 Victor Eijkhout eijkhout@tacc.utexas.edu
%%%%
%%%% loopf.tex : looping constructs in Fortran
%%%%
%%%%%%%%%%%%%%%%%%%%%%%%%%%%%%%%%%%%%%%%%%%%%%%%%%%%%%%%%%%%%%%%

\Level 0 {Loop types}

Fortran has the usual indexed and `while' loops. There are variants of the
basic loop, and both use the \indextermfort{do} keyword. The simplest loop has
\begin{itemize}
\item a loop variable, which needs to be declared;
\item a lower bound and upper bound.
\end{itemize}
We'll see more types of loops below.

\begin{block}{Indexed Do loops}
  \label{sl:doloop}
\begin{lstlisting}
integer :: i

do i=1,10
  ! code with i
end do
\end{lstlisting}

You can include a step size (which can be negative) as a third parameter:
\begin{lstlisting}
do i=1,10,3
  ! code with i
end do
\end{lstlisting}
\end{block}

\begin{block}{While loop}
  \label{sl:whilef}
  The while loop has a pre-test:
\begin{lstlisting}
do while (i<1000)
  print *,i
  i = i*2
end do
\end{lstlisting}
\end{block}

You can label loops, which improves readability, but see also below.
\begin{lstlisting}
outer: do i=1,10
    inner: do j=1,10
    end do inner
end do outer
\end{lstlisting}
The label needs to be on the same line as the \n{do}, and if you use a
label, you need to mention it on the \n{end do} line.

\begin{f77note}
  Do not use label-terminated loops. Do not use non-integer loop variables.
\end{f77note}

\Level 0 {Interruptions of the control flow}

For indeterminate looping, you can use the \n{while} test,
or leave out the loop parameter altogether.
In that case you need the \indextermtt{exit} statement to stop the iteration.

\begin{block}{Exit and cycle}
  \label{sl:loopexit}
  Loop without counter or while test:
\begin{lstlisting}
do
  call random_number(x)
  if (x>.9) exit
  print *,"Nine out of ten exes agree"
end do
\end{lstlisting}

Skip rest of iteration:
\begin{lstlisting}
do i=1,100
  if (isprime(i)) cycle
  ! do something with non-prime
end do
\end{lstlisting}
\end{block}

Cycle and exit can apply to multiple levels, if the do-statements are
labeled.

\begin{lstlisting}
outer : do i = 1,10
inner : do j = 1,10
    if (i+j>15) exit outer
    if (i==j) cycle inner
end do inner
end do outer
\end{lstlisting}

\Level 0 {Implied do-loops}
\label{sec:f-impdo}

There are do loops that you can write in a single line by an
expression and a loop header. In effect, such an
\indextermsub{implied}{do loop} becomes the sum of the indexed
expressions. This is useful
for I/O. For instance, iterate a simple expression:

\begin{block}{Implied do loops}
  \label{sl:implieddo}
\begin{lstlisting}
print *,(2*i,i=1,20)
\end{lstlisting}
You can iterate multiple expressions:
\begin{lstlisting}
print *,(2*i,2*i+1,i=1,20)
\end{lstlisting}
These loops can be nested:
\begin{lstlisting}
print *,( (i*j,i=1,20), j=1,20 )
\end{lstlisting}
\end{block}

This construct is especially useful for printing arrays.

\begin{exercise}
  \label{ex:impl-triangle}
  Use the implied do-loop mechanism to print a triangle:
\begin{lstlisting}
  1
  2 2
  3 3 3
  4 4 4 4
\end{lstlisting}
  up to a number that is input.
\end{exercise}

\Level 0 {Review questions}

\begin{exercise}
  \label{ex:floop-inf}
  What is the output of:
\begin{lstlisting}
do i=1,11,3
  print *,i
end do
\end{lstlisting}
What is the output of:
\begin{lstlisting}
do i=1,3,11
  print *,i
end do
\end{lstlisting}
\end{exercise}
