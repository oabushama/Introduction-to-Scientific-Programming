% -*- latex -*-
%%%%%%%%%%%%%%%%%%%%%%%%%%%%%%%%%%%%%%%%%%%%%%%%%%%%%%%%%%%%%%%%
%%%%
%%%% This TeX file is part of the course
%%%% Introduction to Scientific Programming in C++/Fortran2003
%%%% copyright 2017-9 Victor Eijkhout eijkhout@tacc.utexas.edu
%%%%
%%%% lambda.tex : closures
%%%%
%%%%%%%%%%%%%%%%%%%%%%%%%%%%%%%%%%%%%%%%%%%%%%%%%%%%%%%%%%%%%%%%

The mechanism of \indextermbus{lambda}{expression}s makes
dynamic definition of functions possible.

\begin{block}{Lambda expressions}
  \label{sl:lambda-syntax}
\begin{verbatim}
[capture] ( inputs ) -> outtype { definition };
\end{verbatim}
Example:
\verbatimsnippet{lambdaexp}
Store lambda in a variable:
\verbatimsnippet{lambdavar}
\end{block}

A non-trivial use of lambdas uses the \indexterm{capture} to fix one argument of a
function.
Let's say we want a function that computes exponentials for some fixed
exponent value. We take the 
\indextermtt{cmath} function
\begin{verbatim}
pow( x,exponent );
\end{verbatim}
and fix the exponent:
%
\verbatimsnippet{lambdacapt}
%
Now \n{powerfunction} is a function of one argument, which computes
that argument to a fixed power.

\begin{slide}{Capture parameter}
  \label{sl:lambda-capture}
  Capture value and reduce number of arguments:
  %
  \verbatimsnippet{lambdacapt}
  %
  Now \n{powerfunction} is a function of one argument, which computes
  that argument to a fixed power.
\end{slide}

\Level 1 {Lambda members of classes}

Storing a lambda in a class is hard because it has unique
type. Solution: use \lstinline{std::}\indextermtt{function}.

\begin{block}{Lambda in object}
  \label{sl:lambda-class}
  %
  \verbatimsnippet{lambdaclass}
\end{block}

\begin{block}{Illustration}
  \label{sl:lambda-classed}
  \snippetwithoutput{lambdaclassed}{func}{lambdafun}
\end{block}

\begin{exercise}
  \label{ex:newtonlambda}
  Refer to~\ref{sl:newton-root} for background, and note that finding
  $x$ such that $f(x)=a$ is equivalent to applying Newton to
  $f(x)-\nobreak a$.

  Implement a class \n{valuefinder} and its \n{double find(double)} method.
  %
  \verbatimsnippet{zeroclass}
  %
  used as
  %
  \verbatimsnippet{newtonroot}
\end{exercise}

\begin{exercise}
  \label{ex:morenewtonlambda}
  Can you write a derived class \n{rootfinder} used as
  %
  \verbatimsnippet{newtonlambda1}
\end{exercise}

\Level 1 {Lambda in algorithm}

The \indextermtt{algorithm} header contains a number of functions that
naturally use lambdas. For instance, \indextermtt{any_of} can test
whether any element of a \lstinline{vector} satisfies a condition.
You can then use a lambda to specify the \lstinline{bool} function
that tests the condition.

\Level 1 {Lambda variables by reference}

Normally, captured variables are copied by value. You can capture them
by reference by prefixing an ampersand to them
\begin{lstlisting}
auto maybeinc =
  [&count] (int i) mutable {
      if (f(i)) count++; }
\end{lstlisting}
The lambda expression also needs to be marked
\indextermttdef{mutable}\index{lambda!make mutable}
because by default it is a
\n{const}\index{lamda!const by default}.

\snippetwithoutput[printeach]{counteach}{stl}{counteach}

Lambdas without captures can be converted to a function pointer.
