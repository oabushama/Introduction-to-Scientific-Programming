% -*- latex -*-
%%%%%%%%%%%%%%%%%%%%%%%%%%%%%%%%%%%%%%%%%%%%%%%%%%%%%%%%%%%%%%%%
%%%%
%%%% This TeX file is part of the course
%%%% Introduction to Scientific Programming in C++/Fortran2003
%%%% copyright 2017/8 Victor Eijkhout eijkhout@tacc.utexas.edu
%%%%
%%%% dna.tex : exercises about dna sequencing
%%%%
%%%%%%%%%%%%%%%%%%%%%%%%%%%%%%%%%%%%%%%%%%%%%%%%%%%%%%%%%%%%%%%%

In this set of exercises you will write mechanisms for DNA sequencing.

\Level 0 {Basic functions}

Refer to section~\ref{sec:tuple}.

First we set up some basic mechanisms.

\begin{exercise}
  \label{ex:basecomp}
  There are four bases, \n{A,C,G,T}, and each has a complement:
  $\n{A}\leftrightarrow\n{T}$, $\n{C}\leftrightarrow\n{G}$.
  Implement this through a \n{map}, and write a function
\begin{verbatim}
char BaseComplement(char);
\end{verbatim}
\end{exercise}

\begin{exercise}
  \label{ex:basecount}
  Write code to
  read a \indexterm{Fasta} file into a \n{string}.
  The first line, starting with~\n{>}, is a comment; all other lines
  should be concatenated into a single string denoting the genome.
  
  Read the virus genome in the file \n{lambda_virus.fa}.

  Count the four bases in the genome two different ways.  First use a
  \n{map}.  Time how long this takes. Then do the same thing using an
  array of length four, and a conditional statement.

  Bonus: try to come up with a faster way of counting. Use a vector of
  length~4, and find a way of computing the index directly from the
  letters \n{A,C,G,T}. Hint: research ascii codes and possibly bit operations.
\end{exercise}

A~`read' is a short fragment of DNA, that we want to match against a
genome. We first explore a naive matching algorithm: for each location
in the genome, see if the read matches up.
\begin{verbatim}
ATACTGACCAAGAACGTGATTACTTCATGCAGCGTTACCAT
  ACCAAGAACGTG
    ^ mismatch

ATACTGACCAAGAACGTGATTACTTCATGCAGCGTTACCAT
      ACCAAGAACGTG
      total match
\end{verbatim}

\begin{exercise}
  \label{ex:readmatch}
  Code up the naive algorithm for matching a read. Test it on fake
  reads obtained by copying a substring from the
  genome. Use the genome in \n{phix.fa}.

  Now read the \indexterm{Fastq} file
  \n{ERR266411_1.first1000.fastq}. Fastq files contains groups of four
  lines: the second line in each group contains the reads.
  How many of these reads are matched to the genome?

  Reads are not necessarily a perfect match; in fact, each fourth line
  in the fastq file gives an indication of the `quality' of the
  corresponding read. How many matches do you get if you take a
  substring of the first 30 or so characters of each read?
\end{exercise}
