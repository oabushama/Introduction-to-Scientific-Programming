% -*- latex -*-
%%%%%%%%%%%%%%%%%%%%%%%%%%%%%%%%%%%%%%%%%%%%%%%%%%%%%%%%%%%%%%%%
%%%%
%%%% This TeX file is part of the course
%%%% Introduction to Scientific Programming in C++/Fortran2003
%%%% copyright 2017-9 Victor Eijkhout eijkhout@tacc.utexas.edu
%%%%
%%%% obscure.tex : other stuff
%%%%
%%%%%%%%%%%%%%%%%%%%%%%%%%%%%%%%%%%%%%%%%%%%%%%%%%%%%%%%%%%%%%%%

\Level 0 {Interfaces}
\label{sec:finterface}

If you want to use a procedure in your main program, the compiler
needs to know the signature of the procedure: how many arguments, of
what type, and with what intent. You have seen how the
\indextermfort{contains} clause can be used for this purpose if the
procedure resides in the same file as the main program.

If the procedure is in a separate file, the compiler does not see
definition and usage in one go. To allowed the compiler to do checking
on proper usage, we can use an \indextermfortdef{interface} block. This
is placed at the calling site, declaring the signature of the
procedure.

\begin{multicols}{2}
\textbf{Main program:}
\verbatimsnippet{interfacemain}
\vfill\columnbreak
\textbf{Procedure:}
\verbatimsnippet{interfunc}
\end{multicols}

The \indextermfort{interface} block is not required (an older
\indextermfort{external} mechanism exists for functions), but is
recommended.
It is required if the function takes function arguments.

\Level 1 {Polymorphism}

The \indextermfort{interface} block can be used to define a generic
function:
\begin{lstlisting}
interface f
function f1( ..... )
function f2( ..... )
end interface f
\end{lstlisting}
where \n{f1},\n{f2} are functions that can be distinguished by their
argument types. The generic function~\n{f} then becomes either \n{f1}
or \n{f2} depending on what type of argument it is called with.

\Level 0 {Random numbers}

In this section we briefly discuss the Fortran \emph{random number
  generator}\index{random number!generator!Fortran}.
The basic mechanism is through the library subroutine
\indextermfort{random_number}, which has a single argument of type
\lstinline{REAL} with \lstinline{INTENT(OUT)}:
\begin{lstlisting}
real(4) :: randomfraction
call random_number(randomfraction)
\end{lstlisting}
The result is a random number from the uniform distribution
on~$\left[0,1\right)$.
  
Setting the \indextermbus{random}{seed} is slightly convoluted. The
amount of storage needed to store the seed can be processor and
implementation-dependent, so the routine \indextermfort{random_seed}
can have three types of named argument, exactly one of which can be
specified at any one time. The keyword can be:
\begin{itemize}
\item \lstinline{SIZE} for querying the size of the seed;
\item \lstinline{PUT} for setting the seed; and
\item \lstinline{GET} for querying the seed.
\end{itemize}
A typical fragment for setting the seed would be:
\begin{lstlisting}
integer :: seedsize
integer,dimension(:),allocatable :: seed

call random_seed(size=seedsize)
allocate(seed(seedsize))
seed(:) = ! your integer seed here
call random_seed(put=seed)
\end{lstlisting}

\Level 0 {Timing}

Timing is done with the \indextermfort{system_clock} routine.
\begin{itemize}
\item This call gives an integer, counting clock ticks.
\item To convert to seconds, it can also tell you how many ticks per
  second it has: its \indextermbus{timer}{resolution}.
\end{itemize}

\begin{lstlisting}
  integer :: clockrate,clock_start,clock_end
  call system_clock(count_rate=clockrate)
  print *,"Ticks per second:",clockrate

  call system_clock(clock_start)
  ! code
  call system_clock(clock_end)
  print *,"Time:",(clock_end-clock_start)/REAL(clockrate)
\end{lstlisting}
