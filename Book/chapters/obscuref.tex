% -*- latex -*-
%%%%%%%%%%%%%%%%%%%%%%%%%%%%%%%%%%%%%%%%%%%%%%%%%%%%%%%%%%%%%%%%
%%%%
%%%% This TeX file is part of the course
%%%% Introduction to Scientific Programming in C++/Fortran2003
%%%% copyright 2017-9 Victor Eijkhout eijkhout@tacc.utexas.edu
%%%%
%%%% obscure.tex : other stuff
%%%%
%%%%%%%%%%%%%%%%%%%%%%%%%%%%%%%%%%%%%%%%%%%%%%%%%%%%%%%%%%%%%%%%

\Level 0 {Random numbers}

In this section we briefly discuss the Fortran \emph{random number
  generator}\index{random number!generator!Fortran}.
The basic mechanism is through the library subroutine
\indextermfort{random_number}, which has a single argument of type
\lstinline{REAL} with \lstinline{INTENT(OUT)}:
\begin{lstlisting}
real(4) :: randomfraction
call random_number(randomfraction)
\end{lstlisting}
The result is a random number from the uniform distribution
on~$\left[0,1\right)$.
  
Setting the \indextermbus{random}{seed} is slightly convoluted. The
amount of storage needed to store the seed can be processor and
implementation-dependent, so the routine \indextermfort{random_seed}
can have three types of named argument, exactly one of which can be
specified at any one time. The keyword can be:
\begin{itemize}
\item \lstinline{SIZE} for querying the size of the seed;
\item \lstinline{PUT} for setting the seed; and
\item \lstinline{GET} for querying the seed.
\end{itemize}
A typical fragment for setting the seed would be:
\begin{lstlisting}
integer :: seedsize
integer,dimension(:),allocatable :: seed

call random_seed(size=seedsize)
allocate(seed(seedsize))
seed(:) = ! your integer seed here
call random_seed(put=seed)
\end{lstlisting}

\Level 0 {Timing}

Timing is done with the \indextermfort{system_clock} routine.
\begin{itemize}
\item This call gives an integer, counting clock ticks.
\item To convert to seconds, it can also tell you how many ticks per
  second it has: its \indextermbus{timer}{resolution}.
\end{itemize}

\begin{lstlisting}
  integer :: clockrate,clock_start,clock_end
  call system_clock(count_rate=clockrate)
  print *,"Ticks per second:",clockrate

  call system_clock(clock_start)
  ! code
  call system_clock(clock_end)
  print *,"Time:",(clock_end-clock_start)/REAL(clockrate)
\end{lstlisting}
