
\begin{exercise}
  Make a default constructor for the point class:
\begin{verbatim}
Point() { /* default code */ }
\end{verbatim}
  which you can use as:
\begin{verbatim}
Point p;
\end{verbatim}
\end{exercise}

\begin{exercise}
  Advanced. Can you make a \n{Point} class that can accomodate any
  number of space dimensions? Hint: use a \n{vector};
  section~\ref{sec:stdvector}. Can you make a constructor where you do
  not specify the space dimension explicitly?
\end{exercise}

\Level 0 {Using one class in another}
\label{sec:FuncHasPoint}
\label{sec:poly-rectangle}

\prerequisite{\ref{sec:hasa}}

\begin{exercise}
  \label{ex:geom:line}
  Make a class \n{LinearFunction} with a constructor:
\begin{verbatim}
LinearFunction( Point input_p1,Point input_p2 );
\end{verbatim}
  and a function
\begin{verbatim}
float evaluate_at( float x );
\end{verbatim}
  which you can use as:
\begin{verbatim}
LinearFunction line(p1,p2);
cout << "Value at 4.0: " << line.evaluate_at(4.0) << endl;
\end{verbatim}
\end{exercise}

\begin{exercise}
  \label{ex:geom:line2}
  Make a class \n{LinearFunction} with two constructors:
\begin{verbatim}
LinearFunction( Point input_p2 );
LinearFunction( Point input_p1,Point input_p2 );
\end{verbatim}
where the first stands for a line through the origin.\\
Implement again the \n{evaluate} function so that
\begin{verbatim}
LinearFunction line(p1,p2);
cout << "Value at 4.0: " << line.evaluate_at(4.0) << endl;
\end{verbatim}
\end{exercise}

Suppose you want to write a \n{Rectangle} class, which could have methods such as
\n{float Rectangle::area()} or \n{bool Rectangle::contains(Point)}.
Since rectangle has four corners, you could store four \n{Point}
objects in each \n{Rectangle} object. However, there is redundancy
there: you only need three points to infer the fourth. Let's consider
the case of a rectangle with sides that are horizontal and vertical;
then you need only two points.

\begin{block}{Axi-parallel rectangle class}
  \label{ex:geom:rect}
  Intended API:
\begin{verbatim}
float Rectangle::area();
\end{verbatim}
It would be convenient to store width and height; for 
\begin{verbatim}
bool Rectangle::contains(Point);  
\end{verbatim}
it would be convenient to store bottomleft/topright points.
\end{block}

\begin{exercise}
  \label{ex:geom:rect2}
  Make a class \n{Rectangle} (sides parallel to axes) with two constructors:
\begin{verbatim}
Rectangle(Point bl,Point tr);
Rectangle(Point bl,float w,float h);
\end{verbatim}
  and functions
\begin{verbatim}
float area(); float width(); float height();
\end{verbatim}
  Let the \n{Rectangle} object store two \n{Point} objects.

  Then rewrite your exercise so that the \n{Rectangle} stores only one
  point (say, lower left), plus the width and height.
\end{exercise}

The previous exercise illustrates an important point: for well
designed classes you can change the implementation (for instance motivated
by efficiency) while the program that uses the class does not change.

\Level 0 {Is-a relationship}

\prerequisite{\ref{sec:inheritance}}

\begin{exercise}
  \label{ex:geom:square}
  Take your code where a \n{Rectangle} was defined from one point,
  width, and height.

  Make a class \n{Square} that inherits from \n{Rectangle}. It should
  have the function \n{area} defined, inherited from \n{Rectangle}.

  First ask yourself: what should the constructor of a \n{Square} look like?
\end{exercise}

\begin{exercise}
  \label{ex:geom:line3}
  Revisit the \n{LinearFunction} class.
  Add methods \n{slope} and \n{intercept}.

  Now generalize \n{LinearFunction} to \n{StraightLine}
  class. These two are almost the same except for vertical lines.
  The \n{slope} and \n{intercept} do not apply to vertical lines, so
  design \n{StraightLine} so that it stores the defining points
  internally. Let \n{LinearFunction} inherit.
\end{exercise}

