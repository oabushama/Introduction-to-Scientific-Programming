% -*- latex -*-
%%%%%%%%%%%%%%%%%%%%%%%%%%%%%%%%%%%%%%%%%%%%%%%%%%%%%%%%%%%%%%%%
%%%%
%%%% This TeX file is part of the course
%%%% Introduction to Scientific Programming in C++/Fortran2003
%%%% copyright 2017/8 Victor Eijkhout eijkhout@tacc.utexas.edu
%%%%
%%%% obscure.tex : other stuff
%%%%
%%%%%%%%%%%%%%%%%%%%%%%%%%%%%%%%%%%%%%%%%%%%%%%%%%%%%%%%%%%%%%%%

\Level 0 {Const}

\Level 1 {Const arguments}
\label{sec:constparam}

Function arguments marked \indextermtt{const} can not be altered by
the function code. The following segment gives a compilation error:
%
\verbatimsnippet{constchange}

The use of const arguments is one way of protecting you against yourself.
If an argument is conceptually supposed to stay constant, the compiler
will catch it if you mistakenly try to change it.

\Level 1 {Const references}

A more sophisticated use of const is the
\indextermbus{const}{reference}:
\begin{verbatim}
void f( const int &i ) { .... }
\end{verbatim}
This may look strange. After all, references, and the
pass-by-reference mechanism, were introduced in
section~\ref{sec:passing} to return changed values to the calling
environment. The const keyword negates that possibility of changing
the parameter.

But there is a second reason for using references. Parameters are
passed by value, which means that they are copied, and that includes
big objects such as \n{std::vector}. Using a reference to pass a
vector is much less costly in both time and space, but then there is the
possibility of changes to the vector propagating back to the calling
environment.

Marking a vector argument as const allows
\indextermbus{compiler}{optimization}. Assume the function~\n{f} as
above, used like this:
\begin{verbatim}
std::vector<double> v(n);
for ( .... ) {
  f(v);
  y = v[0];
  ... y ... // code that uses y
}
\end{verbatim}
Since the function call does not alter the vector, \n{y}~is invariant
in the loop iterations, and the compiler changes this code internally to
\begin{verbatim}
std::vector<double> v(n);
int saved_y = v[0];
for ( .... ) {
  f(v);
  ... saved_y ... // code that uses y
}
\end{verbatim}

Consider a class that has methods that return an internal member by
reference, once as const reference and once not:
%
\snippetwithoutput{constref}{const}{constref}

We can make visible the difference between pass by value and pass by
const-reference if we define a class where the
\indextermsub{copy}{constructor} explicitly reports itself:
%
\verbatimsnippet{classwithcopy}
  
Now if we define two functions, with the two parameter passing
mechanisms, we see that passing by value invokes the copy constructor,
and passing by const reference does not:
%
\snippetwithoutput{constcopy}{const}{constcopy}

\Level 1 {Const methods}

We can distinguish two types of methods: those that alter internal
data members of the object, and those that don't. The ones that don't
can be marked \indextermtt{const}:
\begin{verbatim}
class Things {
private:
  int i;
public:
  int get() const { return i; }
  int inc() { return i++; }
}
\end{verbatim}
While this is in no way required, it can be helpful in two ways:
\begin{itemize}
\item It will catch mismatches between the prototype and definition of
  the method. For instance,
\begin{verbatim}
class Things {
private:
  int var;
public:
  f(int &ivar,int c) const {
    var += c; // typo: should be `ivar'
  }
}
\end{verbatim}
Here, the use of \n{var} was a typo, should have been \n{ivar}. Since
the method is marked \n{const}, the compiler will generate an error.
\item It allows the compiler to optimize your code. For instance:
\begin{verbatim}
class Things {
public:
  int f() const { /* ... */ };
  int g() const { /* ... */ };
}
...
Things t;
int x,y,z;
x = t.f();
y = t.g();
z = t.f();
\end{verbatim}
Since the methods did not alter the object, the compiler can conclude
that \n{x,z} are the same, and skip the calculation for~\n{z}.
\end{itemize}

\Level 0 {Auto}

This is not actually obscure, but it intersects many other topics, so
we put it here for now.

\Level 1 {Declarations}

Sometimes the type of a variable is obvious:
\begin{verbatim}
std::vector< std::shared_ptr< myclass >>*
  myvar = new std::vector< std::shared_ptr< myclass >>
                ( 20, new myclass(1.3) );
\end{verbatim}
(Pointer to vector of 20 shared pointers to \n{myclass}, initialized
with unique instances.)  You can write this as:
\begin{verbatim}
auto myvar =
  new std::vector< std::shared_ptr< myclass >>
            ( 20, new myclass(1.3) );
\end{verbatim}

\begin{slide}{Type deduction}
  \label{sl:auto-deduct}
In:
\begin{verbatim}
std::vector< std::shared_ptr< myclass >>*
  myvar = new std::vector< std::shared_ptr< myclass >>
                ( 20, new myclass(1.3) );
\end{verbatim}
the compiler can figure it out:
\begin{verbatim}
auto myvar =
  new std::vector< std::shared_ptr< myclass >>
            ( 20, new myclass(1.3) );
\end{verbatim}
\end{slide}

\begin{block}{Type deduction in functions}:
  \label{sl:auto-fun}
  Return type can be deduced in C++17:
  \verbatimsnippet{autofun}  
\end{block}

\begin{block}{Type deduction in functions}:
  \label{sl:auto-method}
  Return type can be deduced in C++17:
  \verbatimsnippet{autoclass}  
\end{block}

\begin{block}{Auto and references, 1}
  \label{sl:auto-ref1}
  \n{auto} discards references and such:
  %
  \snippetwithoutput{autoplain}{auto}{plainget}
\end{block}

\begin{block}{Auto and references, 2}
  \label{sl:auto-ref2}
  Combine \n{auto} and references:
  %
  \snippetwithoutput{autoref}{auto}{refget}
\end{block}

\begin{block}{Auto and references, 3}
  \label{sl:auto-ref3}
  For good measure:
  %
  \snippetwitherror{constrefget}{auto}{constrefget}
\end{block}

\begin{comment}
  \begin{block}{Auto plus}
    \label{sl:auto-plus-const}
    Keywords like \n{const} and the reference character~\n{\&} can be
    added:
\begin{verbatim}
// class member
  some_object my_object;
// class method:
  some_object &get_some_object() { return my_object; };
// main program:
auto object_copy  = thing.get_some_object();
auto &object_mutable  = thing.get_some_object();
const auto &object_immutable  = thing.get_some_object();
\end{verbatim}
  \end{block}
\end{comment}

\Level 1 {Iterating}

\begin{block}{Auto iterators}
  \label{sl:auto-iterator}
\begin{verbatim}
vector<int> myvector(20);
for ( auto copy_of_int : myvector )
  s += copy_of_int;
for ( auto &ref_to_int : myvector )
  ref_to_int = s;
\end{verbatim}
Can be used with anything that is iteratable\\
(vector, map, your own classes!)
\end{block}

\Level 0 {Iterating over classes}
\label{sec:range-iter}

You know that you can iterate over \n{vector} objects:
\begin{verbatim}
vector<int> myvector(20);
for ( auto copy_of_int : myvector )
  s += copy_of_int;
for ( auto &ref_to_int : myvector )
  ref_to_int = s;
\end{verbatim}
(Many other \ac{STL} classes are iteratable like this.)

This is not magic: it is possible to iterate over any class
\n{iteratable} that has a number of conditions satisfied.

The class needs to have:
\begin{itemize}
\item a method \n{iteratable iteratable::begin()} that gives an
  object in the initial state, which we will call the `iterator object'; likewise
\item a method \n{iteratable iteratable::end()} that gives an 
  object in the final state; furthermore you need
\item an increment operator \n{void iteratable::operator++} that
  advances the iterator object to the next state;
\item a test \n{bool iteratable::operator!=(const iteratable&)} to determine
  whether the iteration can continue; finally
\item a dereference operator \n{iteratable::operator*} that takes the
  iterator object and returns its state.
\end{itemize}

\begin{slide}{Requirements}
  \label{sl:rangemethods}
  \begin{itemize}
  \item a method \n{iteratable iteratable::begin()}: initial state
  \item a method \n{iteratable iteratable::end()}:  final state
  \item an increment operator \n{void iteratable::operator++}: advance
  \item a test \n{bool iteratable::operator!=(const iteratable&)}
  \item a dereference operator \n{iteratable::operator*}: return state
  \end{itemize}
\end{slide}

\begin{block}{Simple illustration}
  \label{sl:bagdata}
  
  Let's make a class, called a \n{bag}, that models a set of integers,
  and we want to enumerate them. For simplicity sake we will make a set
  of contiguous integers:
  %
  \verbatimsnippet{bagdata}
\end{block}

\begin{block}{Internal state}
  \label{sl:bagseek}
  When you create an iterator object it will be copy of the object you
  are iterating over, except that it remembers how far it has
  searched:
  %
  \verbatimsnippet{bagseek}
\end{block}

\begin{block}{Initial/final state}
  \label{sl:bagbeginend}
  The \n{begin} method gives a \n{bag} with the seek parameter
  initialized:
  %
  \verbatimsnippet{bagbeginend}
  %
  These routines are public because they are (implicitly) called by the
  client code.
\end{block}

\begin{block}{Termination test}
  \label{sl:bagtest}
  The termination test method is called on the iterator, comparing it to
  the \n{end} object:
  %
  \verbatimsnippet{bagtest}
\end{block}

\begin{block}{Dereference}
  \label{sl:bagderef}
  Finally, we need the increment method and the dereference. Both access
  the \n{seek} member:
  %
  \verbatimsnippet{bagderef}
\end{block}

\begin{block}{Use case}
  \label{sl:bagfind}

  We can iterate over our own class:
  %
  \snippetwithoutput{bagfinditer}{loop}{bagfind}
\end{block}

If we add a method \n{has} to the class:
%
\verbatimsnippet{baghastest}
%
we can call this:
%
\verbatimsnippet{bagtestcall}
%
Of course, we could have written this function
without the range-based iteration, but this implementation is
particularly elegant.

You can now do exercise~\ref{ex:primerange}, implementing a prime
number generator with this mechanism.

If you think you understand \n{const}, consider that the \n{has}
method is conceptually \n{cost}. But if you add that keyword, the
compiler will complain about that use of \n{*this}, since it is
altered through the \n{begin} method.

\begin{exercise}
  \label{ex:rangeconstiter}
  Find a way to make \n{has} a \n{const} method.
\end{exercise}

\Level 0 {Lambdas}
\label{sec:lambda}

The C++11 mechanism of \indextermbus{lambda}{expression}s makes
dynamic definition of functions possible.

\begin{block}{Lambda expressions}
  \label{sl:lambda-syntax}
\begin{verbatim}
[capture] ( inputs ) -> outtype { definition };
\end{verbatim}
Example:
\verbatimsnippet{lambdaexp}
Store lambda in a variable:
\verbatimsnippet{lambdavar}
\end{block}

A non-trivial use of lambdas uses the \indexterm{capture} to fix one argument of a
function.
Let's say we want a function that computes exponentials for some fixed
exponent value. We take the 
\indextermtt{cmath} function
\begin{verbatim}
pow( x,exponent );
\end{verbatim}
and fix the exponent:
%
\verbatimsnippet{lambdacapt}
%
Now \n{powerfunction} is a function of one argument, which computes
that argument to a fixed power.

\begin{slide}{Capture parameter}
  \label{sl:lambda-capture}
  Capture value and reduce number of arguments:
  %
  \verbatimsnippet{lambdacapt}
  %
  Now \n{powerfunction} is a function of one argument, which computes
  that argument to a fixed power.
\end{slide}

Storing a lambda in a class is hard because it has unique
type. Solution: use \n{std::function}

\begin{block}{Lambda in object}
  \label{sl:lambda-class}
  %
  \verbatimsnippet{lambdaclass}
\end{block}

\begin{block}{Illustration}
  \label{sl:lambda-classed}
  \verbatimsnippet{lambdaclassed}
\end{block}

\begin{exercise}
  \label{ex:newtonlambda}
  Refer to exercise~\ref{ex:newton-root} for background.

  \snippetwithcomment{rootclass}{
    This is a class for zero finding with the newton method. The
    constructor takes a function and the derivative. Figure out what
    functions to supply to implement root finding.}
\end{exercise}

\Level 0 {Casts}
\label{sec:cast}

In C++, constants and variables have clear types. For cases where you
want to force the type to be something else, there is the
\indextermdef{cast} mechanism. With a cast you tell the compiler:
treat this thing as such-and-such a type, no matter how it was
defined.

In C, there was only one casting mechanism:
\begin{verbatim}
sometype x;
othertype y = (othertype)x;
\end{verbatim}
This mechanism is still available as the
\indextermtt{reinterpret_cast}, which does `take this byte and pretend
it is the following type':
\begin{verbatim}
sometype x;
auto y = reinterpret_cast<othertype>(x);
\end{verbatim}

The inheritance mechanism necessitates another casting mechanism.
An object from a derived class contains in it all the information of
the base class. It is easy enough to take a pointer to the derived
class, the bigger object, and cast it to a pointer to the base object.
The other way is harder.

Consider:
\begin{verbatim}
class Base {};
class Derived : public Base {};
Base *dobject = new Derived;
\end{verbatim}
Can we now cast dobject to a pointer-to-derived ?
\begin{itemize}
\item \indextermtt{static_cast} assumes that you know what you are
  doing, and it moves the pointer regardless.
\item \indextermtt{dynamic_cast} checks whether \n{dobject} was
  actually of class \n{Derived} before it moves the pointer, and
  returns \indextermtt{nullptr} otherwise.
\end{itemize}

\begin{remark}
  One further problem with the C-style casts is that their syntax is
  hard to spot, for instance by searching in an editor.
  Because C++ casts have a unique keyword, they are easily recognized.
\end{remark}

\begin{slide}{C++ casts}
  \label{sl:cpp-casts}
  Old-style `take this byte and pretend it is XYZ':
  \n{reinterpret_cast}
  
  Casting with classes:
  \begin{itemize}
  \item \n{static_cast} cast base to derived without check.
  \item \n{dynamic_cast} cast base to derived with check.
  \end{itemize}
  Adding/removing const: \n{const_cast}

  Syntactically clearly recognizable.
\end{slide}

Further reading \url{https://www.quora.com/How-do-you-explain-the-differences-among-static_cast-reinterpret_cast-const_cast-and-dynamic_cast-to-a-new-C++-programmer/answer/Brian-Bi}

\Level 1 {Static cast}

One use of casting is to convert to constants to a `larger' type. For
instance, allocation does not use integers but \indextermtt{size_t}.

\verbatimsnippet{longintcast}

However, if the
conversion is possible, the result may still not be `correct'.
%
\snippetwithoutput{longcast}{cast}{intlong}
%
There are no runtime tests on static casting.

Static casts are a good way of casting back void pointers to what they
were originally.

\begin{slide}{Const cast}
  \label{sl:const-cast}
  \verbatimsnippet{longintcast}
  \snippetwithoutput{longcast}{cast}{intlong}  
\end{slide}

\Level 1 {Dynamic cast}

Consider the case where we have a base class and derived classes.
%
\verbatimsnippet{polybase}
%
Also suppose that we have a function that takes a pointer to the base
class:
%
\verbatimsnippet{polymain}
%
The function can discover what derived class the base pointer refers
to:
%
\verbatimsnippet{polycast}

\begin{slide}{Pointer to base class}
  \label{sl:dyn-base-ptr}
  \begin{multicols}{2}
    Class and derived:
    \verbatimsnippet{polybase}
    \vfill\columnbreak
    Pass base pointer:
    \verbatimsnippet{polymain}
    \vfill\hbox{}
  \end{multicols}
\end{slide}

If we have a pointer to a derived object, stored in a pointer to a
base class object, it's possible to turn it safely into a derived
pointer again:
%
\snippetwithoutput{polycast}{cast}{deriveright}

On the other hand, a \indextermtt{static_cast} would not do the job:
%
\snippetwithoutput{polywrong}{cast}{derivewrong}

\begin{slide}{Cast to derived class}
  \label{sl:dyn-cast}
  This is how to do it:
  %
  \snippetwithoutput{polycast}{cast}{deriveright}
  %
  Do not use this function \n{g}:
  %
  \snippetwithoutput{polywrong}{cast}{derivewrong}
\end{slide}

Note: the base class needs to be polymorphic, meaning that that pure
virtual method is needed. This is not the case with a static cast,
but, as said, this does not work correctly in this case.

\Level 1 {Const cast}

With \indextermtt{const_cast} you can add or remove const from a
variable. This is the only cast that can do this.

\Level 1 {A word about void pointers}

A traditional use for casts in~C was the treatment of
\indextermsub{void}{pointer}s. The need for this is not as severe in
C++ as it was before.

A typical use of void pointers appears in the
PETSc~\cite{petsc-efficient,petsc-home-page} library. Normally when
you call a library routine, you have no further access to what happens
inside that routine. However, PETSc has the functionality for you to
specify a monitor so that you can print out internal quantities.
\begin{verbatim}
int KSPSetMonitor(KSP ksp,
  int (*monitor)(KSP,int,PetscReal,void*),
  void *context,
  // one parameter omitted
  );
\end{verbatim}
Here you can declare your own monitor routine that will be called
internally: the library makes a \indexterm{call-back} to your code.
Since the library can not predict whether your monitor routine may
need further information in order to function, there is the
\n{context} argument, where you can pass a structure as void pointer.

This mechanism is no longer needed in C++ where you would use a
\indexterm{lambda} (section~\ref{sec:lambda}):
\begin{verbatim}
KSPSetMonitor( ksp,
  [mycontext] (KSP k,int ,PetscReal r) -> int {
    my_monitor_function(k,r,mycontext); } );
\end{verbatim}


\Level 0 {lvalue vs rvalue}
\label{sec:lrvalue}

The terms `lvalue' and `rvalue' sometimes appear in compiler error
messages.
\begin{verbatim}
int foo() {return 2;}

int main()
{
    foo() = 2;

    return 0;
}

# gives:
test.c: In function 'main':
test.c:8:5: error: lvalue required as left operand of assignment
\end{verbatim}

See the `lvalue' and `left operand'? To first order of approximation
you're forgiven for thinking that an \indextermdef{lvalue} is something
on the left side of an assignment. The name actually means `locator
value': something that's associated with a specific location in
memory. Thus an lvalue is, also loosely, something that can be modified.

An \indextermdef{rvalue} is then something that appears on the right
side of an assignment, but is really defined as everything that's not
an lvalue. Typically, rvalues can not be modified.

The assignment \n{x=1} is legal because a variable \n{x} is at some specific
location in memory, so it can be assigned to. On the other hand,
\n{x+1=1} is not legal, since \n{x+1} is at best a temporary,
therefore not at a specific memory location, and thus not an lvalue.

Less trivial examples:
\begin{verbatim}
int foo() { x = 1; return x; }
int main() {
  foo() = 2;
}
\end{verbatim}
is not legal because \n{foo} does not return an lvalue. However,
\begin{verbatim}
class foo {
private:
  int x;
public:
  int &xfoo() { return x; };
};
int main() {
  foo x;
  x.xfoo() = 2;
\end{verbatim}
is legal because the function \n{xfoo} returns a reference to the
non-temporary variable \n{x} of the \n{foo} object.

Not every lvalue can be assigned to: in
\begin{verbatim}
const int a = 2;
\end{verbatim}
the variable \n{a} is an lvalue, but can not appear on the left hand
side of an assignment.

\Level 1 {Conversion}

Most lvalues can quickly be converted to rvalues:
\begin{verbatim}
int a = 1;
int b = a+1;
\end{verbatim}
Here \n{a} first functions as lvalue, but becomes an rvalue in the
second line.

The ampersand operator takes an lvalue and gives an rvalue:
\begin{verbatim}
int i;
int *a = &i;
&i = 5; // wrong
\end{verbatim}

\Level 1 {References}

The ampersand operator yields a reference. It needs to be assigned
from an lvalue, so
\begin{verbatim}
std::string &s = std::string(); // wrong
\end{verbatim}
is illegal. The type of \n{s} is an `lvalue reference' and it can not
be assigned from an rvalue.

On the other hand
\begin{verbatim}
const std::string &s = std::string();
\end{verbatim}
works, since \n{s} can not be modified any further.

\Level 1 {Rvalue references}
\label{sec:rvalue-ref}

A new feature of C++ is
intended to minimize the amount of data copying through
\indexterm{move semantics}.

Consider a copy assignment operator
\begin{verbatim}
BigThing& operator=( const BigThing &other ) {
  BigThing tmp(other); // standard copy
  std::swap( /* tmp data into my data */ );
  return *this;
};
\end{verbatim}
This calls a copy constructor and a destructor on \n{tmp}. (The use of
a temporary makes this safe under exceptions. The \indextermtt{swap}
method never throws an exception, so there is no danger of half-copied
memory.)

However, if you assign
\begin{verbatim}
thing = BigThing(stuff);
\end{verbatim}
Now a constructor and destructor is called for the temporary rvalue object on
the right-hand side.

Using a syntax that is new in \indexterm{C++}, we create an
\indextermbus{rvalue}{reference}:
\begin{verbatim}
BigThing& operator=( BigThing &&other ) {
  swap( /* other into me */ );
  return *this;
}
\end{verbatim}

\Level 0 {Move semantics}

With overloaded addition on matrices (or any other big object):
\begin{lstlisting}
Matrix operator+(Matrix &a,Matrix &b);
\end{lstlisting}
the actual addition will involve a copy:
\begin{lstlisting}
Matrix c = a+b;
\end{lstlisting}

Use a move constructor:
\begin{lstlisting}
class Matrix {
private:
  Representation rep;
public:
  Matrix(Matrix &&a) {
    rep = a.rep;
    a.rep = {};
  }
};
\end{lstlisting}

