% -*- latex -*-
%%%%%%%%%%%%%%%%%%%%%%%%%%%%%%%%%%%%%%%%%%%%%%%%%%%%%%%%%%%%%%%%
%%%%
%%%% This TeX file is part of the course
%%%% Introduction to Scientific Programming in C++/Fortran2003
%%%% copyright 2017-9 Victor Eijkhout eijkhout@tacc.utexas.edu
%%%%
%%%% modulef.tex : fortran modules
%%%%
%%%%%%%%%%%%%%%%%%%%%%%%%%%%%%%%%%%%%%%%%%%%%%%%%%%%%%%%%%%%%%%%

Fortran has a clean mechanism for importing data, functions, types
that are defined in another file.

\begin{block}{Module definition}
  \label{sl:fmod-def}
  Modules look like a program, but without executable code:
  %
  \verbatimsnippet{pointmod}
\end{block}

\begin{block}{Module use}
  \label{sl:fmod-use}
  Module imported through \indextermfort{use} statement;\\
  comes before \n{implicit none}
  %
  \verbatimsnippet{pointmodprog}
\end{block}

\begin{exercise}
  \label{ex:fmod-rect}
  Take exercise~\ref{ex:ftype-rect} and put all type definitions and
  all functions in a module.
\end{exercise}

\Level 0 {Modules for program modularization}

Modules are Fortran's mechanism for supporting
\indextermsub{separate}{compilation}: you can put yuor module in one
file, your main program in another, and compile them separately.

\begin{block}{Separate compilation of modules}
  \label{sl:fcompile-mods}
  Suppose program is split over \n{theprogram.F90} and \n{themodule.F90}.
  \begin{itemize}
  \item \n{icpc -c themodule.F90}; this gives
  \item an \indextermsub{object}{file} that will be linked later, and
  \item a \n{.mod} file (with the name of the module, not of the
    file);
  \item \n{icpc -c theprogram.F90} will read the \n{.mod} file; and finally
  \item \n{icpc -o myprogram theprogram.o themodule.o} uses the
    compiler as \indexterm{linker} to form the executable.
  \end{itemize}
  The module needs to be compiled before the program.
\end{block}

\Level 0 {Modules}
\label{sec:modulef}

A module is a container for definitions of subprograms and types, and
for data such as constants and variables. A~module is not a 
structure or object: there is only one instance.

What do you use a module for?
\begin{itemize}
\item Type definitions: it is legal to have the same type definition
  in multiple program units, but this is not a good idea. 
  Write the definition just once in a module and make it available
  that way.
\item Function definitions: this makes the functions available in multiple
  sources files of the same program, or in multiple programs.
\item Define constants: for physics simulations, put all constants in
  one module and use that, rather than spelling out the constants each
  time.
\item Global variables: put variables in a module if they do not fit
  an obvious scope.
  \begin{f77note}
  Modules are much cleaner than common blocks. Do not use those.
  \end{f77note}
\end{itemize}
%

Any routines come after the \n{contains}\index{contains!in modules}

A module is made available with the \indextermtt{use} keyword, which
needs to go before the \n{implicit none}.
%
\begin{block}{Module use}
  \label{sl:moduleuse}
  \verbatimsnippet{moduleuse}
  Also possible:
\begin{lstlisting}
Use mymodule, Only: func1,func2
Use mymodule, func1 => new_name1
\end{lstlisting}
\end{block}

By default, all the contents of a module is usable by a subprogram
that uses it. However, a keyword \indextermtt{private} make module
contents available only inside the module.
You can make the default behaviour explicit by using the
\indextermtt{public} keyword. Both \n{public,private} can be used as
attributes on definitions in the module.
There is a keyword \indextermtt{protected} for data members that
are public, but can not be altered by code outside the module.

If you compile a module, you will find a \n{.mod} file in your
directory. (This is little like a~\n{.h} file in~C++.)
If this file is not present, you can not \n{use} the module in another
program unit, so you need to compile the file containing the module
first.

\begin{exercise}
  \label{ex:fpointmod}
  Write a module \n{PointMod} that defines a type \n{Point} and a
  function \n{distance} to make this code work:
  %
  \verbatimsnippet{pointmainex}
  %
  Put the program and module in two separate files and compile thusly:
\begin{lstlisting}
ifort -g -c pointmod.F90
ifort -g -c pointmain.F90
ifort -g  -o pointmain pointmod.o pointmain.o 
\end{lstlisting}
\end{exercise}

\Level 1 {Polymorphism}

\begin{lstlisting}
module somemodule

INTERFACE swap
MODULE PROCEDURE swapreal, swapint, swaplog, swappoint
END INTERFACE

  contains
  subroutine swapreal
  ...
  end subroutine swapreal
  subroutine swapint
  ...
  end subroutine swapint
\end{lstlisting}

\Level 1 {Operator overloading}

\begin{lstlisting}
MODULE operator_overloading
IMPLICIT NONE
...
INTERFACE OPERATOR (+)
MODULE PROCEDURE concat
END INTERFACE
\end{lstlisting}

including the assignment operator:

\begin{lstlisting}
INTERFACE ASSIGNMENT (=) 
subroutine_interface_body
END INTERFACE
\end{lstlisting}

This mechanism can also be used for dot-operators:

\begin{lstlisting}
INTERFACE OPERATOR (.DIST.)
MODULE PROCEDURE calcdist
END INTERFACE
\end{lstlisting}

