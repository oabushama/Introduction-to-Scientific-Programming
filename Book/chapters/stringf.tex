% -*- latex -*-
%%%%%%%%%%%%%%%%%%%%%%%%%%%%%%%%%%%%%%%%%%%%%%%%%%%%%%%%%%%%%%%%
%%%%
%%%% This TeX file is part of the course
%%%% Introduction to Scientific Programming in C++/Fortran2003
%%%% copyright 2017/8 Victor Eijkhout eijkhout@tacc.utexas.edu
%%%%
%%%% stringf.tex : string handling in Fortran
%%%%
%%%%%%%%%%%%%%%%%%%%%%%%%%%%%%%%%%%%%%%%%%%%%%%%%%%%%%%%%%%%%%%%

\Level 0 {String denotations}

A string can be enclosed in single or double quotes. That makes it
easier to have the other type in the string.

\verbatimsnippet{fquotes}
%\snippetwithoutput{fquotes}{stringf}{quote}

\Level 0 {Characters}

\Level 0 {Strings}

The \indextermfort{len} function gives the length of the string as
it was allocated, not how much non-blank content you put in it.

\snippetwithoutput{fstrlen}{stringf}{strlen}

To get the more intuitive length of a string, that is, the location of
the last non-blank character, you need to \indextermfort{trim} the string.

Intrinsic functions: \n{LEN(string)}, \n{INDEX(substring,string)},
\n{CHAR(int)}, \n{ICHAR(char)}, \n{TRIM(string)}

\snippetwithoutput{fconcat}{stringf}{concat}

\Level 0 {Strings versus character arrays}

