% -*- latex -*-
%%%%%%%%%%%%%%%%%%%%%%%%%%%%%%%%%%%%%%%%%%%%%%%%%%%%%%%%%%%%%%%%
%%%%
%%%% This TeX file is part of the course
%%%% Introduction to Scientific Programming in C++/Fortran2003
%%%% copyright 2017/8 Victor Eijkhout eijkhout@tacc.utexas.edu
%%%%
%%%% struct.tex : about structures
%%%%
%%%%%%%%%%%%%%%%%%%%%%%%%%%%%%%%%%%%%%%%%%%%%%%%%%%%%%%%%%%%%%%%

\Level 0 {Why structures?}
\label{sec:struct}

You have seen the basic datatypes in section~\ref{sec:ctypes}. These
are enough to program whatever you want, but it would be nice if the
language had some datatypes that are more abstract, closer to the
terms in which you think about your application. For instance, if you
are programming something to do with geometry, you had rather talk
about points than explicitly having to manipulate their coordinates.

Structures are a
first way to define your own datatypes. A~\indextermttdef{struct}
acts like a datatype for which you choose the name. A~\n{struct}
contains other datatypes; these can be elementary, or other structs.
%
\verbatimsnippet{structdef}

The elements of a structure are also called
\emph{members}\index{member!of struct}.
You can give them an initial value:
\begin{verbatim}
struct vector { double x=0.; double y=0.; } ;
\end{verbatim}
  
\begin{slide}{Bundling information}
  \label{sl:struct-why}
  Sometimes a number of variables belong logically together. For
  instance two doubles can be the $x,y$ components of a vector.

  This can be captured in the \n{struct} construct.

  \verbatimsnippet{structdef}

  (This can go in the main program or before it.)

The elements of a structure are usually called \indexterm{members}.
\end{slide}

\Level 0 {The basics of structures}

\begin{block}{How to use structures}
  \label{sl:structinprog}
  \begin{enumerate}
  \item Declare what is in your structure;
  \item Make some structures;
  \item Use them.
  \end{enumerate}
\begin{verbatim}
// declaration of the struct
struct AStructName { int num; double val; }
int main() {
  // declaration of struct variables
  AStructName mystruct1,mystruct2;
  .... code that uses your structures ....
}
\end{verbatim}
\end{block}

\begin{block}{Struct initialization}
  \label{sl:structinit}
  You assign a whole struct, or set defaults in the definition.
  %
  \verbatimsnippet{pointinit}
\end{block}

\begin{block}{Using structures}
  \label{sl:struct-use}
  Once you have defined a structure, you can make variables of that
  type. Setting and initializing them takes a new syntax:
  \snippetwithoutput{structuse}{struct}{point}
  Period syntax: `apostrophe-s'.
\end{block}

Note: if you use initializations in the \n{struct} definition,
you can not use the brace-assignment.

\begin{block}{Functions on structures}
  \label{sl:struct-pass}
  You can pass a structure to a function:
  %\verbatimsnippet{structpass}
  \snippetwithoutput{structpass}{struct}{pointfun}
\end{block}

\begin{block}{Returning structures}
  \label{sl:struct-return}
  You can return a structure from a function:
  %
  \snippetwithoutput{structreturn}{struct}{pointadd}

  (In case you're wondering about scopes and lifetimes here: the
  explanation is that the returned value is copied.)
\end{block}

If you are doing the prime project (chapter~\ref{ch:prime}) you can
now do exercise~\ref{ex:prime:struct}.

\begin{exercise}
  \label{ex:vecstruct}
  Write a function \n{inner_product} that takes two \n{vector}
  structures and computes the inner product.
\end{exercise}

\begin{exercise}
  \label{ex:matstruct}
  Write a $2\times 2$ matrix class (that is, a structure storing 4
  real numbers), and write a function \n{multiply}
  that multiplies a matrix times a vector.

  Can you make a matrix structure that is based on the vector
  structure, for instance using vectors to store the matrix rows, and
  then using the inner product method to multiply matrices?
\end{exercise}

\begin{block}{Passing structures by reference}
  \label{sl:struct-passref}
  Prevent copying cost by passing by reference, use \n{const} to
  prevent changes:
  \verbatimsnippet{structpassref}
\end{block}



