% -*- latex -*-
%%%%%%%%%%%%%%%%%%%%%%%%%%%%%%%%%%%%%%%%%%%%%%%%%%%%%%%%%%%%%%%%
%%%%
%%%% This TeX file is part of the course
%%%% Introduction to Scientific Programming in C++/Fortran2003
%%%% copyright 2017-9 Victor Eijkhout eijkhout@tacc.utexas.edu
%%%%
%%%% cpp.tex : about the C preprocessor
%%%%
%%%%%%%%%%%%%%%%%%%%%%%%%%%%%%%%%%%%%%%%%%%%%%%%%%%%%%%%%%%%%%%%

In your source files you have seen lines starting with a hash sign,
like
\begin{lstlisting}
#include <iostream>
\end{lstlisting}
Such lines are interpreted by the
%
\emph{C~preprocessor}\index{C preprocessor|see{preprocessor}}.

Your source file is transformed to another source file, in a
source-to-source translation stage, and only that second file is
actually compiled by the
%
\emph{compiler}\index{compiler!and preprocessor}.
In the case of an \n{#include} statement, the preprocessing stage
takes form of literaly inserting another file, here a
%
\indexterm{header file}
into your source.

There are more sophisticated uses of the preprocessor.

\Level 0 {Textual substitution}
\index{preprocessor!macros|(}

Suppose your program has a number of arrays and loop bounds that are
all identical. To make sure the same number is used, you can create a
variable, and pass that to routines as necessary.
\begin{lstlisting}
void dosomething(int n) {
  for (int i=0; i<n; i++) ....
}

int main() {
  int n=100000;

  double array[n];
   
  dosomething(n);
\end{lstlisting}
You can also use a \emph{preprocessor macro}:
\begin{lstlisting}
#define N 100000
void dosomething() {
  for (int i=0; i<N; i++) ....
}

int main() {
  double array[N];
   
  dosomething();
\end{lstlisting}
It is traditional to use all uppercase for such macros.

\Level 0 {Parametrized macros}

Instead of simple text substitution, you can have
%
\emph{parametrized preprocessor macros}\index{preprocessor!macro!parametrized}
\begin{lstlisting}
#define CHECK_FOR_ERROR(i) if (i!=0) return i
...
ierr = some_function(a,b,c); CHECK_FOR_ERROR(ierr);
\end{lstlisting}

When you introduce parameters, it's a good idea to use lots of parentheses:
\begin{lstlisting}
// the next definition is bad!
#define MULTIPLY(a,b) a*b
...
x = MULTIPLY(1+2,3+4);
\end{lstlisting}
Better
\begin{lstlisting}
#define MULTIPLY(a,b) (a)*(b)
...
x = MULTIPLY(1+2,3+4);
\end{lstlisting}

Another popular use of macros is to simulate multi-dimensional indexing:
\begin{lstlisting}
#define INDEX2D(i,j,n) (i)*(n)+j
...
double array[m,n];
for (int i=0; i<m; i++)
  for (int j=0; j<n; j++)
    array[ INDEX2D(i,j,n) ] = ...
\end{lstlisting}

\begin{exercise}
  Write a macro that simulates 1-based indexing:
\begin{lstlisting}
#define INDEX2D1BASED(i,j,n)  ????
...
double array[m,n];
for (int i=1; i<=m; i++)
  for (int j=n; j<=n; j++)
    array[ INDEX2D1BASED(i,j,n) ] = ...
\end{lstlisting}
\end{exercise}

\index{preprocessor!macros|)}

\Level 0 {Conditionals}

\index{preprocessor!conditionals|(}

There are a couple of \emph{preprocessor conditions}.

\Level 1 {Check on a value}

The \n{#if} macro tests on nonzero. A common application is to
temporarily remove code from compilation:
\begin{lstlisting}
#if 0
  bunch of code that needs to
  be disabled
#endif
\end{lstlisting}

\Level 1 {Check for macros}

The \indexpragma{ifdef}
%\n{#ifdef}\index{#ifdef@{\texttt{\#ifdef}}}
test tests for a macro being defined. Conversely,
\indexpragma{ifndef}
%\n{#ifndef}\index{#ifndef@{\texttt{\#ifndef}}}
tests for a macro not being defined. For instance,
\begin{lstlisting}
#ifndef N
#define N 100
#endif
\end{lstlisting}
Why would a macro already be defined? Well you can do that on the
compile line:
\begin{lstlisting}
  icpc -c file.cc -DN=500
\end{lstlisting}

Another application for this test is in preventing recursive inclusion
of header files; see section~\ref{ex:globalvar}.

\index{preprocessor!conditionals|)}

\Level 1 {Including a file only once}

It is easy to wind up including a file such as \n{iostream} more than
once, if it is included in multiple other header files. This adds to
your compilation time, or may lead to subtle problems. A~header file
may even circularly include itself. To prevent this, header files
often have a structure%
\index{#ifndef@{\texttt{\#ifndef}}}
\begin{lstlisting}
// this is foo.h
#ifndef FOO_H
#define FOO_H

// the things that you want to include

#endif
\end{lstlisting}
Now the file will effectively be included only once: the second time
it is included its content is skipped.

Many compilers support the pragma
\indexpragma{once}
%\n{once}\index{#pragma once@{\texttt{\#pragma once}}}%
%\index{once@{\texttt{once}}|see{\texttt{\#pragma once}}}
(which, however, is not a language standard) that has the same
effect:
\begin{lstlisting}
// this is foo.h
#pragma once

// the things you want to include only once
\end{lstlisting}

\Level 0 {Other pragmas}

\begin{itemize}
\item Packing data structure without padding bytes by \indexpragma{pack}
  \begin{lstlisting}
    #pragma pack(push, 1)
    // data structures
    #pragma pack(pop)
  \end{lstlisting}
\end{itemize}
