% -*- latex -*-
%%%%%%%%%%%%%%%%%%%%%%%%%%%%%%%%%%%%%%%%%%%%%%%%%%%%%%%%%%%%%%%%
%%%%
%%%% This TeX file is part of the course
%%%% Introduction to Scientific Programming in C++/Fortran2003
%%%% copyright 2017/8 Victor Eijkhout eijkhout@tacc.utexas.edu
%%%%
%%%% prime.tex : exercises for prime number finding
%%%%
%%%%%%%%%%%%%%%%%%%%%%%%%%%%%%%%%%%%%%%%%%%%%%%%%%%%%%%%%%%%%%%%

\Level 0 {Arithmetic}

\prerequisite{\ref{sec:expr}}

\begin{exercise}
  \label{ex:prime:sum}
  Read two integers into two variables, and print their sum, product,
  quotient, modulus.
  
  %% Do this also with real variables.

  %% What is $5/3$ as integer? If you compute it as a real number~$x$, is
  %% $3x$ exactly equal to~$5$?
\end{exercise}

A less common operator is the modulo operator~\n{\char`\%}.

\begin{exercise}
  \label{ex:prime:modvar}
  Read two numbers and print out their modulus. Two ways:
  \begin{itemize}
  \item use the \n{cout} function to print the expression, or
  \item assign the expression to a variable, and print that variable.
  \end{itemize}
\end{exercise}

\Level 0 {Conditionals}

\prerequisite{\ref{sec:if}}

\begin{exercise}
  \label{ex:prime:divtest}
  Read two numbers and print a message like
\begin{verbatim}
3 is a divisor of 9
\end{verbatim}
  if the first is an exact
  divisor of the second, and another message
\begin{verbatim}
4 is not a divisor of 9
\end{verbatim}
if it is not.
\end{exercise}

\Level 0 {Looping}

\prerequisite{\ref{sec:for}}

\begin{exercise}
  \label{ex:prime:test}
  Read an integer and determine whether it is prime by testing for the
  smaller numbers whether they are a
  divisor of that number. 

  Print a final message
\begin{verbatim}
Your number is prime
\end{verbatim}
or
\begin{verbatim}
Your number is not prime: it is divisible by ....
\end{verbatim}
where you report just one found factor.
\end{exercise}

\begin{exercise}
  \label{ex:prime:test2}
  Rewrite the previous exercise with a boolean variable to represent
  the primeness of the input number.
\end{exercise}

\Level 0 {Functions}

\prerequisite{\ref{ch:function}}

Above you wrote several lines of code to test whether a number was
prime.

\begin{exercise}
  \label{ex:prime:func}
  Write a function \n{is_prime} that takes an integer input, and returns a boolean
  corresponding to whether the input was prime.
\begin{verbatim}
int main() {
  bool isprime;
  isprime = prime_test_function(13);
\end{verbatim}
  Read the number in, and print the value of the boolean.
\end{exercise}

\Level 0 {While loops}

\prerequisite{\ref{sec:loopuntil}}

\begin{exercise}
  \label{ex:prime:while}
  Take your prime number testing function \n{is_prime}, and use it to
  write program that prints multiple primes:
  \begin{itemize}
  \item Read an integer \n{how_many} from the input, indicating how
    many (successive) prime numbers should be printed.
  \item Print that many successive primes, each on a separate line.
  \item (Hint: keep a variable
    \n{number_of_primes_found} that is increased whenever a new prime is found.)
  \end{itemize}
\end{exercise}

\Level 0 {Structures}

\prerequisite{\ref{sec:struct}, \ref{sec:reference}}

A~\n{struct} functions to bundle together a number of data item. We
only discuss this as a preliminary to classes.

\begin{exercise}
  \label{ex:prime:struct}
  Rewrite the exercise that found a predetermined number of primes,
  putting the \n{number_of_primes_found} and
  \n{last_number_tested} variables in a structure. Your main program should
  now look like:
  %
  \verbatimsnippet{primestructmain}
  %
%  You also need to use the structure in the \n{nextprime} exercise.
\end{exercise}

\Level 0 {Classes and objects}

\prerequisite{\ref{sec:object}, \ref{sec:objectf}}

In exercise~\ref{ex:prime:struct} you made a structure that contains
the data for a primesequence, and you have separate functions that
operate on that structure or on its members.

\begin{exercise}
  \label{ex:prime:sequence}
  Write a class \n{primegenerator} that contains members
  \n{number_of_primes_found} and
  \n{last_number_tested}, and methods \n{nextprime}, \n{isprime}.
  %
  {\footnotesize Hint: the
  function \n{nextprime} does not need the object as argument,
  because the members are in the object, and therefore global
  to that function.}

  Your main program should look as follows:
  %
  \verbatimsnippet{primesequencemain}
\end{exercise}

In the previous exercise you defined the \n{primegenerator} class, and
you made one object of that class:
\begin{verbatim}
primegenerator sequence;
\end{verbatim}
But you can make multiple generators, that all have their own internal
data and are therefore independent of each other.

\begin{exercise}
  \label{ex:goldbach:conj}
  The \indexterm{Goldbach conjecture} says that every even number,
  from~4 on, is the sum of two primes $p+q$. Write a program to test this
  for the even numbers up to 20~million.

  Make an outer loop over the even numbers~$e$. In each iteration,
  make a \n{primegenerator} object to generate $p$ values.
  For each~$p$ test whether~$e-p$ is prime.

  For each even number, print out how it is the sum of two primes. If
  multiple possibilities exist, only print the first one you find.
\end{exercise}

\begin{exercise}
  \label{ex:prime:goldbach-pqr}
  The \indexterm{Goldbach conjecture} says that every even number~$2n$
  (starting at~4), is the sum of two primes $p+q$: \[ 2n=p+q.\]
  Equivalently, every number~$n$ is equidistant from two primes:
  \[ n=\frac{p+q}2\qquad\hbox{or}\qquad q-n = n-p.\]
  In particular this holds for each prime number:
  \[ \forall_{p\,\mathrm{prime}}\exists_{q\,\mathrm{prime}}\colon
  \hbox{$r\equiv p+(p-q)$ is prime}. \]

  Write a program that tests this. You need two prime number
  generators, one for the $p$-sequence and one for the $q$-sequence.
  For each $p$ value, 
  when the program finds the $q$~value, print the $q,p,r$ triple and
  move on to the next~$p$.

  Allocate an array where you record all the $p-q$ distances that you
  found. Print some elementary statistics, for instance: what is the average, do the
  distances increase or decrease with~$p$?
\end{exercise}

\Level 0 {Arrays}

Another algorithm for finding prime numbers is the
\indexterm{Eratosthenes sieve}. It goes like this.
\begin{enumerate}
\item You take a range of integers, starting at~2.
\item Now look at the first number. That's a prime number.
\item Scratch out all of its multiples.
\item Find the next number that's not scratched out; since that's
  not a multiple of a previous number, it must be a prime
  number. Report it, and go back to the previous step.
\end{enumerate}

The new mechanism you need for this is the data structure for storing
all the integers.
\begin{verbatim}
int N = 1000;
vector<int> integers(N);
\end{verbatim}

\begin{exercise}
  Read in an integer that denotes the largest number you want to test.
  Make an array of integers that long. Set the elements to the
  successive integers.
\end{exercise}

\Level 0 {Inheritance and virtual methods}

If you turn the Eratosthenes sieve into a generator class, you see
that it has a \n{nextprime} function, just like the other generator
class you wrote above.

\begin{exercise}
  Use inheritance to express the commonality between the classes:
  \begin{itemize}
  \item Make a \n{generator} base class; it probably remembers the
    last prime found;
  \item Make \n{bruteforce_generator} and \n{sieve_generator} classes
    that derive from \n{generator}, and that each have their own
    \n{nextprime} method.
  \end{itemize}
\end{exercise}

Since the derived classes have different \n{nextprime} methods, you
can not really put that method in the base class. However:

\begin{exercise}
  Add \n{nextprime} as a virtual method to the base class.
\end{exercise}
