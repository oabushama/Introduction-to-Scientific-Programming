% -*- latex -*-
%%%%%%%%%%%%%%%%%%%%%%%%%%%%%%%%%%%%%%%%%%%%%%%%%%%%%%%%%%%%%%%%
%%%%
%%%% This TeX file is part of the course
%%%% Introduction to Scientific Programming in C++/Fortran2003
%%%% copyright 2017-9 Victor Eijkhout eijkhout@tacc.utexas.edu
%%%%
%%%% prime.tex : exercises for prime number finding
%%%%
%%%%%%%%%%%%%%%%%%%%%%%%%%%%%%%%%%%%%%%%%%%%%%%%%%%%%%%%%%%%%%%%

In this chapter you will do a number of exercises regarding prime
numbers that build on each other. Each section lists the required
prerequisites. Conversely, the exercises here are also referenced from
the earlier chapters.

\Level 0 {Arithmetic}

\prerequisite{\ref{sec:expr}}

\begin{exercise}
  \label{ex:prime:modvar}
  Read two numbers and print out their modulus. Two ways:
  \begin{itemize}
  \item use the \n{cout} function to print the expression, or
  \item assign the expression to a variable, and print that variable.
  \end{itemize}
\end{exercise}

\Level 0 {Conditionals}

\prerequisite{\ref{sec:if}}

\begin{exercise}
  \label{ex:prime:divtest}
  Read two numbers and print a message like
\begin{lstlisting}
3 is a divisor of 9
\end{lstlisting}
  if the first is an exact
  divisor of the second, and another message
\begin{lstlisting}
4 is not a divisor of 9
\end{lstlisting}
if it is not.
\end{exercise}

\Level 0 {Looping}

\prerequisite{\ref{sec:for}}

\begin{exercise}
  \label{ex:prime:test}
  Read an integer and determine whether it is prime by testing for the
  smaller numbers whether they are a
  divisor of that number. 

  Print a final message
\begin{lstlisting}
Your number is prime
\end{lstlisting}
or
\begin{lstlisting}
Your number is not prime: it is divisible by ....
\end{lstlisting}
where you report just one found factor.
\end{exercise}

\begin{exercise}
  \label{ex:prime:test2}
  Rewrite the previous exercise with a boolean variable to represent
  the primeness of the input number.
\end{exercise}

\Level 0 {Functions}

\prerequisite{\ref{ch:function}}

Above you wrote several lines of code to test whether a number was
prime.

\begin{exercise}
  \label{ex:prime:func}
  Write a function \n{test_if_prime} that has an integer parameter, and returns a boolean
  corresponding to whether the parameter was prime.
\begin{lstlisting}
int main() {
  bool isprime;
  isprime = test_if_prime(13);
\end{lstlisting}
  Read the number in, and print the value of the boolean.

  Does your function have one or two \n{return} statements?
  Can you imagine what the other possibility looks like?
  Do you have an argument for or against it?
\end{exercise}

\Level 0 {While loops}

\prerequisite{\ref{sec:loopuntil}}

\begin{exercise}
  \label{ex:prime:while}
  Take your prime number testing function \n{test_if_prime}, and use it to
  write a program that prints multiple primes:
  \begin{itemize}
  \item Read an integer \n{how_many} from the input, indicating how
    many (successive) prime numbers should be printed.
  \item Print that many successive primes, each on a separate line.
  \item (Hint: keep a variable
    \n{number_of_primes_found} that is increased whenever a new prime is found.)
  \end{itemize}
\end{exercise}

\Level 0 {Structures}

\prerequisite{\ref{sec:struct}, \ref{sec:reference}}

A~\n{struct} functions to bundle together a number of data item. We
only discuss this as a preliminary to classes.

\begin{exercise}
  \label{ex:prime:struct}
  Rewrite the exercise that found a predetermined number of primes,
  putting the \n{number_of_primes_found} and
  \n{last_number_tested} variables in a structure. Your main program should
  now look like:
  %
  \verbatimsnippet{primestructmain}
  %
  Hint: the variable \n{last_number_tested} does not appear in the
  main program. Where does it get updated? Also, there is no update of
  \n{number_of_primes_found} in the main program. Where do you think
  it would happen?
\end{exercise}

\Level 0 {Classes and objects}

\prerequisite{\ref{sec:object}}

In exercise~\ref{ex:prime:struct} you made a structure that contains
the data for a primesequence, and you have separate functions that
operate on that structure or on its members.

\begin{exercise}
  \label{ex:prime:sequence}
  Write a class \n{primegenerator} that contains 
  \begin{itemize}
  \item members  \n{how_many_primes_found} and
    \n{last_number_tested},
  \item a method \n{nextprime};
  \item Also write a function \n{isprime} that does not need to be
    in the class.
  \end{itemize}

  Your main program should look as follows:
  %
  \verbatimsnippet{primesequencemain}
\end{exercise}

In the previous exercise you defined the \n{primegenerator} class, and
you made one object of that class:
\begin{lstlisting}
primegenerator sequence;
\end{lstlisting}
But you can make multiple generators, that all have their own internal
data and are therefore independent of each other.

\begin{exercise}
  \label{ex:goldbach:conj}
  The \indexterm{Goldbach conjecture} says that every even number,
  from~4 on, is the sum of two primes $p+q$. Write a program to test this
  for the even numbers up to a bound that you read in.

  This is a great exercise for a top-down approach!
  Make an outer loop over the even numbers~$e$. In each iteration,
  make a \n{primegenerator} object to generate $p$ values.
  For each~$p$ test whether~$e-p$ is prime.

  For each even number~\n{e} then print \n{e,p,q}, for instance:
  \begin{verbatim}
The number 10 is 3+7
  \end{verbatim}
  If multiple possibilities exist, only print the first one you find.
\end{exercise}

\begin{exercise}
  \label{ex:prime:goldbach-pqr}
  The \indexterm{Goldbach conjecture} says that every even number~$2n$
  (starting at~4), is the sum of two primes $p+q$: \[ 2n=p+q.\]
  Equivalently, every number~$n$ is equidistant from two primes:
  \[ n=\frac{p+q}2\qquad\hbox{or}\qquad q-n = n-p.\]
  In particular this holds for each prime number:
  \[ \forall_{r\,\mathrm{prime}}\exists_{p,q\,\mathrm{prime}}\colon
  \hbox{$r=(p+q)/2$ is prime}. \]

  Write a program that tests this. You need two prime number
  generators, one for the $p$-sequence and one for the $q$-sequence.
  For each $p$ value, 
  when the program finds the $q$~value, print the $q,p,r$ triple and
  move on to the next~$p$.

  Allocate an array where you record all the $p-q$ distances that you
  found. Print some elementary statistics, for instance: what is the average, do the
  distances increase or decrease with~$p$?
\end{exercise}

\Level 0 {Eratosthenes sieve}

The Eratosthenes sieve is an algorithm for prime numbers that
step-by-step filters out the multiples of any prime it finds.
\begin{enumerate}
\item Start with the integers from~2: $2,3,4,5,6,\ldots$
\item The first number, 2, is a prime: record it and remove all
  multiples, giving
  \[ 3,5,7,9.11,13,15,17\dots \]
\item The first remaining number, 3, is a prime: record it and remove
  all multiples, giving
  \[ 5,7,11,13,17,19,23,25,29\ldots \]
\item The first remaining number, 5, is a prime: record it and remove
  all multiples, giving
  \[ 7,11,13,17,19,23,29,\ldots \]
\end{enumerate}

\Level 1 {Arrays implementation}
\label{sec:arraysieve}

The sieve is easily implemented with an array that stores all integers.

\begin{exercise}
  Read in an integer that denotes the largest number you want to test.
  Make an array of integers that long. Set the elements to the
  successive integers.
\end{exercise}

\Level 1 {Streams implementation}
\label{sec:streamsieve}

The disadvantage of using an array is that we need to allocate
an array. What's more, the size is determined by how many integers we
are going to test, not how many prime numbers we want to generate.
We are going to take the idea above of having a generator object, and
apply that to the sieve algorithm: we will now have multiple generator
objects, each taking the previous as input and erasing certain
multiples from it.

\begin{exercise}
  Write a \n{stream} class that generates integers and use it through
  a pointer.
  %
  \answerwithoutput{streamints}{sieve}{ints}
\end{exercise}

Next, we need a stream that takes another stream as input, and filters
out values from it.

\begin{exercise}
  Write a class \n{filtered_stream} with a constructor
\begin{lstlisting}
filtered_stream(int filter,shared_ptr<stream> input);
\end{lstlisting}
  that
  \begin{enumerate}
  \item Implements \n{next}, giving filtered values,
  \item by calling the \n{next} method of the input stream and
    filtering out values.
  \end{enumerate}
  %
  \answerwithoutput{streamodds}{sieve}{odds}
\end{exercise}

Now you can implement the Eratosthenes sieve by making a
\n{filtered_stream} for each prime number.

\begin{exercise}
  Write a program that generates prime numbers as follows.
  \begin{itemize}
  \item Maintain a \n{current} stream, that is initially the stream of
    prime numbers.
  \item Repeatedly:
    \begin{itemize}
    \item Record the first item from the current stream, which is a
      new prime number;
    \item and set \n{current} to a new stream that takes \n{current}
      as input, filtering out multiples of the prime number just found.
    \end{itemize}
  \end{itemize}
\end{exercise}

\Level 0 {Range implementation}

\prerequisite{\ref{sec:range-iter}}

\begin{exercise}
  \label{ex:primerange}
  Make a \n{primes} class that can be ranged:
  %
  \answerwithoutput{primerangecall}{primes}{range}
\end{exercise}
