% -*- latex -*-
%%%%%%%%%%%%%%%%%%%%%%%%%%%%%%%%%%%%%%%%%%%%%%%%%%%%%%%%%%%%%%%%
%%%%
%%%% This TeX file is part of the course
%%%% Introduction to Scientific Programming in C++/Fortran2003
%%%% copyright 2017-9 Victor Eijkhout eijkhout@tacc.utexas.edu
%%%%
%%%% projectstyle.tex : style guide for submissions
%%%%
%%%%%%%%%%%%%%%%%%%%%%%%%%%%%%%%%%%%%%%%%%%%%%%%%%%%%%%%%%%%%%%%

Your project writeup is at least as important as the code.  Here are
some common-sense guidelines for a good writeup. However, not all
parts may apply to your project. Use your good judgement.

\heading{Style}

First of all, observe correct spelling and grammar.
Use full sentences.

\heading{Completeness}

Your writeup needs to have the same elements as a good paper:
\begin{itemize}
\item Title and author, including EID.
\item A one-paragraph abstract.
\item A~bibliography at the end.
\end{itemize}

\heading{Introduction}

The reader of your document should not be familiar with the project
description, or even the problem it addresses.  Indicate what the
problem is, give theoretical background if appropriate, possibly
sketch a historic background, and describe in global terms how you set
out to solve the problem, as well as your findings.

\heading{Code}

Your report should describe in a global manner the algorithms you
developed, and you should include relevant code snippets. If you want
to include full listings, relegate that to an appendix: code snippets
should illustrate especially salient points.

Do not use screen shots of your code: at the very least use a
\n{verbatim} environment, but using the \n{listings} package is very
much recommended.

\heading{Results and discussion}

Present tables and/or graphs when appropriate, but also include
verbiage to explain what conclusions can be drawn from them.

You can also discuss possible extensions of your work to cases not covered.

\heading{Summary}

Summarize your work and findings.


