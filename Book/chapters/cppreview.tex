% -*- latex -*-
%%%%%%%%%%%%%%%%%%%%%%%%%%%%%%%%%%%%%%%%%%%%%%%%%%%%%%%%%%%%%%%%
%%%%
%%%% This TeX file is part of the course
%%%% Introduction to Scientific Programming in C++/Fortran2003
%%%% copyright 2017-9 Victor Eijkhout eijkhout@tacc.utexas.edu
%%%%
%%%% cppreview.tex : review questions over C++
%%%%
%%%%%%%%%%%%%%%%%%%%%%%%%%%%%%%%%%%%%%%%%%%%%%%%%%%%%%%%%%%%%%%%


\Level 0 {Arithmetic}

\begin{enumerate}
\item
  Given
\begin{verbatim}
int n;
\end{verbatim}
write code that
uses elementary mathematical operators to compute n-cubed: $n^3$.

Do you get the correct result for all~$n$? Explain.
\item What is the output of:
\begin{verbatim}
int m=32, n=17;
cout << n%m << endl;
\end{verbatim}
\end{enumerate}

\Level 0 {Looping}

\begin{enumerate}
\item Suppose a function
\begin{verbatim}
bool f(int);
\end{verbatim}
is given, which is true for some positive input value. Write a main program that
finds the smallest positive input value for which \n{f} is true.
\item Suppose a function
\begin{verbatim}
bool f(int);
\end{verbatim}
is given, which is true for some negative input value. Write a main program that
finds the (negative) input with smallest absolute value for which \n{f} is true.
\end{enumerate}

\Level 0 {Functions}

\begin{exercise}
  \label{ex:iter-recurrence}
  The following code snippet computes in a loop the recurrence
  \[ v_{i+1} = av_i+b,\qquad \hbox{$v_0$ given.} \]
  Write a recursive function
\begin{lstlisting}
float v = value_n(n,a,b,v0);
\end{lstlisting}
  that computes the value~$v_n$ for $n\geq0$.
\end{exercise}

\Level 0 {Vectors}

\begin{exercise}
  \label{ex:lotsatypos}
The following program has several syntax and logical errors. The
intended purpose is to read an integer~$N$, and sort the integers
$1,\ldots,N$ into two vectors, one for the odds and one for the evens.
The odds should then be multiplied by two.

Your assignment is to debug this program. For 10 points of credit,
find 10 errors and correct them. Extra errors found will count as
bonus points. For logic errors, that is, places that are syntactically
correct, but still `do the wrong thing', indicate in a few words the problem with
the program logic.

\begin{lstlisting}
#include <iostream>
using std::cout; using std:cin;
using std::vector;

int main() {
  vector<int> evens,odd;
  cout << "Enter an integer value " << endl;
  cin << N;
  for (i=0; i<N; i++) {
    if (i%2=0) {
      odds.push_back(i);
    else
      evens.push_back(i);
  }
  for ( auto o : odds )
    o /= 2
  return 1
}
\end{lstlisting}
\end{exercise}

\Level 0 {Vectors}

\begin{exercise}
  \label{ex:vector-recurrence}
  Take another look at exercise~\ref{ex:iter-recurrence}. Now assume
  that you want to save the values~$v_i$ in an array
  \lstinline{vector<float> values}. Write code that does that, using
  first the iterative, then the recursive computation. Which do you prefer?
\end{exercise}

\Level 0 {Objects}

\begin{exercise}
  \label{ex:pointset-add}
  Let a class \lstinline{Point} class be given. 
  How would you design a class \lstinline{SetOfPoints} (which models a set of
  points) so that you could write

\begin{lstlisting}
Point p1,p2,p3;
SetOfPoints pointset;
// add points to the set:
pointset.add(p1); pointset.add(p2);
\end{lstlisting}

 Give the relevant data members and methods of the class.
\end{exercise}


\begin{exercise}
  You are programming a video game. There are moving elements, and you
  want to have an object for each. Moving elements need to have a
  method \n{move} with an argument that indicates a time duration, and
  this method updates the position of the element, using the speed of
  that object and the duration.

  Supply the missing bits of code.
\begin{verbatim}
class position {
  /* ... */
public:
  position() {};
  position(int initial) { /* ... */ };
  void move(int distance) { /* ... */ };
};
class actor {
protected:
  int speed;
  position current;

public:
  actor() { current = position(0); };
  void move(int duration) {
    /* THIS IS THE EXERCISE: */
    /* write the body of this function */
  };
};
class human : public actor {
public:
  human() // EXERCISE: write the constructor
};
class airplane : public actor {
public:
  airplane() // EXERCISE: write the constructor
};

int main() {
  human Alice;
  airplane Seven47;
  Alice.move( 5 );
  Seven47.move( 5 );
\end{verbatim}
\end{exercise}

