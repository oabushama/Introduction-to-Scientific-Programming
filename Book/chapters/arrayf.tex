
\begin{exercise}
  \label{ex:fmatmul}
  Compare implementations of the matrix-matrix product.
  \begin{enumerate}
  \item Write the regular \n{i,j,k} implementation, and store it as
    reference.
  \item Use the \n{DOT_PRODUCT} function, which eliminates the \n{k}
    index. How does the timing change? Print the maximum absolute
    distance between this and the reference result.
  \item Use the \n{MATMUL} function. Same questions.
  \item Bonus question: investigate the \n{j,k,i} and \n{i,k,j}
    variants. Write them both with array sections and individual array
    elements. Is there a difference in timing?
  \end{enumerate}
  Does the optimization level make a difference in timing?
\end{exercise}

\Level 1 {Restricting with \tt{where}}

\indextermfort{where}
\begin{verbatim}
where ( A<0 ) B = 0
\end{verbatim}

Full form:
\begin{verbatim}
WHERE ( logical argument )
  sequence of array statements
ELSEWHERE
  sequence of array statements
END WHERE
\end{verbatim}

\Level 1 {Global condition tests}

Reduction of a test on all array elements:
\indextermfort{all}
\begin{verbatim}
REAL(8),dimension(N,N) :: A
LOGICAL :: positive,positive_row(N),positive_col(N)
positive = ALL( A>0)
positive_row = ALL( A>0,1 )
positive_col = ALL( A>0,2 )
\end{verbatim}

\begin{exercise}
  Use array statements (that is, no loops) to fill a two-dimensional
  array~\n{A} with random numbers between zero and~one. Then fill two
  arrays \n{Abig} and \n{Asmall} with the elements of~\n{A} that are
  great than~$0.5$, or less than~$0.5$ respectively:
  \[ A_{\scriptstyle\mathrm{big}}(i,j) =
  \begin{cases}
    A(i,j)&\hbox{if $A(i,j)\geq 0.5$}\\ 0&\hbox{otherwise}
  \end{cases}
  \]
  \[ A_{\scriptstyle\mathrm{small}}(i,j) =
  \begin{cases}
    0&\hbox{if $A(i,j)\geq 0.5$}\\ A(i,j)&\hbox{otherwise}
  \end{cases}
  \]
  Using more array statements, add \n{Abig} and \n{Asmall}, and test
  whether the sum is close enough to~\n{A}.
\end{exercise}

Similar to \n{all}, there is a function~\indextermfort{any} that tests
if any array element satisfies the test.
\begin{verbatim}
if ( Any( Abs(A-B)>
\end{verbatim}

\Level 0 {Array operations}

\Level 1 {Loops without looping}

In addition to ordinary do-loops, Fortran has mechanisms that save you
typing, or can be more efficient in some circumstances.
\begin{itemize}
\item Slicing: if your loop assigns to an array from another array,
  you can use slice notation:
\begin{verbatim}
a(:) = b(:)
c(1:n) = d(2:n+1)
\end{verbatim}
\item The \indextermtt{forall} keyword also indicates an array assignment:
\begin{verbatim}
forall (i=1:n)
  a(i) = b(i)
  c(i) = d(i+1)
end forall
\end{verbatim}
You can tell that this is for arrays only, because the loop index has
to be part of the left-hand side of every assignment.

This mechanism is prone to misunderstanding and therefore now
deprecated.
It is not a parallel loop! For that, the following mechanism is preferred.
\item The \indexterm{do concurrent} is a true do-loop. With the
  \n{concurrent} keyword the user specifies that the
  iterations of a loop are independent, and can therefore possibly be
  done in parallel:
\begin{verbatim}
do concurrent (i=1:n)
  a(i) = b(i)
  c(i) = d(i+1)
end do
\end{verbatim}
\end{itemize}

\Level 1 {Loops without dependencies}

Here are some illustrations of simple array copying with the above
mechanisms.

\verbatimsnippet{blockrecur}
\begin{verbatim}
Original     1   2   3   4   5   6   7   8   9  10
Recursive    0   2   4   6   8  10  12  14  16  18
\end{verbatim}

\verbatimsnippet{blockslice}
\begin{verbatim}
Original     1   2   3   4   5   6   7   8   9  10
Sliced       0   2   4   6   8  10  12  14  16  18
\end{verbatim}

\verbatimsnippet{blockforall}
\begin{verbatim}
Original     1   2   3   4   5   6   7   8   9  10
Forall       0   2   4   6   8  10  12  14  16  18
\end{verbatim}

\verbatimsnippet{blockconc}
\begin{verbatim}
Original     1   2   3   4   5   6   7   8   9  10
Concurrent   0   2   4   6   8  10  12  14  16  18
\end{verbatim}

\begin{exercise}
  Create arrays \n{A,C} of length~\n{2N}, and \n{B}~of length~\n{N}.
  Now implement
  \[ B_i = (A_{2i}+A_{2i+1})/2,\quad i=1,\ldots N \]
  and
  \[ C_i = A_{i/2},\quad i=1,\ldots 2N \]
  using all four mechanisms. Make sure you get the same result every time.
\end{exercise}

\Level 1 {Loops with dependencies}

For parallel execution of a loop, all iterations have to be independent.
This is not the case if the loop has a \indexterm{recurrence}, and in
this case, the `do concurrent' mechanism is not appropriate.
%
Here are the above four constructs, but applied to a loop with a dependence.
%
\verbatimsnippet{slicerecur}
%
\begin{verbatim}
Original   1   2   3   4   5   6   7   8   9  10
Recursiv   1   2   4   8  16  32  64 128 256 512
\end{verbatim}

The slicing version of this:
%
\verbatimsnippet{sliceslice}
%
\begin{verbatim}
Original   1   2   3   4   5   6   7   8   9  10
Sliced     1   2   4   6   8  10  12  14  16  18
\end{verbatim}
%
acts as if the right-hand side is saved in a temporary array, and
subsequently assigned to the left-hand side.

Using `forall' is equivalent to slicing:
%
\verbatimsnippet{sliceforall}
%
\begin{verbatim}
Original     1   2   3   4   5   6   7   8   9  10
Forall       1   2   4   6   8  10  12  14  16  18
\end{verbatim}

On the other hand, `do concurrent' does not use temporaries, so it is
more like the sequential version:
%
\verbatimsnippet{sliceconc}
%
\begin{verbatim}
Original     1   2   3   4   5   6   7   8   9  10
Concurrent   1   2   4   8  16  32  64 128 256 512
\end{verbatim}
Note that the result does not have to be equal to the sequential
execution: the compiler is free to rearrange the iterations any way it
sees fit.

\Level 0 {Review questions}

\begin{exercise}
  \label{ex:farray-assign-slice}
  Let the following declarations be given, and assume that all arrays
  are properly initialized:
\begin{verbatim}
real                   :: x
real, dimension(10)    :: a, b
real, dimension(10,10) :: c, d
\end{verbatim}

Comment on the following lines: are they legal, if so what do they do?
\begin{enumerate}
\item \n{a = b}
\item \n{a = x}
\item \n{a(1:10) = c(1:10)}
\end{enumerate}

How would you:
\begin{enumerate}
\item Set the second row of \n{c} to~\n{b}?
\item Set the second row of \n{c} to the elements of~\n{b}, last-to-first?
\end{enumerate}
\end{exercise}
