% -*- latex -*-
%%%%%%%%%%%%%%%%%%%%%%%%%%%%%%%%%%%%%%%%%%%%%%%%%%%%%%%%%%%%%%%%
%%%%
%%%% This TeX file is part of the course
%%%% Introduction to Scientific Programming in C++/Fortran2003
%%%% copyright 2017/8 Victor Eijkhout eijkhout@tacc.utexas.edu
%%%%
%%%% string.tex : text strings
%%%%
%%%%%%%%%%%%%%%%%%%%%%%%%%%%%%%%%%%%%%%%%%%%%%%%%%%%%%%%%%%%%%%%

\Level 0 {Characters}

\begin{block}{Characters and ints}
  \label{sl:int-char}
  \begin{itemize}
  \item Type \indextermttdef{char};
  \item represents `7-bit ASCII': printable and (some) unprintable
    characters.
  \item Single quotes: \n{char c = 'a'}
  \item Equivalent to (short) integer: \n{'x'-'a'} is distance \n{a--x}
  \end{itemize}
\end{block}

\begin{exercise}
  \label{ex:print-ichar}
  Write a program that accepts an integer $0\cdots26$ and prints the
  so-manieth letter of the alphabet.

  Extend your program so that if the input is negative, it prints the
  minus-so-manieth uppercase letter of the alphabet.
\end{exercise}

\Level 0 {Basic string stuff}
\label{sec:string}

\begin{block}{String declaration}
  \label{sl:string-declare}
\begin{verbatim}
#include <string>
using std::string;

// .. and now you can use `string'
\end{verbatim}
(Do not use the C legacy mechanisms.)
\end{block}

\begin{block}{String creation}
  \label{sl:string-create}
  A \indexterm{string} variable contains a string of characters.
\begin{verbatim}
string txt;
\end{verbatim}
You can initialize the string variable (use \n{-std=c++11}), or assign it dynamically:
\begin{verbatim}
string txt{"this is text"};
string moretxt("this is also text");
txt = "and now it is another text";
\end{verbatim}
\end{block}

\begin{block}{Concatenation}
  \label{sl:string-plus}
  Strings can be \emph{concatenated}\index{string!concatenation}:
\begin{verbatim}
txt = txt1+txt2;
txt += txt3;
\end{verbatim}
\end{block}

\begin{block}{String is like vector}
  \label{sl:string-vector}
  You can query the \emph{size}\index{string!size}:
\begin{verbatim}
int txtlen = txt.size();
\end{verbatim}
or use subscripts:
\begin{verbatim}
cout << "The second character is <<" << 
        txt[1] << ">>" << endl;
\end{verbatim}
\end{block}

\begin{block}{More vector methods}
  \label{sl:string-vector-methods}
  Other methods for the vector class apply: \n{insert}, \n{empty},
  \n{erase}, \n{push_back}, et cetera.

  \url{http://en.cppreference.com/w/cpp/string/basic_string}
\end{block}

\begin{exercise}
  \label{ex:printdigits}
  Write a function to print out the digits of a number: \n{156} should
  print \n{one five six}. Use a vector or array of strings, containing
  the names of the digits.

  Start by writing a program that reads a single digit and prints its name.

  For the full program it is easiest to generate the  digits last-to-first.
  Then figure out how to print them reversed.
\end{exercise}

\begin{exercise}
  \label{ex:printnumber}
  Write a function to convert an integer to a string: the input
  \n{205} should give \n{two hundred fifteen}, et cetera.
\end{exercise}

\begin{exercise}
  Write a pattern matcher, where a period~\n{.} matches any one
  character, and \n{x*} matches any number of~`\n{x}' characters.

  For example:
  \begin{itemize}
  \item The string \n{abc} matches \n{a.c} but \n{abbc} doesn't.
  \item The string \n{abbc} matches \n{ab*c}, as does \n{ac}, but
    \n{abzbc} doesn't.
  \end{itemize}
\end{exercise}

\Level 0 {Conversion}

\n{to_string}

