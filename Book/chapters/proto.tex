% -*- latex -*-
%%%%%%%%%%%%%%%%%%%%%%%%%%%%%%%%%%%%%%%%%%%%%%%%%%%%%%%%%%%%%%%%
%%%%
%%%% This TeX file is part of the course
%%%% Introduction to Scientific Programming in C++/Fortran2003
%%%% copyright 2017-9 Victor Eijkhout eijkhout@tacc.utexas.edu
%%%%
%%%% proto.tex : about prototypes and separate compilation
%%%%
%%%%%%%%%%%%%%%%%%%%%%%%%%%%%%%%%%%%%%%%%%%%%%%%%%%%%%%%%%%%%%%%

\Level 0 {Prototypes for functions}
\label{sec:proto}

In most of the programs you have written in this course, you put any
functions or classes above the main program, so that the compiler
could inspect the definition before it encountered the use. However,
the compiler does not actually need the whole definition, say of a
function: it is enough to know its name, the types of the input
parameters, and the return type.

Such a minimal specification of a function is known as function
\indextermdef{prototype}; for instance
\begin{lstlisting}
int tester(float);
\end{lstlisting}

\begin{block}{Prototypes and forward declarations, 1}
  \label{sl:forward-proto1}
  A first use of prototypes is \indexterm{forward declaration}.

  Some people like defining functions after the main:
\begin{multicols}{2}
\begin{lstlisting}
  int f(int);
  int main() {
    f(5);
  };
  int f(int i) {
    return i;
  }
\end{lstlisting}
\columnbreak
versus before:
\begin{lstlisting}
int f(int i) {
  return i;
}
int main() {
  f(5);
};
\end{lstlisting}
\end{multicols}
\end{block}

\begin{block}{Prototypes and forward declarations, 2}
  \label{sl:forward-proto2}
  You also need forward declaration for mutually recursive functions:
\begin{lstlisting}
int f(int);
int g(int i) { return f(i); }
int f(int i) { return g(i); }
\end{lstlisting}
\end{block}

Prototypes are useful if you spread your program over multiple
files. You would put your functions in one file
and the main program in another.

\begin{multicols}{2}  
\begin{lstlisting}
// file: def.cxx
int tester(float x) {
  .....
}
\end{lstlisting}
\vfill\columnbreak
\begin{lstlisting}
// file : main.cxx
int main() {
  int t = tester(...);
  return 0;
}
\end{lstlisting}
\end{multicols}

In this example a function \n{tester}
is defined in a different file from the main program, so we need to
tell \n{main} what the function looks like in order for the main
program to be compilable:

\begin{lstlisting}
// file : main.cxx
int tester(float);
int main() {
  int t = tester(...);
  return 0;
}
\end{lstlisting}

Splitting your code over multiple
files and using
\indextermsub{separate}{compilation} 
is good software engineering practice for a number of
reasons. 
\begin{enumerate}
\item If your code gets large, compiling only the necessary files cuts
  down on compilation time.
\item Your functions may be useful in other projects, by yourself or
  others, so you can reuse the code without cutting and pasting it
  between projects.
\item It makes it easier to search through a file without being
  distracted by unrelated bits of code.
\end{enumerate}

\begin{slide}{Prototypes for separate compilation}
\label{sl:separate-proto}
\begin{multicols}{2}  
\begin{lstlisting}
// file: def.cxx
int tester(float x) {
  .....
}
\end{lstlisting}
\vfill\columnbreak
\begin{lstlisting}
// file : main.cxx
int tester(float);

int main() {
  int t = tester(...);
  return 0;
}
\end{lstlisting}
\end{multicols}
\end{slide}

(However, you would not do things like this in practice. See 
section~\ref{sec:headerfile} about header files.)

\Level 1 {Separate compilation}
\index{object file|see{file, object}}

\begin{block}{Compiling and linking}
  \label{sl:compile-link}
  Your regular compile line
\begin{verbatim}
icpc -o yourprogram yourfile.cc
\end{verbatim}
  actually does two things: compilation, and linking. You can do those
  separately:
  \begin{enumerate}
  \item First you compile
\begin{verbatim}
icpc -c yourfile.cc
\end{verbatim}
  which gives you a file \n{yourfile.o}, a so-called
  \indextermsub{object}{file}; and
  \item Then you use the compiler as \indexterm{linker} to give you
    the \indextermsub{executable}{file}:
\begin{verbatim}
icpc -o yourprogram yourfile.o
\end{verbatim}
  \end{enumerate}
\end{block}

In this particular example you may wonder what the big deal is.
That will become clear if you have multiple source files: now you
invoke the compile line for each source, and you link them only once.

\begin{block}{Dealing with multiple files}
  \label{sl:link-multiple}
  Compile each file separately, then link:
\begin{verbatim}
icpc -c mainfile.cc
icpc -c functionfile.cc
icpc -o yourprogram mainfile.o functionfile.o
\end{verbatim}  
\end{block}

At this point, you should learn about the \indexterm{Make} tool for
managing your programming project.

\Level 1 {Header files}
\label{sec:headerfile}
\label{sec:hfile}

Even better than writing the prototype every time you need the
function is to have a \indexterm{header file}:

\begin{block}{Prototypes and header files}
  \label{sl:proto-header}
\begin{lstlisting}
// file: def.h
int tester(float);
\end{lstlisting}
The header file gets included both in the definitions file and the
main program:
\begin{multicols}{2}  
\begin{lstlisting}
// file: def.cxx
#include "def.h"
int tester(float x) {
  .....
}
\end{lstlisting}
\vfill\columnbreak
\begin{lstlisting}
// file : main.cxx
#include "def.h"

int main() {
  int t = tester(...);
  return 0;
}
\end{lstlisting}
\end{multicols}
What happens if you leave out the \lstinline$#include "def.h"$ in both cases?
\end{block}

Having a header file is an important safety measure:
\begin{itemize}
\item Suppose you change your function definition, changing its return
  type:
\item The compiler will complain when you compile the definitions
  file;
\item So you change the prototype in the header file;
\item Now the compiler will complain about the main program, so you
  edit that too.
\end{itemize}

It is necessary to include the header file in the main program. It is
not strictly necessary to include it in the definitions file, but
doing so means that you catch potential errors: if you change the
function definitions, but forget to update the header file, this is
caught by the compiler.

\begin{remark}
  By the way, why does that compiler even recompile the main program,
  even though it was not changed? Well, that's because you used a
  \indexterm{makefile}. See the tutorial.
\end{remark}
\begin{remark}
  Header files were able to catch more errors in~C than they do
  in~C++. With polymorphism of functions, it is no longer an error to
  have 
\begin{lstlisting}
// header.h
int somefunction(int);
\end{lstlisting}
and
\begin{lstlisting}
#include "header.h"

int somefunction( float x ) { .... }
\end{lstlisting}
\end{remark}

\Level 1 {C and C++ headers}

You have seen the following syntaxes for including header files:
\begin{lstlisting}
#include <header.h>
#include "header.h"
\end{lstlisting}
The first is typically used for system files, with the second
typically for files in your own project. There are some header files
that come from the C~standard library such as \n{math.h}; the
idiomatic way of including them in C++ is
\begin{lstlisting}
#include <cmath>
\end{lstlisting}

\Level 0 {Prototypes for class methods}

\begin{block}{Class prototypes}
  \label{sl:class-proto}
  Header file:
  \verbatimsnippet{classheaderdefine}

  Implementation file:
  \verbatimsnippet{classheaderimpl}
\end{block}

\begin{block}{Data members in proto}
  \label{sl:class-proto-private}
  Data members, even private ones, need to be in the header file:
\begin{lstlisting}
class something {
private:
  int localvar;
public:
  double somedo(vector);
};
\end{lstlisting}
Implementation file:
\begin{lstlisting}
double something::somedo(vector v) {
   .... something with v ....
   .... something with localvar ....
};
\end{lstlisting}
\end{block}

\begin{review}
  \label{rev:proto-c-cpp}
  For each of the following answer: is this a valid function
  definition or function prototype.
  \begin{itemize}
  \item \verb+int foo();+
  \item \verb+int foo() {};+
  \item \verb+int foo(int) {};+
  \item \verb+int foo(int bar) {};+
  \item \verb+int foo(int) { return 0; };+
  \item \verb+int foo(int bar) { return 0; };+
  \end{itemize}
\end{review}

\Level 0 {Header files and templates}

The use of \emph{templates}\index{templates!and separate compilation}
(see chapter~\ref{ch:template})
often make separate compilation impossible: in order to compile the
templated definitions the compiler needs to know with what types they
will be used.

\Level 0 {Namespaces and header files}

Namespaces
(see chapter~\ref{ch:namespace})
are convenient, but they carry a danger in that they may define
functions without the user of the namespace being aware of it.

Therefore, one should never put \n{using namespace} in a header
file.

\Level 0 {Global variables and header files}
\label{ex:globalvar}

If you have a variable that you want known everywhere, you can make it
a \indextermsub{global}{variable}:
\begin{lstlisting}
int processnumber;
void f() {
  ... processnumber ...
}
int main() {
  processnumber = // some system call
};
\end{lstlisting}
It is then defined in the main program and any functions defined in your program file.

Warning: it is tempting to define variables global but this is a
dangerous practice.

If your program has multiple files, you should not put `\n{int processnumber}'
in the other files, because that would create a new variable, that is
only known to the functions in that file. Instead use:
\begin{lstlisting}
extern int processnumber;
\end{lstlisting}
which says that the global variable \n{processnumber} is defined in
some other file.

What happens if you put that variable in a
%
\emph{header file}\index{header file!and global variables}%
\index{variable!global!in header file}%
? Since the
%
\emph{preprocessor}\index{preprocessor!and header files}%
\index{header file!treatment by preprocessor}
acts as if the header is textually inserted, this again leads to
a separate global variable per file. The solution then is more
complicated:
\begin{lstlisting}
//file: header.h
#ifndef HEADER_H
#define HEADER_H
#ifndef EXTERN
#define EXTERN extern
#fi
EXTERN int processnumber
#fi

//file: aux.cc
#include "header.h"

//file: main.cc
#define EXTERN
#include "header.h"
\end{lstlisting}
(This sort of preprocessor magic is discussed in chapter~\ref{ch:cpp}.)

This also prevents recursive inclusion of header files.

