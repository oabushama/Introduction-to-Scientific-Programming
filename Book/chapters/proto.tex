% -*- latex -*-
%%%%%%%%%%%%%%%%%%%%%%%%%%%%%%%%%%%%%%%%%%%%%%%%%%%%%%%%%%%%%%%%
%%%%
%%%% This TeX file is part of the course
%%%% Introduction to Scientific Programming in C++/Fortran2003
%%%% copyright 2017/8 Victor Eijkhout eijkhout@tacc.utexas.edu
%%%%
%%%% proto.tex : about prototypes and separate compilation
%%%%
%%%%%%%%%%%%%%%%%%%%%%%%%%%%%%%%%%%%%%%%%%%%%%%%%%%%%%%%%%%%%%%%

\Level 0 {Prototypes for functions}
\label{sec:proto}

In most of the programs you have written in this course, you put any
functions or classes above the main program, so that the compiler
could inspect the definition before it encountered the use. However,
the compiler does not actually need the whole definition, say of a
function: it is enough to know its name, the types of the input
parameters, and the return type.

Such a minimal specification of a function is known as function
\indextermdef{prototype}; for instance
\begin{verbatim}
int tester(float);
\end{verbatim}

A first use of prototypes is \indexterm{forward declaration}:
\begin{verbatim}
int f(int);
int g(int i) { return f(i); }
int f(int i) { return g(i); }
\end{verbatim}

Prototypes are useful if you spread your program over multiple
files. You would put your functions in one file
and the main program in another. Splitting your code over multiple
files is good software engineering practice for a number of
reasons. For instance, it decreases compilation
time: if you make a small change, only that file needs to be recompiled.

In this example a function \n{tester}
is defined in a different file from the main program, so we need to
tell \n{main} what the function looks like in order for the main
program to be compilable:
\begin{multicols}{2}  
\begin{verbatim}
// file: def.cxx
int tester(float x) {
  .....
}
\end{verbatim}
\vfill\columnbreak
\begin{verbatim}
// file : main.cxx
int tester(float);

int main() {
  int t = tester(...);
  return 0;
}
\end{verbatim}
\end{multicols}
(However, you would not do things like this in practice. See the next
section about header files.)

\Level 1 {Header files}
\label{sec:hfile}

Even better than writing the prototype every time you need the
function is to have a \indexterm{header file}:
\begin{verbatim}
// file: def.h
int tester(float);
\end{verbatim}
The header file gets included both in the definitions file and the
main program:
\begin{multicols}{2}  
\begin{verbatim}
// file: def.cxx
#include "def.h"
int tester(float x) {
  .....
}
\end{verbatim}
\vfill\columnbreak
\begin{verbatim}
// file : main.cxx
#include "def.h"

int main() {
  int t = tester(...);
  return 0;
}
\end{verbatim}
\end{multicols}

Having a header file is an important safety measure:
\begin{itemize}
\item Suppose you change your function definition, changing its return
  type:
\item The compiler will complain when you compile the definitions
  file;
\item So you change the prototype in the header file;
\item Now the compiler will complain about the main program, so you
  edit that too.
\end{itemize}

It is necessary to include the header file in the main program. It is
not strictly necessary to include it in the definitions file, but
doing so means that you catch potential errors: if you change the
function definitions, but forget to update the header file, this is
caught by the compiler.

\begin{remark}
  By the way, why does that compiler even recompile the main program,
  even though it was not changed? Well, that's because you used a
  \indexterm{makefile}. See the tutorial.
\end{remark}
\begin{remark}
  Header files were able to catch more errors in~C than they do
  in~C++. With polymorphism of functions, it is no longer an error to
  have 
\begin{verbatim}
// header.h
int somefunction(int);
\end{verbatim}
and
\begin{verbatim}
#include "header.h"

int somefunction( float x ) { .... }
\end{verbatim}
\end{remark}

\Level 1 {C and C++ headers}

You have seen the following syntaxes for including header files:
\begin{verbatim}
#include <header.h>
#include "header.h"
\end{verbatim}
The first is typically used for system files, with the second
typically for files in your own project. There are some header files
that come from the C~standard library such as \n{math.h}; the
idiomatic way of including them in C++ is
\begin{verbatim}
#include <cmath>
\end{verbatim}

\Level 0 {Global variables}
\label{ex:globalvar}

If you have a variable that you want known everywhere, you can make it
a \indextermsub{global}{variable}:
\begin{verbatim}
int processnumber;
void f() {
  ... processnumber ...
}
int main() {
  processnumber = // some system call
};
\end{verbatim}
It is then defined in the main program and any functions defined in your program file.

Warning: it is tempting to define variables global but this is a
dangerous practice.

If your program has multiple files, you should not put `\n{int processnumber}'
in the other files, because that would create a new variable, that is
only known to the functions in that file. Instead use:
\begin{verbatim}
extern int processnumber;
\end{verbatim}
which says that the global variable \n{processnumber} is defined in
some other file.

What happens if you put that variable in a
%
\emph{header file}\index{header file!and global variables}%
\index{variable!global!in header file}%
? Since the
%
\emph{preprocessor}\index{preprocessor!and header files}%
\index{header file!treatment by preprocessor}
acts as if the header is textually inserted, this again leads to
a separate global variable per file. The solution then is more
complicated:
\begin{verbatim}
//file: header.h
#ifndef HEADER_H
#define HEADER_H
#ifndef EXTERN
#define EXTERN extern
#fi
EXTERN int processnumber
#fi

//file: aux.cc
#include "header.h"

//file: main.cc
#define EXTERN
#include "header.h"
\end{verbatim}
(This sort of preprocessor magic is discussed in chapter~\ref{ch:cpp}.)

This also prevents recursive inclusion of header files.

\Level 0 {Prototypes for class methods}

\begin{block}{Class prototypes}
  \label{sl:class-proto}
  Header file:
\begin{verbatim}
class something {
public:
  double somedo(vector);
};
\end{verbatim}

Implementation file:
\begin{verbatim}
double something::somedo(vector v) {
   .... something with v ....
};
\end{verbatim}
Strangely, data members also go in the header file.
\end{block}

\Level 0 {Header files and templates}

The use of \emph{templates}\index{templates!and separate compilation}
often make separate compilation impossible: in order to compile the
templated definitions the compiler needs to know with what types they
will be used.

\Level 0 {Namespaces and header files}

Never put \n{using namespace} in a header file.
