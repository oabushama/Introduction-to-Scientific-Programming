% -*- latex -*-
%%%%%%%%%%%%%%%%%%%%%%%%%%%%%%%%%%%%%%%%%%%%%%%%%%%%%%%%%%%%%%%%
%%%%
%%%% This TeX file is part of the course
%%%% Introduction to Scientific Programming in C++/Fortran2003
%%%% copyright 2017/8 Victor Eijkhout eijkhout@tacc.utexas.edu
%%%%
%%%% function.tex : functions
%%%%
%%%%%%%%%%%%%%%%%%%%%%%%%%%%%%%%%%%%%%%%%%%%%%%%%%%%%%%%%%%%%%%%

A~\indexterm{function}
(or~\emph{subprogram}\index{subprogram|see{function}}) is a way to
abbreviate a block of code and replace it by a single line.

\begin{itemize}
\item Find a block of code that has a clearly identifiable function.
\item Turn this into a function: the function definition will contain
  that block, plus a header and (maybe) a return statement.
\item The function definition is placed before the main program.
\item The function is called by its name.
\end{itemize}

%1
\begin{slide}{Turn blocks of code into functions}
  \label{sl:function-intro}
  \begin{itemize}
  \item Code fragment with clear function:
  \item Turn into \emph{subprogram}: function \emph{definition}.
  \item Use by single line: function \emph{call}.
  \end{itemize}
\end{slide}

\begin{block}{Example}
  \label{sl:function-example}
  \begin{multicols}{2}
    \small
    The code for an odd/even test
\begin{verbatim}
for (int i=0; i<N; i++) {
  cout << i;
  if (i%2==0)
    cout << " is even";
  else
    cout << " is odd";
  cout << endl;
}
\end{verbatim}
\vfill\columnbreak
becomes
\begin{verbatim}
void report_evenness(int n) {
  cout << i;
  if (i%2==0)
    cout << " is even";
  else
    cout << " is odd";
  cout << endl;
}
...
int main() {
  ...
  for (int i=0; i<N; i++)
    report_evenness(i);
}
\end{verbatim}
  \end{multicols}
Code becomes more readable: introduce application terminology.
\end{block}

\begin{block}{Code reuse}
  \label{sl:function-reuse}
Example: multiple norm calculations:
  \begin{multicols}{2}
    \small
    Repeated code:
\begin{verbatim}
float s = 0;
for (int i=0; i<nx; i++)
  s += abs(x[i]);
cout << "Inf norm x: " << s << endl;
s = 0;
for (int i=0; i<ny; i++)
  s += abs(y[i]);
cout << "Inf norm y: " << s << endl;
\end{verbatim}
\vfill\columnbreak
becomes:
\begin{verbatim}
int InfNorm( float a[],int n) {
  float s = 0;
  for (int i=0; i<n; i++)
    s += abs(a[i]);
  return s;
}
int main() {
  ... // stuff
  cout << "Inf norm x: " << InfNorm(x,nx) << endl;
  cout << "Inf norm y: " << InfNorm(y,ny) << endl;
\end{verbatim}
  \end{multicols}
  Code becomes shorter, easier to maintain.\\
  (Don't worry about array stuff in this example)\\
  Introduces application terminology.
\end{block}

\Level 0 {Function definition and call}

\begin{slide}{Function definition and call}
  \label{sl:def-call}
  \begin{multicols}{2}
\begin{verbatim}
for (int i=0; i<N; i++) {
  cout << i;
  if (i%2==0)
    cout << " is even";
  else
    cout << " is odd";
  cout << endl;
}
\end{verbatim}
\columnbreak
\begin{verbatim}
void report_evenness(int n) {
  cout << n;
  if (n%2==0)
    cout << " is even";
  else
    cout << " is odd";
  cout << endl;
}
...
int main() {
  ...
  for (int i=0; i<N; i++)
    report_evenness(i);
}
\end{verbatim}
  \end{multicols}
\end{slide}

In the \indextermbus{function}{definition}:
\begin{itemize}
\item The keyword \indextermtt{void} indicates that the function does
  not give any results back to the main program.
\item The name \n{report_evenness} is picked by you.
\item The parenthetical \n{(int n)} is called the `argument list' or
  `parameter list': it
  says that the function takes an \n{int} as input. For purposes of
  the function, the int will have the name~\n{n}, but this is not
  necessarily the same as the name in the main program.
\item The `body' of the function, the code that is going to be
  executed, is enclosed in curly brackets.
\end{itemize}

The \indextermbus{function}{call} consists of
\begin{itemize}
\item The name of the function, and
\item In between parentheses, the value of the input argument.
\end{itemize}

In the previous example, the function had an input, and performed some
screen output. To have a function that takes part in the computation
of your program, you would write something like:
\begin{verbatim}
int my_computation(int i) {
   return i+3;
}
...
// in the main:
int result;
result = my_computation(5);
\end{verbatim}

Functions are defined before the main program, and used in that program:
Here is a program with a function that doubles its input:

\begin{block}{Program with function}
  \label{sl:fun-example}
  \small
\begin{verbatim}
#input <iostream>
using namespace std;

int double_this(int n) {
  int twice_the_input = 2*n;
  return twice_the_input;
}

int main() {
  int number = 3;
  cout << "Twice three is: " << 
    double_this(number) << endl;
  return 0;
}
\end{verbatim}
\end{block}

Functions can be motivated as making your code more structured and intelligible.
The source where you use the function call becomes shorter,
and the function
name makes the code more descriptive. This is sometimes called
`self-documenting code'.

Sometimes introducing a function can be motivated from a point of
\indexterm{code reuse}: if the same block of code appears in two
places in your source, your replace this by one function definition,
and two (single line) function calls.
The two occurences of the function code do not have to be identical:


\begin{block}{Code reuse}
  \label{sl:reuse}
\begin{verbatim}
double x,y, v,w;
y = ...... computation from x .....
w = ...... same computation, but from v .....
\end{verbatim}
can be replaced by
\begin{verbatim}
double computation(double in) {
  return .... computation from `in' ....
}

y = computation(x);
w = computation(v);
\end{verbatim}
\end{block}

A final argument for using functions is code maintainability:
\begin{itemize}
\item Easier to debug: if you use the same (or roughly the same) block
  of code twice, and you find an error, you need to fix it twice.
\item Maintainance: if a block occurs twice, and  you change something in such a block
  once, you have to remember to change the other occurrences too.
\item Localization: any variables that only serve the calculation in
  the function now have a limited \indexterm{scope}.
\begin{verbatim}
void print_mod(int n,int d) {
  int m = n%d;
  cout << "The modulus of " << n << " and " << d 
       << " is " << m << endl;
\end{verbatim}
\end{itemize}

\begin{slide}{Why functions?}
  \label{sl:func-why}
  \begin{itemize}
  \item Easier to read
  \item Shorter code: reuse
  \item Cleaner code: local variables are no longer in the main program.
  \item Maintainance and debugging
  \end{itemize}
\end{slide}

Loosely, a function takes input and computes some result which is then returned.
Formally, a~function consists of:
\begin{itemize}
\item \indextermbusdef{function}{result type}: you need to indicate
  the type of the result;
\item name: you get to make this up;
\item zero or more \indextermbus{function}{parameters} or
  \indextermbus{function}{arguments}: the data that the function
  operates on; and the
\item \indextermbusdef{function}{body}: the statements that make up
  the function. The function body is a \emph{scope}: it can have local
  variables. (You can not nest function definitions.)
\item a \indextermdef{return} statement. Which doesn't have to be
  the last statement, by the way.
\end{itemize}

\begin{slide}{Anatomy of a function definition}
  \label{sl:func-anatomy}
  \begin{itemize}
  \item Result type: what's computed. \n{void} if no result
  \item Name: make it descriptive.
  \item Arguments: zero or more.\\
    \n{int i,double x,double y}
  \item Body: any length. This is a scope.
  \item Return statement: usually at the end, but can be anywhere; the
    computed result.
  \end{itemize}
\end{slide}

The function can then be used in the main program, or in another function:
\begin{block}{Function call}
  \label{sl:func-call}
  The function call
  \begin{enumerate}
  \item copies the value of the \indextermbus{function}{argument}
    to the \indextermbus{function}{parameter};
  \item causes the function body to be executed, and
  \item the function call is replaced by whatever you \n{return}.
  \item (If the function does not return anything, for instance because
    it only prints output, you declare the return type to be \indextermtt{void}.)
  \end{enumerate}
\end{block}

\begin{block}{Functions without input, without return result}
  \label{sl:func-ex1}
\begin{verbatim}
void print_header() {
  cout << "****************" << endl;
  cout << "* Output       *" << endl;
  cout << "****************" << endl;
  }
int main() {
  print_header();
  cout << "The results for day 25:" << endl;
  // code that prints results ....
  return 0;
}
\end{verbatim}
\end{block}

\begin{block}{Functions with input}
  \label{sl:func-ex2}
\begin{verbatim}
void print_header(int day) {
  cout << "****************" << endl;
  cout << "* Output       *" << endl;
  cout << "****************" << endl;
  cout << "The results for day " << day << ":" << endl;
  }
int main() {
  print_header(25);
  // code that prints results ....
  return 0;
}
\end{verbatim}
\end{block}

\begin{block}{Functions with return result}
  \label{sl:func-return}
\begin{verbatim}
#include <cmath>
double pi() {
  return 4*atan(1.0);
}
\end{verbatim}
The \n{atan} is a \indextermsub{standard}{function}
\end{block}

A function body defines a
%
\emph{scope}\index{scope!of function body}%
\index{function!defines scope}:
the local variables of the function calculation are invisible to the
calling program.

Functions can not be nested: you can not define a function inside the
body of another function.

\begin{exercise}
  \label{ex:newton-root}
  Early computers had no hardware for computing a square
  root. Instead, they used \indexterm{Newton's method}. Suppose you
  want to compute \[ x=\sqrt{y}. \] This is equivalent to finding the
  zero of \[ f_y(x) = x^2-y. \]
  Newton's method does this by evaluating \[ x_{\mathrm{next}}=x-f_y(x)/f_y'(x) \]
  until the guess is accurate enough.
  \begin{itemize}
  \item Write functions \n{f(x,y)} and \n{deriv(x,y)}, that compute
    $f_y(x)$ and $f_y'(x)$.
  \item Read a value $y$ and iterate until $|\n{f(x,y)}|<10^{-5}$. Print~$x$.
  \item Second part: write a function \n{newton_root} that computes~$\sqrt{y}$.
  \end{itemize}
\end{exercise}

\Level 0 {Parameter passing}
\label{sec:passing}
\index{parameter|seealso{function, parameter}}

C++~functions resemble mathematical functions:
you have seen that a function can have an input and an
output. In fact, they can have multiple inputs.

We start by studying functions that look like these mathematical
functions. They involve a \indextermbus{parameter}{passing} mechanism
called
\emph{passing by value}\index{parameter!pass by value}.
%
Later we will then look at
\emph{passing by reference}\index{parameter!pass by reference}.

\Level 1 {Pass by value}

The following style of programming is very much inspired by
mathematical functions, and is known as \indextermdef{functional programming}:
\begin{itemize}
\item A function has one result, which is returned through a return
  statement. The function call then looks like
\begin{verbatim}
y = f(x1,x2,x3);
\end{verbatim}
\item The definition of the C++ parameter passing mechanism says that
  input arguments are copied to the function, meaning that they don't
  change in the calling program:
  %
  \snippetwithoutput{passvalue}{func}{passvalue}

  We say that the input argument is
  \emph{passed by value}\index{parameter!pass by value}.
  In this example, the function parameter \n{x} acts as a local
  variable in the function, and it is initialized with a copy fo the
  value of \n{number} in the main program.
\end{itemize}

\begin{slide}{Mathematical type function}
  \label{sl:func-functional}
  Pretty good design:
  \begin{itemize}
  \item pass data into a function,
  \item return result through \n{return} statement.
  \item Parameters are copied into the function.
%  \item Mark parameters as \n{const}, just in case:
  \item \indexterm{pass by value}
  \item `functional programming'
  \end{itemize}
\end{slide}
\begin{slide}{Functional programming example}
  \label{sl:func-functional-ex}
  \snippetwithoutput{passvalue}{func}{passvalue}
\end{slide}

Passing a variable to a routine passes the value; in the routine, the
variable is local.

\verbatimsnippet{localparm}

You can local changes:

\verbatimsnippet{readonlyparm}

\Level 1 {Pass by reference}
\label{sec:pass-by-ref}
\index{pass by reference|see{parameter, pass by reference}}
  
Having only one output is a limitation on functions. Therefore there
is a mechanism for altering the input parameters and returning
(possibly multiple) results that way. You do this by not copying
values into the function parameters, but by turning the function
parameters into aliases of the variables at the place where the
function is called.

We need the concept of a \indextermdef{reference}:
\begin{block}{Reference}
  \label{sl:cpp-reference}
  A reference indicated with an ampersand, and it acts as an alias of
  the thing it references.
  %
  \snippetwithoutput{refint}{basic}{ref}
  %
  (You will not use references often this way.)
\end{block}

You can make a function parameter into a reference of a variable in
the main program:

\begin{block}{Parameter passing by reference}
  \label{sl:pass-by-ref}
The function parameter \n{n} becomes a reference to the variable \n{i}
in the main program:
\begin{verbatim}
void f(int &n) {
  n = /* some expression */ ;
};
int main() {
  int i;
  f(i);
  // i now has the value that was set in the function
}
\end{verbatim}
\end{block}

Using the ampersand, the parameter is
\emph{passed by reference}\index{pass by reference}:
instead of copying the value, the function receives a \emph{reference}:
information where the variable is stored in memory.

We also the following terminology for function parameters:
\begin{itemize}
\item \emph{input}\index{parameter!input|textbf} parameters: passed by
  value, so that it only functions as input to the function, and no
  result is output through this parameter;
\item \emph{output}\index{parameter!output|textbf} parameters: passed
  by reference so that they return an `output' value to the program.
\item \emph{throughput}\index{parameter!throughput|textbf} parameters:
  these are passed by reference, and they have an initial value when
  the function is called. In C++, unlike Fortran, there is no real
  separate syntax for these.
\end{itemize}

\begin{slide}{Results other than through return}
  \label{sl:func-err-return}
  Also good design:
  \begin{itemize}
  \item Return no function result,
  \item or return \indexterm{return status} (0~is success, nonzero various
    informative statuses), and
  \item return other information by changing the parameters.
  \item \emph{pass by reference}
  \item Parameters are also called `input', `output', `throughput'.
  \end{itemize}
\end{slide}

\begin{block}{Pass by reference example 1}
  \label{sl:pass-reference1}
  \snippetwithoutput{setbyref}{basic}{setbyref}
  Compare the difference with leaving out the reference.
\end{block}

As an example, consider a function that tries to read a value from a
file. With anything file-related, you always have to worry about the
case of the file not existing and such. So our function return:
\begin{itemize}
\item a boolean value to indicate whether the read succeeded, and
\item the actual value if the read succeeded.
\end{itemize}
The following is a common idiom, where the success value is returned
through the \n{return} statement, and the value through a parameter.

\begin{block}{Pass by reference example 2}
  \label{sl:pass-reference2}
\begin{verbatim}
bool can_read_value( int &value ) {
  int file_status = try_open_file();
  if (file_status==0) 
    value = read_value_from_file();
  return file_status!=0;
}

int main() {
  int n;
  if (!can_read_value(n))
    // if you can't read the value, set a default
    n = 10;
\end{verbatim}
\end{block}

\begin{exercise}
  \label{ex:swap}
  Write a function \n{swapij} of two parameters that exchanges the input values:
\begin{verbatim}
int i=2,j=3;
swapij(i,j);
// now i==3 and j==2
\end{verbatim}
\end{exercise}

\begin{exercise}
  \label{ex:div-remain}
  Write a function that tests divisibility and returns a remainder:

{\small
\begin{verbatim}
int number,divisor,remainder;
// get the number and divisor from the user
if ( is_divisible(number,divisor,remainder) )
  cout << number << " is divisible by " << divisor << endl;
else
  cout << number << "/" << divisor <<
      " has remainder " << remainder << endl;
\end{verbatim}
}
\end{exercise}

\Level 0 {Recursive functions}
\label{sec:recursion}

In mathematics, sequences are often recursively defined. For instance,
the sequence of factorials $n\mapsto f_n\equiv n!$ can be defined as
\[ f_0=1,\qquad \forall_{n>0}\colon f_n=n\times f_{n-1}. \]
Instead of using a subscript, we write an argument in parentheses
\[ F(n) = n \times F(n-1) \qquad \hbox{if $n>0$, otherwise~$1$} \]
and now it looks like a C++ function:
\begin{verbatim}
int factorial(int n)
\end{verbatim}
is the function header of a factorial function. So what would the
function body be? We need a \n{return} statement, and what we return
should be $n \times F(n-1)$:
\begin{verbatim}
int factorial(int n) {
  return n*factorial(n-1);
} // almost correct, but not quite
\end{verbatim}
So what happens if you write
\begin{verbatim}
int f3; f3 = factorial(3);
\end{verbatim}
Well,
\begin{itemize}
\item The expression \n{factorial(3)} calls the \n{factorial}
  function, substituting~\n{3} for the argument~\n{n}.
\item The return statement returns \n{n*factorial(n-1)}, in this case
  \n{3*factorial(2)}.
\item But what is \n{factorial(2)}? Evaluating that expression means
  that the \n{factorial} function is called again, but now with \n{n}
  equal to~2.
\item Evaluating \n{factorial(2)} returns \n{2*factorial(1)},\ldots
\item \ldots~which returns \n{1*factorial(0)},\ldots
\item \ldots~which return~\ldots
\item Uh oh. We forgot to include the case where \n{n} is zero. Let's
  fix that:
\begin{verbatim}
int factorial(int n) {
  if (n==0)
    return 1;
  else
    return n*factorial(n-1);
}
\end{verbatim}
\item Now \n{factorial(0)} is~1, so \n{factorial(1)} is
  \n{1*factorial(0)}, which is~1,\ldots
\item \ldots~so \n{factorial(2)} is~2, and \n{factorial(3)} is~6.
\end{itemize}

\begin{slide}{Recursion}
  \label{sl:func-recur}
  Functions are allowed to call themselves, which is known
  as~\indexterm{recursion}. You can define factorial as
  \[ F(n) = n \times F(n-1) \qquad \hbox{if $n>1$, otherwise~$1$} \]
\begin{verbatim}
int factorial( int n ) {
  if (n==1)
    return 1;
  else
    return n*factorial(n-1);
}
\end{verbatim}
\end{slide}

\begin{exercise}
  \label{ex:recur-sum}
  The sum of squares:
  \[ S_n = \sum_{n=1}^N n^2 \]
  can be defined recursively as
  \[ S_1=1,\qquad S_n = n^2 + S_{n-1}. \]
  Write a recursive function that implements this second definition.
  Test it on numbers that are input by the user.

  Then write a program that prints the first 100 sums of squares.
\end{exercise}

\begin{exercise}
  \label{ex:recur-fib}
  Write a recursive function for computing Fibonacci numbers:
  \[ F_0=1,\qquad F_1=1,\qquad F_{n}=F_{n-1}+F_{n-2} \]
  First write a program that computes $F_n$ for a value~$n$ that is
  input by the user.

  Then write a program that prints out a sequence of Fibonacci
  numbers; the user should input how many.
\end{exercise}

If you let your Fibonacci program print out each time it computes a
value, you'll see that most values are computed several times. (Math
question: how many times?) This is wasteful in running time. This
problem is addressed in section~\ref{sec:memo}.

\Level 0 {More function topics}

\Level 1 {Default arguments}

\begin{block}{Default arguments}
  \label{sl:def-arg}
  Functions can have \indextermsub{default}{argument}(s):
\begin{verbatim}
double distance( double x, double y=0. ) {
  return sqrt( (x-y)*(x-y) );
}
  ...
  d = distance(x); // distance to origin
  d = distance(x,y); // distance between two points
\end{verbatim}
Any default argument(s) should come last in the parameter list.
\end{block}

\Level 1 {Polymorphic functions}
\label{sec:polyfunc}

\begin{block}{Polymorphic functions}
  \label{sl:func-poly}
  You can have multiple functions with the same name:
\begin{verbatim}
double sum(double a,double b) {
  return a+b; }
double sum(double a,double b,double c) {
  return a+b+c; }
\end{verbatim}
Distinguished by type or number of input arguments: can not differ only in return type.
\end{block}

\Level 0 {Review questions}

\begin{exercise}
  \label{ex:cpp-funcloop1}
  Suppose a function
\begin{verbatim}
bool f(int);
\end{verbatim}
is given, which is true for some positive input value. Write a main program that
finds the smallest positive input value for which \n{f} is true.
\end{exercise}

\begin{exercise}
  \label{ex:cpp-funcloop2}
  Suppose a function
\begin{verbatim}
bool f(int);
\end{verbatim}
is given, which is true for some negative input value. Write a main program that
finds the (negative) input with smallest absolute value for which \n{f} is true.
\end{exercise}
