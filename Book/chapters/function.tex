
\Level 0 {Parameter passing}
\label{sec:passing}
\index{parameter|see{function, parameter}}

C++~functions resemble mathematical functions:
you have seen that a function can have an input and an
output. In fact, they can have multiple inputs.
The following style of programming is very much inspired by
mathematical functions, and is known as \indextermdef{functional programming}:
\begin{itemize}
\item A function has one result, which is returned through a return
  statement. The function call then looks like
\begin{verbatim}
y = f(x1,x2,x3);
\end{verbatim}
\item The definition of the C++ parameter passing mechanism says that
  input arguments are copied to the function, meaning that they don't
  change in the calling program:
  %
  \snippetwithoutput{passvalue}{func}{passvalue}

  We say that the input argument is
  \emph{passed by value}\index{parameter!pass by value}.
  In this example, the function parameter \n{x} acts as a local
  variable in the function, and it is initialized with a copy fo the
  value of \n{number} in the main program.
\end{itemize}

\begin{slide}{Mathematical type function}
  \label{sl:func-functional}
  Pretty good design:
  \begin{itemize}
  \item pass data into a function,
  \item return result through \n{return} statement.
  \item Parameters are copied into the function.
%  \item Mark parameters as \n{const}, just in case:
  \item \indexterm{pass by value}
  \end{itemize}
  \snippetwithoutput{passvalue}{func}{passvalue}
\end{slide}

Having only one output is a limitation on functions. Therefore there
is a mechanism for altering the input parameters and returning
(possibly multiple) results that way. To do this, you use an ampersand
for the parameter in the function definition:

\begin{block}{Parameter passing by reference}
  \label{sl:pass-by-ref}
\begin{verbatim}
void f(int &i) {
  i = /* some expression */ ;
};
int main() {
  int i;
  f(i);
  // i now has the value that was set in the function
}
\end{verbatim}
\end{block}

Using the ampersand, the parameter is
\emph{passed by reference}\index{pass by reference}:
instead of copying the value, the function receives a \emph{reference}:
information where the variable is stored in memory.

We also the following terminology for function parameters:
\begin{itemize}
\item \emph{input}\index{parameter!input|textbf} parameters: passed by
  value, so that it only functions as input to the function, and no
  result is output through this parameter;
\item \emph{output}\index{parameter!output|textbf} parameters: passed
  by reference so that they return an `output' value to the program.
\item \emph{throughput}\index{parameter!throughput|textbf} parameters:
  these are passed by reference, and they have an initial value when
  the function is called. In C++, unlike Fortran, there is no real
  separate syntax for these.
\end{itemize}

\begin{slide}{Results other than through return}
  \label{sl:func-err-return}
  Also good design:
  \begin{itemize}
  \item Return no function result,
  \item or return \index{return status} (0~is success, nonzero various
    informative statuses), and
  \item return other information by changing the parameters.
  \item \indexterm{pass by reference}
  \item Parameters are also called `input', `output', `throughput'.
  \end{itemize}
\end{slide}

\begin{block}{Pass by reference example 1}
  \label{sl:pass-reference0}
\begin{verbatim}
void f( int &i ) {
  i = 5;
}
int main() {
  int var = 0;
  f(var);
  cout << var << endl;
\end{verbatim}
Compare the difference with leaving out the reference.
\end{block}

As an example, consider a function that tries to read a value from a
file. With anything file-related, you always have to worry about the
case of the file not existing and such. So our function return:
\begin{itemize}
\item a boolean value to indicate whether the read succeeded, and
\item the actual value if the read succeeded.
\end{itemize}
The following is a common idiom, where the success value is returned
through the \n{return} statement, and the value through a parameter.

\begin{block}{Pass by reference example 2}
  \label{sl:pass-reference}
\begin{verbatim}
bool can_read_value( int &value ) {
  int file_status = try_open_file();
  if (file_status==0) 
    value = read_value_from_file();
  return file_status!=0;
}

int main() {
  int n;
  if (!can_read_value(n))
    // if you can't read the value, set a default
    n = 10;
\end{verbatim}
\end{block}

\begin{exercise}
  \label{ex:swap}
  Write a function \n{swapij} of two parameters that exchanges the input values:
\begin{verbatim}
int i=2,j=3;
swapij(i,j);
// now i==3 and j==2
\end{verbatim}
\end{exercise}

\begin{exercise}
  \label{ex:div-remain}
  Write a function that tests divisibility and returns a remainder:

{\small
\begin{verbatim}
int number,divisor,remainder;
// get the number and divisor from the user
if ( is_divisible(number,divisor,remainder) )
  cout << number << " is divisible by " << divisor << endl;
else
  cout << number << "/" << divisor <<
      " has remainder " << remainder << endl;
\end{verbatim}
}
\end{exercise}

\Level 0 {Recursive functions}
\label{sec:recursion}

In mathematics, sequences are often recursively defined. For instance,
the sequence of factorials $n\mapsto f_n\equiv n!$ can be defined as
\[ f_0=1,\qquad \forall_{n>0}\colon f_n=n\times f_{n-1}. \]
Instead of using a subscript, we write an argument in parentheses
\[ F(n) = n \times F(n-1) \qquad \hbox{if $n>0$, otherwise~$1$} \]
and now it looks like a C++ function:
\begin{verbatim}
int factorial(int n)
\end{verbatim}
is the function header of a factorial function. So what would the
function body be? We need a \n{return} statement, and what we return
should be $n \times F(n-1)$:
\begin{verbatim}
int factorial(int n) {
  return n*factorial(n-1);
} // almost correct, but not quite
\end{verbatim}
So what happens if you write
\begin{verbatim}
int f3; f3 = factorial(3);
\end{verbatim}
Well,
\begin{itemize}
\item The expression \n{factorial(3)} calls the \n{factorial}
  function, substituting~\n{3} for the argument~\n{n}.
\item The return statement returns \n{n*factorial(n-1)}, in this case
  \n{3*factorial(2)}.
\item But what is \n{factorial(2)}? Evaluating that expression means
  that the \n{factorial} function is called again, but now with \n{n}
  equal to~2.
\item Evaluating \n{factorial(2)} returns \n{2*factorial(1)},\ldots
\item \ldots~which returns \n{1*factorial(0)},\ldots
\item \ldots~which return~\ldots
\item Uh oh. We forgot to include the case where \n{n} is zero. Let's
  fix that:
\begin{verbatim}
int factorial(int n) {
  if (n==0)
    return 1;
  else
    return n*factorial(n-1);
}
\end{verbatim}
\item Now \n{factorial(0)} is~1, so \n{factorial(1)} is
  \n{1*factorial(0)}, which is~1,\ldots
\item \ldots~so \n{factorial(2)} is~2, and \n{factorial(3)} is~6.
\end{itemize}

\begin{slide}{Recursion}
  \label{sl:func-recur}
  Functions are allowed to call themselves, which is known
  as~\indexterm{recursion}. You can define factorial as
  \[ F(n) = n \times F(n-1) \qquad \hbox{if $n>1$, otherwise~$1$} \]
\begin{verbatim}
int factorial( int n ) {
  if (n==1)
    return 1;
  else
    return n*factorial(n-1);
}
\end{verbatim}
\end{slide}

\begin{exercise}
  \label{ex:recur-sum}
  The sum of squares:
  \[ S_n = \sum_{n=1}^N n^2 \]
  can be defined recursively as
  \[ S_1=1,\qquad S_n = n^2 + S_{n-1}. \]
  Write a recursive function that implements this second definition.
  Test it on numbers that are input by the user.

  Then write a program that prints the first 100 sums of squares.
\end{exercise}

\begin{exercise}
  \label{ex:recur-fib}
  Write a recursive function for computing Fibonacci numbers:
  \[ F_0=1,\qquad F_1=1,\qquad F_{n}=F_{n-1}+F_{n-2} \]
  First write a program that computes $F_n$ for a value~$n$ that is
  input by the user.

  Then write a program that prints out a sequence of Fibonacci
  numbers; the user should input how many.
\end{exercise}

If you let your Fibonacci program print out each time it computes a
value, you'll see that most values are computed several times. (Math
question: how many times?) This is wasteful in running time. This
problem is addressed in section~\ref{sec:memo}.

\Level 0 {More function topics}

\Level 1 {Default arguments}

\begin{block}{Default arguments}
  \label{sl:def-arg}
  Functions can have \indextermsub{default}{argument}(s):
\begin{verbatim}
double distance( double x, double y=0. ) {
  return sqrt( (x-y)*(x-y) );
}
  ...
  d = distance(x); // distance to origin
  d = distance(x,y); // distance between two points
\end{verbatim}
Any default argument(s) should come last in the parameter list.
\end{block}

\Level 1 {Polymorphic functions}
\label{sec:polyfunc}

\begin{block}{Polymorphic functions}
  \label{sl:func-poly}
  You can have multiple functions with the same name:
\begin{verbatim}
double sum(double a,double b) {
  return a+b; }
double sum(double a,double b,double c) {
  return a+b+c; }
\end{verbatim}
Distinguished by input parameters: can not differ only in return type.
\end{block}

\Level 0 {Review questions}

\begin{exercise}
  \label{ex:cpp-funcloop1}
  Suppose a function
\begin{verbatim}
bool f(int);
\end{verbatim}
is given, which is true for some positive input value. Write a main program that
finds the smallest positive input value for which \n{f} is true.
\end{exercise}

\begin{exercise}
  \label{ex:cpp-funcloop2}
  Suppose a function
\begin{verbatim}
bool f(int);
\end{verbatim}
is given, which is true for some negative input value. Write a main program that
finds the (negative) input with smallest absolute value for which \n{f} is true.
\end{exercise}
