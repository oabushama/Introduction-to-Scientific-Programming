% -*- latex -*-
%%%%%%%%%%%%%%%%%%%%%%%%%%%%%%%%%%%%%%%%%%%%%%%%%%%%%%%%%%%%%%%%
%%%%
%%%% This TeX file is part of the course
%%%% Introduction to Scientific Programming in C++11/Fortran2003
%%%% copyright 2017/8 Victor Eijkhout eijkhout@tacc.utexas.edu
%%%%
%%%% cpp4c.tex : C++ for C programmers
%%%%
%%%%%%%%%%%%%%%%%%%%%%%%%%%%%%%%%%%%%%%%%%%%%%%%%%%%%%%%%%%%%%%%

\documentclass[11pt]{boek3}

\usepackage{comment,listings,pdflscape,verbatim}
\makeatletter
\def\verbatim@startline{\verbatim@line{\leavevmode\kern\unitindent\relax}}
\makeatother

\usepackage[pdftex,colorlinks]{hyperref}
\usepackage{amssymb}
\usepackage[fleqn]{amsmath}
\usepackage{graphicx,undertilde,arydshln,wrapfig}
\usepackage{times,makeidx,multirow,multicol}
\usepackage{dirtree}

\input bookmacs

%%%%
%%%% Where is this course?
%%%%
\includecomment{tacc}

\makeindex

\begin{document}

\title{C++ for C Programmers}
\author{Victor Eijkhout}
\date{2018}
\maketitle

\tableofcontents

\Level 0 {Introduction}

The C++ language is, to first order of approximation, a~superset of
the C~language. Thus, many C programs are also legal C++
programs. C++~offers obvious improvement, such as object-oriented
programing. If you already understand C control structures and
functions, this booklet will teach you object oriented programming in
C~syntax.

However, viewing C++ as `C~with classes' does not lead to
idiomatic C++ programs. There are in fact a number of C~constructs
that should not be used. For instance, C++~has ways of dealing with
dynamic memory allocation that are both easier to use, and less
dangerous in that they take care of de-allocation for you.

Thus, this booklet will also teach what C~mechanisms not to use and
how to improve your code with their C++ replacements.

\CHAPTER{Input/output}{io}
\CHAPTER{Arrays}{array}
\CHAPTER{Strings}{string}
\CHAPTER{Parameter passing}{parampassing}
\CHAPTER{References}{address}
\CHAPTER{Classes and objects}{object}
\CHAPTER{Pointers}{pointer}
\CHAPTER{Namespaces}{namespace}
\CHAPTER{Templates}{template}
\CHAPTER{Closures}{lambda}
\CHAPTER{Error handling}{error}

\Level 0 {Index}
\begin{multicols}{2}
\printindex
\end{multicols}

\closeout\chapterlist
\end{document}
