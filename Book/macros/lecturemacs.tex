% -*- latex -*-
%%%%%%%%%%%%%%%%%%%%%%%%%%%%%%%%%%%%%%%%%%%%%%%%%%%%%%%%%%%%%%%%
%%%%
%%%% This TeX file is part of the course
%%%% Introduction to Scientific Programming in C++/Fortran2003
%%%% copyright 2017-9 Victor Eijkhout eijkhout@tacc.utexas.edu
%%%%
%%%% macros for lecture slides
%%%%
%%%%%%%%%%%%%%%%%%%%%%%%%%%%%%%%%%%%%%%%%%%%%%%%%%%%%%%%%%%%%%%%

\newif\ifInBook \InBookfalse
% redefine code location wrt lectures
\def\codedir{../../code}
% missing macro in beamer
\newdimen\unitindent
\unitindent=20pt

\usepackage{xifthen}

%%%%
%%%% Indexing, disabled.
%%%%
\newcommand\indexterm[1]{{#1}}
\newcommand\indextermdef[1]{{#1}}
\newcommand\indextermsub[2]{{#1 #2}}
\newcommand\indextermbus[2]{{#1 #2}}
\newcommand{\indextermsubdef}[2]{{#1 #2}}
\newcommand\indextermbusdef[2]{{#1 #2}}
\def\indextermtt{\bgroup \catcode`\_=12 \ttfamily \let\next=}
\let\indextermfort\indextermtt
\def\indextermttdef{\bgroup \catcode`\_=12 \ttfamily\slshape \let\next=}

\newcommand\inv{^{-1}}
\def\n#{\bgroup \catcode`\_=12 \catcode`\>=12 \catcode`\<=12
  \catcode`\&=12 \catcode`\^=12 \catcode`\~=12 \def\\{\char`\\}\relax
  \tt \let\next=}

\makeatletter
\renewcommand{\verbatim@font}{\ttfamily\footnotesize}
\makeatother

%%%%
%%%% Outlines
%%%%
\usepackage{outliner}
\OutlineLevelStart 0{\frame{\part{#1}\Large\bf #1}}
\OutlineLevelStart 1{\section{#1}}%{\frame{\section{#1}\Large\bf#1}}
\def\sectionframe#{\Level 0 }
\usepackage{framed}
\colorlet{shadecolor}{blue!15}
\OutlineLevelStart 2{\subsection{#1}
  \frame{\begin{shaded}\large #1\end{shaded}}}

%%%%
%%%% Exercise slides
%%%%
\newcounter{excounter}
\newcommand\exerciseslide
    [1]{\frame{
        \refstepcounter{excounter}
        \frametitle{Exercise \arabic{excounter}}
        \input #1
}}
\newcommand\optexerciseslide
    [1]{\frame{
        \refstepcounter{excounter}
        \frametitle{Optional exercise \arabic{excounter}}
        \input #1        
}}
\newcounter{revcounter}
\newcommand\reviewslide
    [1]{\frame{\frametitle{Review quiz
          \arabic{revcounter}}
        \input #1
        \stepcounter{revcounter}
}}
\newcommand\projectslide
    [1]{\frame{\frametitle{Project Exercise
          \arabic{excounter}}
        \input #1
        \stepcounter{excounter}
}}

\newenvironment{cnote}{\begin{quotation}\emph{C difference:\ }}{\end{quotation}}

%%%%
%%%% Acronyms
%%%%
\input acmacs
