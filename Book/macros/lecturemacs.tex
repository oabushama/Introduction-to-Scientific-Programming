% -*- latex -*-
%%%%%%%%%%%%%%%%%%%%%%%%%%%%%%%%%%%%%%%%%%%%%%%%%%%%%%%%%%%%%%%%
%%%%
%%%% This TeX file is part of the course
%%%% Introduction to Scientific Programming in C++/Fortran2003
%%%% copyright 2017/8 Victor Eijkhout eijkhout@tacc.utexas.edu
%%%%
%%%% macros for lecture slides
%%%%
%%%%%%%%%%%%%%%%%%%%%%%%%%%%%%%%%%%%%%%%%%%%%%%%%%%%%%%%%%%%%%%%

%%%%
%%%% Indexing, disabled.
%%%%
\newcommand\indexterm[1]{\emph{#1}}
\newcommand\indextermdef[1]{\emph{#1}}
\newcommand\indextermsub[2]{\emph{#1 #2}}
\newcommand\indextermbus[2]{\emph{#1 #2}}
\newcommand\indextermbusdef[2]{\emph{#1 #2}}
\def\indextermtt{\bgroup \catcode`\_=12 \ttfamily \let\next=}
\let\indextermfort\indextermtt
\def\indextermttdef{\bgroup \catcode`\_=12 \ttfamily\slshape \let\next=}

\newcommand\inv{^{-1}}
\def\n#{\bgroup \catcode`\_=12 \catcode`\>=12 \catcode`\<=12
  \catcode`\&=12 \catcode`\^=12 \catcode`\~=12 \def\\{\char`\\}\relax
  \tt \let\next=}

%%%%
%%%% Snippets
%%%%
\newif\ifInBook \InBookfalse
\input snippetmacs

\makeatletter
\renewcommand{\verbatim@font}{\ttfamily\tiny}
\makeatother

%% \newcommand{\snippetwithoutput}[3]{
%%   \begin{multicols}{2}
%%     \small\textbf{Code:}
%%     \verbatimsnippet{#1}
%%     \vfill\columnbreak
%%     \textbf{Output:}
%%     \immediate\write18{ cd ../../code/#2 && make #3 && make run_#3 > run.out }
%%     \verbatiminput{../../code/#2/run.out}
%%     \vfill\hbox{}\columnbreak
%%   \end{multicols}
%% }
%% \newcommand{\snippetwitherror}[3]{
%%   % go into vertical mode
%%   \par
%%   % make nice two-column layout
%%   \vbox{
%%   \begin{multicols}{2}
%%     \def\verbatim@startline{\verbatim@line{\leavevmode\relax}}
%%     \footnotesize\textbf{Code:}
%%     \verbatimsnippet{#1}
%%     \par\hbox{}\vfill\columnbreak
%%     \textbf{Output from running #3 in code directory #2:}
%%     \hbox{}
%%     \immediate\write18{ cd ../../code/#2 && make #3.o > #3.out 2>&1 }
%%     \verbatiminput{../../code/#2/#3.out}
%%     \par\hbox{}\vfill\hbox{}
%%   \end{multicols}
%%   }
%% }


%%%%
%%%% Outlines
%%%%
\usepackage{outliner}
\OutlineLevelStart 0{\frame{\part{#1}\Large\bf #1}}
\OutlineLevelStart 1{\section{#1}}%{\frame{\section{#1}\Large\bf#1}}
\def\sectionframe#{\Level 0 }
\usepackage{framed}
\colorlet{shadecolor}{blue!15}
\OutlineLevelStart 2{\subsection{#1}
  \frame{\begin{shaded}\large #1\end{shaded}}}

%%%%
%%%% Exercise slides
%%%%
\newcounter{excounter}
\newcommand\exerciseslide
    [1]{\frame{\frametitle{Exercise
          \arabic{excounter}}
        \input #1
        \stepcounter{excounter}
}}
\newcommand\optexerciseslide
    [1]{\frame{\frametitle{Optional exercise
          \arabic{excounter}}
        \input #1
        \stepcounter{excounter}
}}
\newcommand\projectslide
    [1]{\frame{\frametitle{Project Exercise
          \arabic{excounter}}
        \input #1
        \stepcounter{excounter}
}}

\newenvironment{cnote}{\begin{quotation}\emph{C difference:\ }}{\end{quotation}}

%%%%
%%%% Acronyms
%%%%
\input acmacs
