% -*- latex -*-
%%%%%%%%%%%%%%%%%%%%%%%%%%%%%%%%%%%%%%%%%%%%%%%%%%%%%%%%%%%%%%%%
%%%%
%%%% This TeX file is part of the course
%%%% Introduction to Scientific Programming in C++/Fortran2003
%%%% copyright 2017/8 Victor Eijkhout eijkhout@tacc.utexas.edu
%%%%
%%%% bookmacs.tex : macros for typsetting the source of the textbook
%%%%
%%%%%%%%%%%%%%%%%%%%%%%%%%%%%%%%%%%%%%%%%%%%%%%%%%%%%%%%%%%%%%%%

%%%%
%%%% Chapters (why are we doing this?)
%%%%
\newwrite\nx
\newwrite\chapterlist
\openout\chapterlist=chapternames.tex
\newcommand\CHAPTER[2]{
\Level 0 {#1}\label{ch:#2}
\global\chaptersourcelist{}
\def\chapshortname{#2}
{\SetBaseLevel 1 \input chapters/#2
 % section with source files
 \listchaptersources
 % add chapter to list of chapters
 \write\chapterlist{\chapshortname}
 \openout\nx=exercises/\chapshortname-nx.tex
 \write\nx{\arabic{excounter}}
 \closeout\nx
 \SetBaseLevel 0
}}

% each chapter has a list of sources
\newtoks\chaptersourcelist
\newcommand\addchaptersource[1]{
  \edef\temp{\global\chaptersourcelist={\the\chaptersourcelist #1}}\temp
}
\newcommand\listchaptersources{
  \expandafter\ChapterSourceHeader\the\chaptersourcelist\LSR
  %\tracingmacros=2 \tracingonline=1
  %\texttt{\the\chaptersourcelist}\par
  \expandafter\ListSourcesRecursively\the\chaptersourcelist\LSR
}
\def\LSR{\LSR}
\def\ChapterSourceHeader#1\LSR{
  \def\test{#1\LSR}
  \ifx\test\LSR
  \else
    \Level 0 {Sources used in this chapter}
  \fi
}
\def\ListSourcesRecursively#1{
  \def\test{#1}
  \ifx\test\LSR
  \else
    % list the file
    \textbf{Listing of code/#1}:
    {\footnotesize \verbatiminput{../code/#1.cxx}}
    {\footnotesize \verbatiminput{../code/#1.F90}}
    \par
    % continue
    \expandafter\ListSourcesRecursively
  \fi
}

%%%%
%%%% Page layout
%%%%
\usepackage{geometry}
\addtolength{\textwidth}{.5in}
\addtolength{\textheight}{.5in}
\addtolength{\evensidemargin}{-.5in}

\usepackage{fancyhdr}
\pagestyle{fancy}\fancyhead{}\fancyfoot{}
% remove uppercase from fancy defs
\makeatletter
\def\chaptermark#1{\markboth {{\ifnum \c@secnumdepth>\m@ne
 \thechapter. \ \fi #1}}{}}
\def\sectionmark#1{\markright{{\ifnum \c@secnumdepth >\z@
 \thesection. \ \fi #1}}}
\makeatother
% now the fancy specs
%\fancyhead[LE]{\thepage \hskip.5\unitindent/\hskip.5\unitindent \leftmark}
%\fancyhead[RO]{\rightmark \hskip.5\unitindent/\hskip.5\unitindent \thepage}
\fancyhead[LE]{\leftmark}
\fancyfoot[LE]{\thepage}
\fancyhead[RO]{\rightmark}
\fancyfoot[RO]{\thepage}
\fancyfoot[RE]{\footnotesize\sl Introduction to Scientific Programming}
\fancyfoot[LO]{\footnotesize\sl Victor Eijkhout}

%%%%
%%%% Outlining
%%%%
\usepackage{outliner}
\OutlineLevelStart0{\chapter{#1}}
\OutlineLevelStart1{\section{#1}}
\OutlineLevelCont1{\section{#1}}
\OutlineLevelStart2{\subsection{#1}}
\OutlineLevelStart3{\subsubsection{#1}}
\setcounter{secnumdepth}{4}
\OutlineLevelStart4{\paragraph{\bf #1}}

\newcommand\heading[1]{\paragraph*{\textbf{#1}}}

%%%%
%%%% Snippets
%%%%
\newif\ifInBook \InBooktrue
\input snippetmacs

%%%%
%%%% Text and slide blocks
%%%%

\newcounter{frcounter}[chapter]
\newcounter{blcounter}[chapter]
\makeatletter
\renewcommand\thefrcounter{\@arabic\c@chapter.\@arabic\c@frcounter}
\renewcommand\theblcounter{\@arabic\c@chapter.\@arabic\c@blcounter}
\makeatother

\newcommand\framenumber{\arabic{chapter}.\arabic{frcounter}}
\newcommand\blocknumber{\arabic{chapter}.\arabic{frcounter}}
%% block: write out a frame, and read back in
\generalcommentarg{block}
  {\refstepcounter{frcounter}%
    \begingroup
    \def\ProcessCutFile{}\par
    \def\PrepareCutFile{\immediate\write\CommentStream
      {\noexpand\begin{frame}[containsverbatim]{\CommentBlockName}%
        \noexpand\nobreak\noexpand\smallskip}}%
    \def\FinalizeCutFile{\immediate\write\CommentStream
      {\noexpand\end{frame}}}%
    \edef\tmp{\def\noexpand\CommentCutFile
      {frames/\chapshortname-block-\arabic{frcounter}.tex}}\tmp
  }
  {\input{\CommentCutFile}
   \endgroup
  }
%% slide: write out a frame, and don't read back in
\generalcommentarg{slide}
  {\refstepcounter{frcounter}%
    \begingroup
    \def\ProcessCutFile{}\par
    \def\PrepareCutFile{\immediate\write\CommentStream
      {\noexpand\begin{frame}[containsverbatim]{\CommentBlockName}\par}}%
    \def\FinalizeCutFile{\immediate\write\CommentStream
      {\noexpand\end{frame}}}%
    \edef\tmp{\def\noexpand\CommentCutFile
      {frames/\chapshortname-block-\arabic{frcounter}.tex}}\tmp
  }
  {\endgroup}

\renewenvironment{frame}[2][keyword]
                 {\textsl{#2}\par
                   %\begin{quote}
                   \iffalse \leavevmode
                     \hbox{\kern-\unitindent
                     \textsl{\blocknumber. #2 }\hspace{1em}}%
                   \raggedright \ignorespaces \fi
                 }
                 {}%{\end{quote}
\newenvironment{cnote}{\begin{quotation}\noindent\emph{C difference:\ }}{\end{quotation}}
\newenvironment{f77note}{\begin{quotation}\noindent\emph{F77 note:\ }}{\end{quotation}}
\newenvironment{advanced}{\begin{quotation}\noindent\emph{Advanced note:\ }}{\end{quotation}}

%%%%
%%%% More
%%%%

{\catcode`\^^I=13 \globaldefs=1
\newcommand\listing[2]{\begingroup\small\par\vspace{1ex}
  \catcode`\^^I=13 \def^^I{\leavevmode\hspace{40pt}}
  \noindent\fbox{#1}
  \verbatiminput{#2}\endgroup}
  \newcommand\codelisting[1]{\begingroup\small\par\vspace{1ex}
  \catcode`\^^I=13 \def^^I{\leavevmode\hspace{40pt}}
  \noindent\fbox{#1}
  \verbatiminput{#1}\endgroup}
}
\lstdefinestyle{reviewcode}{
  belowcaptionskip=1\baselineskip,
  breaklines=true,
  frame=L,
  xleftmargin=\unitindent,
  showstringspaces=false,
  basicstyle=\footnotesize\ttfamily,
  keywordstyle=\bfseries\color{blue},
  commentstyle=\color{red!60!black},
  identifierstyle=\slshape\color{black},
  stringstyle=\color{green!60!black},
  columns=fullflexible,
  keepspaces=true,
}

\newcommand\inv{^{-1}}\newcommand\invt{^{-t}}
\newcommand\setspan[1]{[\![#1]\!]}
\newcommand\fp[2]{#1\cdot10^{#2}}
\newcommand\fxp[2]{\langle #1,#2\rangle}
\def\n#{\bgroup \catcode`\_=12 \catcode`\>=12 \catcode`\<=12 \catcode`\#=12
  \catcode`\&=12 \catcode`\^=12 \catcode`\~=12 \def\\{\char`\\}\relax
  \tt \let\next=}

\newcommand\prerequisite[1]{
  \begin{quote}
    \textsl{Before doing this section, make sure you study section~#1.}
  \end{quote}
}

\newcommand\diag{\mathop{\mathrm {diag}}}
\newcommand\argmin{\mathop{\mathrm {argmin}}}
\newcommand\defined{
  \mathrel{\lower 5pt \hbox{${\equiv\atop\mathrm{\scriptstyle D}}$}}}
\newcommand\lulubreak{\message{Hard page break!}\pagebreak\relax}

\newcommand\bbP{\mathbb{P}}
\newcommand\bbR{\mathbb{R}}

\newtheorem{remark}{Remark}
\expandafter\ifx\csname definition\endcsname\relax
    \newtheorem{definition}{Definition}
\fi
\expandafter\ifx\csname theorem\endcsname\relax
    \newtheorem{theorem}{Theorem}
\fi
\expandafter\ifx\csname lemma\endcsname\relax
    \newtheorem{lemma}{Lemma}
\fi
\expandafter\ifx\csname proof\endcsname\relax
 \newenvironment{proof}{\begin{quotation}\small\sl\noindent Proof.\ \ignorespaces}
     {\end{quotation}}
\fi
%% \newenvironment{highermath}
%%     {\bigskip\begin{quotation}\noindent\emph{MMM}}
%%     {\end{quotation}\bigskip\noindent\ignorespaces}

\newcommand{\indexterm}[1]{\emph{#1}\index{#1}}
\newcommand{\indextermunix}[1]{{\ttfamily\slshape #1}\index{#1@\texttt{#1}}}
\newcommand{\indextermdef}[1]{\emph{#1}\index{#1|textbf}}
\newcommand{\indextermp}[1]{\emph{#1s}\index{#1}}
\newcommand{\indextermsub}[2]{\emph{#1 #2}\index{#2!#1}}
\newcommand{\indextermsubdef}[2]{\emph{#1 #2}\index{#2!#1|textbf}}
\newcommand{\indextermsubp}[2]{\emph{#1 #2s}\index{#2!#1}}
\newcommand{\indextermbus}[2]{\emph{#1 #2}\index{#1!#2}}
\newcommand{\indextermbusdef}[2]{\emph{#1 #2}\index{#1!#2|textbf}}
\newcommand{\indextermstart}[1]{\emph{#1}\index{#1|(}}
\newcommand{\indextermend}[1]{\index{#1|)}}
\newcommand{\indexstart}[1]{\index{#1|(}}
\newcommand{\indexend}[1]{\index{#1|)}}

{ \catcode`\_=13
\gdef\indexmpishow#{\bgroup \catcode`\_=13 \def_{\char95\discretionary{}{}{}}
  \catcode`\>=12 \catcode`\<=12
  \catcode`\&=12 \catcode`\^=12 \catcode`\~=12 \def\\{\char`\\}\relax
  \tt \afterassignment\mpitoindex\edef\indexedmpi}
\gdef\indexmpidef#{\bgroup \catcode`\_=13 \def_{\char95\discretionary{}{}{}}
  \catcode`\>=12 \catcode`\<=12
  \catcode`\&=12 \catcode`\^=12 \catcode`\~=12 \def\\{\char`\\}\relax
  \tt \afterassignment\mpitoindexbf\edef\indexedmpi}
}
\def\mpitoindex{%\tracingmacros=2
  \edef\tmp{\noexpand\n{\indexedmpi}%
            \noexpand\index{\indexedmpi@{\noexpand\tt{\indexedmpi}}}}%
  \tmp
  \egroup
}
\def\mpitoindexbf{%\tracingmacros=2
  \edef\tmp{\noexpand\n{\indexedmpi}%
            \noexpand\index{\indexedmpi@{\noexpand\tt{\indexedmpi}}|textbf}}%
  \tmp
  \egroup
}
\let\indextermtt\indexmpishow
\let\indextermttdef\indexmpidef

%%
%% Fortran special case
%%
{ \catcode`\_=13
\gdef\indextermfort#{\bgroup \catcode`\_=13 \def_{\char95\discretionary{}{}{}}
  \catcode`\>=12 \catcode`\<=12
  \catcode`\&=12 \catcode`\^=12 \catcode`\~=12 \def\\{\char`\\}\relax
  \tt \afterassignment\forttoindex\edef\indexedmpi}
}
\def\forttoindex{%\tracingmacros=2
  \edef\tmp{\noexpand\n{\indexedmpi}%
            \noexpand\index{\indexedmpi@{{\noexpand\tt\indexedmpi}
                (Fortran keyword)}}}%
  \tmp
  \egroup
}

\makeatletter
\newcommand\indexac[1]{\emph{\ac{#1}}%
  %\tracingmacros=2 \tracingcommands=2
  \edef\tmp{\noexpand\index{%
    \expandafter\expandafter\expandafter
        \@secondoftwo\csname fn@#1\endcsname%
    @\acl{#1} (#1)}}\tmp}
\newcommand\indexacp[1]{\emph{\ac{#1}}%
  %\tracingmacros=2 \tracingcommands=2
  \edef\tmp{\noexpand\index{%
    \expandafter\expandafter\expandafter
        \@secondoftwo\csname fn@#1\endcsname%
    @\aclp{#1} (#1)}}\tmp}
\newcommand\indexacf[1]{\emph{\acf{#1}}%
  \edef\tmp{\noexpand\index{%
    \expandafter\expandafter\expandafter
        \@secondoftwo\csname fn@#1\endcsname
    @\acl{#1} (#1)}}\tmp}
\newcommand\indexacstart[1]{%
  \edef\tmp{\noexpand\index{%
    \expandafter\expandafter\expandafter
        \@secondoftwo\csname fn@#1\endcsname
    @\acl{#1} (#1)|(}}\tmp}
\newcommand\indexacend[1]{%
  \edef\tmp{\noexpand\index{%
    \expandafter\expandafter\expandafter
        \@secondoftwo\csname fn@#1\endcsname
    @\acl{#1} (#1)|)}}\tmp}
\makeatother

\newenvironment{exercises}{\begin{enumerate}}{\end{enumerate}}
\newcommand{\project}{\item[Project]}
\excludecomment{quiz}

\newcounter{excounter}[chapter]
\newcounter{revcounter}[chapter]
\makeatletter
\renewcommand\theexcounter{\@arabic\c@chapter.\@arabic\c@excounter}
\renewcommand\therevcounter{\@arabic\c@chapter.\@arabic\c@revcounter}
\makeatother

\newcommand\exercisenumber{\arabic{chapter}.\arabic{excounter}}
\generalcomment{exercise}
  {\refstepcounter{excounter}%
   \begingroup\def\ProcessCutFile{}\par
   \edef\tmp{\def\noexpand\CommentCutFile
                 {exercises/\chapshortname-ex\arabic{excounter}.tex}}\tmp
   }
  {\begin{quote}
   \leavevmode
   \hbox{\kern-\unitindent 
         \textbf Exercise \exercisenumber.\hspace{1em}}%
     \ignorespaces
   \input{\CommentCutFile}
   \end{quote}
   \endgroup}
\newcommand\reviewnumber{\arabic{chapter}.\arabic{revcounter}}
\generalcomment{review}
  {\refstepcounter{revcounter}%
   \begingroup\def\ProcessCutFile{}\par
   \edef\tmp{\def\noexpand\CommentCutFile
                 {exercises/\chapshortname-rev\arabic{revcounter}.tex}}\tmp
   }
  {\begin{quote}
   \leavevmode
   \hbox{\kern-\unitindent 
         \textbf Review \reviewnumber.\hspace{1em}}%
     \ignorespaces
   \input{\CommentCutFile}
   \end{quote}
   \endgroup}

\newif\ifIncludeAnswers
\input qa

\makeatletter
\generalcomment{answer}
  {\begingroup
   \edef\tmp{\def\noexpand\CommentCutFile
                 {answers/\chapshortname-an\noexpand\arabic{excounter}.tex}}\tmp
   \def\ProcessCutFile{}}
  {\ifIncludeAnswers
   \leavevmode
   \begin{sl}\small
   \def\verbatim@startline{\verbatim@line{\leavevmode\relax}}%
   \hbox{\textbf Solution to exercise \arabic{chapter}.\arabic{excounter}.\hspace{1em}}%
     \ignorespaces\it
   \input{\CommentCutFile}
   \end{sl}\fi
   \endgroup}
\makeatother

\newenvironment{exerciseB}[1]
  {\par
   {\bfseries Problem \arabic{chapter}.\arabic{excounter}}
   \parbox[t]{2in}{\slshape #1}%
   \par
  }
  {\par}


\input acmacs
