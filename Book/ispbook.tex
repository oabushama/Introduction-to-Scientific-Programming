% -*- latex -*-
%%%%%%%%%%%%%%%%%%%%%%%%%%%%%%%%%%%%%%%%%%%%%%%%%%%%%%%%%%%%%%%%
%%%%
%%%% This TeX file is part of the course
%%%% Introduction to Scientific Programming in C++11/Fortran2003
%%%% copyright 2017-9 Victor Eijkhout eijkhout@tacc.utexas.edu
%%%%
%%%% ispbook.tex : main file of the textbook
%%%%
%%%%%%%%%%%%%%%%%%%%%%%%%%%%%%%%%%%%%%%%%%%%%%%%%%%%%%%%%%%%%%%%

\documentclass[11pt]{boek3}

\usepackage{comment,pdflscape,verbatim}
\makeatletter
\def\verbatim@startline{\verbatim@line{\leavevmode\kern\unitindent\relax}}
\makeatother

\usepackage[pdftex,colorlinks]{hyperref}
\usepackage{amssymb}
\usepackage[fleqn]{amsmath}
\usepackage{graphicx,undertilde,arydshln,wrapfig,xifthen}
\usepackage{times,makeidx,multirow,multicol}
\usepackage{dirtree}

\input bookmacs
\input snippetmacs

%%%%
%%%% Where is this course?
%%%%
\includecomment{tacc}

\makeindex

\begin{document}

\title{Introduction to Scientific Programming in C++/Fortran2003}
\author{Victor Eijkhout}
\date{2018}
\maketitle

\tableofcontents

\part{Introduction}

\CHAPTER{Introduction}{intro}
\CHAPTER{Warming up}{warmup}
\CHAPTER{Teachers guide}{teach}

\part{C++}
\label{part:cpp}
\lstset{language=C++,style=reviewcode}
\def\ISPcodeext{cxx}

\CHAPTER{Basic elements of C++}{elements}
\CHAPTER{Conditionals}{if}
\CHAPTER{Looping}{loop}
\CHAPTER{Functions}{function}
\CHAPTER{Scope}{scope}
\CHAPTER{Structures}{struct}
\CHAPTER{Classes and objects}{object}
\CHAPTER{Arrays}{array}
\CHAPTER{Strings}{string}
\CHAPTER{Input/output}{io}
\CHAPTER{References}{address}
%\CHAPTER{Polymorphism}{poly}
%\CHAPTER{Memory}{memory}
\CHAPTER{Pointers}{pointer}
\CHAPTER{C-style pointers and arrays}{cpointer}
\CHAPTER{Const}{const}
\CHAPTER{Prototypes}{proto}
\CHAPTER{Namespaces}{namespace}
\CHAPTER{Preprocessor}{cpp}
\CHAPTER{Templates}{template}
\CHAPTER{Error handling}{error}
\CHAPTER{Standard Template Library}{stl}
\CHAPTER{Obscure stuff}{obscure}
\CHAPTER{C++ for C programmers}{ccpp}

%\CHAPTER{More exercises}{exercises}

\part{Fortran}
\label{part:f}
\lstset{language=Fortran}
\def\ISPcodeext{F90}

\CHAPTER{Basics of Fortran}{elementsf}
\CHAPTER{Conditionals}{iff}
\CHAPTER{Loop constructs}{loopf}
\CHAPTER{Scope}{scopef}
\CHAPTER{Subprograms and modules}{functionf}
\CHAPTER{String handling}{stringf}
\CHAPTER{Structures, eh, types}{structf}
\CHAPTER{Modules}{modulef}
\CHAPTER{Classes and objects}{classf}
\CHAPTER{Arrays}{arrayf}
\CHAPTER{Pointers}{pointf}
\CHAPTER{Input/output}{iof}
\CHAPTER{Leftover topics}{obscuref}

\part{Exercises and projects}
\lstset{language=C++}

\CHAPTER{Exercises}{simplex}
\CHAPTER{Prime numbers}{prime}
\CHAPTER{Geometry}{geom}
\CHAPTER{Infectuous disease simulation}{infect}
\CHAPTER{PageRank}{google}
\CHAPTER{Redistricting}{gerry}
\CHAPTER{Amazon delivery truck scheduling}{amazon}
\CHAPTER{Memory allocation}{malloc}
\CHAPTER{DNA Sequencing}{dna}
\CHAPTER{Cryptography}{crypt}
\lstset{language=Fortran,style=reviewcode}
\CHAPTER{Climate change}{climate}
\lstset{language=C++,style=reviewcode}

\part{Advanced topics}

\lstset{language=C++}
\CHAPTER{Programming strategies}{topdown}
\CHAPTER{Tiniest of introductions to algorithms and data structures}{algodata}
\CHAPTER{Complexity}{efficiency}

\part{Index and such}

\Level 0 {Index}
\begin{multicols}{2}
\printindex
\end{multicols}

\bibliographystyle{plain}
\bibliography{vle}

\closeout\chapterlist
\end{document}
